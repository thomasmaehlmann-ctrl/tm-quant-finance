% Options for packages loaded elsewhere
% Options for packages loaded elsewhere
\PassOptionsToPackage{unicode}{hyperref}
\PassOptionsToPackage{hyphens}{url}
\PassOptionsToPackage{dvipsnames,svgnames,x11names}{xcolor}
%
\documentclass[
  letterpaper,
  DIV=11,
  numbers=noendperiod]{scrreprt}
\usepackage{xcolor}
\usepackage{amsmath,amssymb}
\setcounter{secnumdepth}{5}
\usepackage{iftex}
\ifPDFTeX
  \usepackage[T1]{fontenc}
  \usepackage[utf8]{inputenc}
  \usepackage{textcomp} % provide euro and other symbols
\else % if luatex or xetex
  \usepackage{unicode-math} % this also loads fontspec
  \defaultfontfeatures{Scale=MatchLowercase}
  \defaultfontfeatures[\rmfamily]{Ligatures=TeX,Scale=1}
\fi
\usepackage{lmodern}
\ifPDFTeX\else
  % xetex/luatex font selection
\fi
% Use upquote if available, for straight quotes in verbatim environments
\IfFileExists{upquote.sty}{\usepackage{upquote}}{}
\IfFileExists{microtype.sty}{% use microtype if available
  \usepackage[]{microtype}
  \UseMicrotypeSet[protrusion]{basicmath} % disable protrusion for tt fonts
}{}
\makeatletter
\@ifundefined{KOMAClassName}{% if non-KOMA class
  \IfFileExists{parskip.sty}{%
    \usepackage{parskip}
  }{% else
    \setlength{\parindent}{0pt}
    \setlength{\parskip}{6pt plus 2pt minus 1pt}}
}{% if KOMA class
  \KOMAoptions{parskip=half}}
\makeatother
% Make \paragraph and \subparagraph free-standing
\makeatletter
\ifx\paragraph\undefined\else
  \let\oldparagraph\paragraph
  \renewcommand{\paragraph}{
    \@ifstar
      \xxxParagraphStar
      \xxxParagraphNoStar
  }
  \newcommand{\xxxParagraphStar}[1]{\oldparagraph*{#1}\mbox{}}
  \newcommand{\xxxParagraphNoStar}[1]{\oldparagraph{#1}\mbox{}}
\fi
\ifx\subparagraph\undefined\else
  \let\oldsubparagraph\subparagraph
  \renewcommand{\subparagraph}{
    \@ifstar
      \xxxSubParagraphStar
      \xxxSubParagraphNoStar
  }
  \newcommand{\xxxSubParagraphStar}[1]{\oldsubparagraph*{#1}\mbox{}}
  \newcommand{\xxxSubParagraphNoStar}[1]{\oldsubparagraph{#1}\mbox{}}
\fi
\makeatother

\usepackage{color}
\usepackage{fancyvrb}
\newcommand{\VerbBar}{|}
\newcommand{\VERB}{\Verb[commandchars=\\\{\}]}
\DefineVerbatimEnvironment{Highlighting}{Verbatim}{commandchars=\\\{\}}
% Add ',fontsize=\small' for more characters per line
\usepackage{framed}
\definecolor{shadecolor}{RGB}{241,243,245}
\newenvironment{Shaded}{\begin{snugshade}}{\end{snugshade}}
\newcommand{\AlertTok}[1]{\textcolor[rgb]{0.68,0.00,0.00}{#1}}
\newcommand{\AnnotationTok}[1]{\textcolor[rgb]{0.37,0.37,0.37}{#1}}
\newcommand{\AttributeTok}[1]{\textcolor[rgb]{0.40,0.45,0.13}{#1}}
\newcommand{\BaseNTok}[1]{\textcolor[rgb]{0.68,0.00,0.00}{#1}}
\newcommand{\BuiltInTok}[1]{\textcolor[rgb]{0.00,0.23,0.31}{#1}}
\newcommand{\CharTok}[1]{\textcolor[rgb]{0.13,0.47,0.30}{#1}}
\newcommand{\CommentTok}[1]{\textcolor[rgb]{0.37,0.37,0.37}{#1}}
\newcommand{\CommentVarTok}[1]{\textcolor[rgb]{0.37,0.37,0.37}{\textit{#1}}}
\newcommand{\ConstantTok}[1]{\textcolor[rgb]{0.56,0.35,0.01}{#1}}
\newcommand{\ControlFlowTok}[1]{\textcolor[rgb]{0.00,0.23,0.31}{\textbf{#1}}}
\newcommand{\DataTypeTok}[1]{\textcolor[rgb]{0.68,0.00,0.00}{#1}}
\newcommand{\DecValTok}[1]{\textcolor[rgb]{0.68,0.00,0.00}{#1}}
\newcommand{\DocumentationTok}[1]{\textcolor[rgb]{0.37,0.37,0.37}{\textit{#1}}}
\newcommand{\ErrorTok}[1]{\textcolor[rgb]{0.68,0.00,0.00}{#1}}
\newcommand{\ExtensionTok}[1]{\textcolor[rgb]{0.00,0.23,0.31}{#1}}
\newcommand{\FloatTok}[1]{\textcolor[rgb]{0.68,0.00,0.00}{#1}}
\newcommand{\FunctionTok}[1]{\textcolor[rgb]{0.28,0.35,0.67}{#1}}
\newcommand{\ImportTok}[1]{\textcolor[rgb]{0.00,0.46,0.62}{#1}}
\newcommand{\InformationTok}[1]{\textcolor[rgb]{0.37,0.37,0.37}{#1}}
\newcommand{\KeywordTok}[1]{\textcolor[rgb]{0.00,0.23,0.31}{\textbf{#1}}}
\newcommand{\NormalTok}[1]{\textcolor[rgb]{0.00,0.23,0.31}{#1}}
\newcommand{\OperatorTok}[1]{\textcolor[rgb]{0.37,0.37,0.37}{#1}}
\newcommand{\OtherTok}[1]{\textcolor[rgb]{0.00,0.23,0.31}{#1}}
\newcommand{\PreprocessorTok}[1]{\textcolor[rgb]{0.68,0.00,0.00}{#1}}
\newcommand{\RegionMarkerTok}[1]{\textcolor[rgb]{0.00,0.23,0.31}{#1}}
\newcommand{\SpecialCharTok}[1]{\textcolor[rgb]{0.37,0.37,0.37}{#1}}
\newcommand{\SpecialStringTok}[1]{\textcolor[rgb]{0.13,0.47,0.30}{#1}}
\newcommand{\StringTok}[1]{\textcolor[rgb]{0.13,0.47,0.30}{#1}}
\newcommand{\VariableTok}[1]{\textcolor[rgb]{0.07,0.07,0.07}{#1}}
\newcommand{\VerbatimStringTok}[1]{\textcolor[rgb]{0.13,0.47,0.30}{#1}}
\newcommand{\WarningTok}[1]{\textcolor[rgb]{0.37,0.37,0.37}{\textit{#1}}}

\usepackage{longtable,booktabs,array}
\usepackage{calc} % for calculating minipage widths
% Correct order of tables after \paragraph or \subparagraph
\usepackage{etoolbox}
\makeatletter
\patchcmd\longtable{\par}{\if@noskipsec\mbox{}\fi\par}{}{}
\makeatother
% Allow footnotes in longtable head/foot
\IfFileExists{footnotehyper.sty}{\usepackage{footnotehyper}}{\usepackage{footnote}}
\makesavenoteenv{longtable}
\usepackage{graphicx}
\makeatletter
\newsavebox\pandoc@box
\newcommand*\pandocbounded[1]{% scales image to fit in text height/width
  \sbox\pandoc@box{#1}%
  \Gscale@div\@tempa{\textheight}{\dimexpr\ht\pandoc@box+\dp\pandoc@box\relax}%
  \Gscale@div\@tempb{\linewidth}{\wd\pandoc@box}%
  \ifdim\@tempb\p@<\@tempa\p@\let\@tempa\@tempb\fi% select the smaller of both
  \ifdim\@tempa\p@<\p@\scalebox{\@tempa}{\usebox\pandoc@box}%
  \else\usebox{\pandoc@box}%
  \fi%
}
% Set default figure placement to htbp
\def\fps@figure{htbp}
\makeatother





\setlength{\emergencystretch}{3em} % prevent overfull lines

\providecommand{\tightlist}{%
  \setlength{\itemsep}{0pt}\setlength{\parskip}{0pt}}



 


\KOMAoption{captions}{tableheading}
\makeatletter
\@ifpackageloaded{bookmark}{}{\usepackage{bookmark}}
\makeatother
\makeatletter
\@ifpackageloaded{caption}{}{\usepackage{caption}}
\AtBeginDocument{%
\ifdefined\contentsname
  \renewcommand*\contentsname{Table of contents}
\else
  \newcommand\contentsname{Table of contents}
\fi
\ifdefined\listfigurename
  \renewcommand*\listfigurename{List of Figures}
\else
  \newcommand\listfigurename{List of Figures}
\fi
\ifdefined\listtablename
  \renewcommand*\listtablename{List of Tables}
\else
  \newcommand\listtablename{List of Tables}
\fi
\ifdefined\figurename
  \renewcommand*\figurename{Figure}
\else
  \newcommand\figurename{Figure}
\fi
\ifdefined\tablename
  \renewcommand*\tablename{Table}
\else
  \newcommand\tablename{Table}
\fi
}
\@ifpackageloaded{float}{}{\usepackage{float}}
\floatstyle{ruled}
\@ifundefined{c@chapter}{\newfloat{codelisting}{h}{lop}}{\newfloat{codelisting}{h}{lop}[chapter]}
\floatname{codelisting}{Listing}
\newcommand*\listoflistings{\listof{codelisting}{List of Listings}}
\makeatother
\makeatletter
\makeatother
\makeatletter
\@ifpackageloaded{caption}{}{\usepackage{caption}}
\@ifpackageloaded{subcaption}{}{\usepackage{subcaption}}
\makeatother
\usepackage{bookmark}
\IfFileExists{xurl.sty}{\usepackage{xurl}}{} % add URL line breaks if available
\urlstyle{same}
\hypersetup{
  pdftitle={Quantitatives Portfoliomanagement mit Python},
  pdfauthor={Thomas Mählmann},
  colorlinks=true,
  linkcolor={blue},
  filecolor={Maroon},
  citecolor={Blue},
  urlcolor={Blue},
  pdfcreator={LaTeX via pandoc}}


\title{Quantitatives Portfoliomanagement mit Python}
\author{Thomas Mählmann}
\date{}
\begin{document}
\maketitle

\renewcommand*\contentsname{Table of contents}
{
\hypersetup{linkcolor=}
\setcounter{tocdepth}{2}
\tableofcontents
}

\bookmarksetup{startatroot}

\chapter{\texorpdfstring{Willkommen zu \emph{KI \& Datenanalyse mit
Python}}{Willkommen zu KI \& Datenanalyse mit Python}}\label{willkommen-zu-ki-datenanalyse-mit-python}

Dieses Buch führt dich Schritt für Schritt in die Welt der Datenanalyse
und des maschinellen Lernens ein.

\begin{Shaded}
\begin{Highlighting}[]
\ImportTok{import}\NormalTok{ pandas }\ImportTok{as}\NormalTok{ pd}
\ImportTok{import}\NormalTok{ numpy }\ImportTok{as}\NormalTok{ np}

\BuiltInTok{print}\NormalTok{(}\StringTok{\textquotesingle{}Hallo Quarto{-}Buch!\textquotesingle{}}\NormalTok{)}
\end{Highlighting}
\end{Shaded}

\bookmarksetup{startatroot}

\chapter{Absolute Optimierung}\label{absolute-optimierung}

Gegenstand der \textbf{Portfoliobildung} bzw.
\textbf{Portfoliorealisierung} ist die wertmäßige Aufteilung des
Anlagebetrages auf die Assets des in Frage kommenden Anlageuniversums.

Bei der Portfoliobildung durch Anwendung von Optimierungsverfahren
erfolgt die Bestimmung der optimalen Portfoliostruktur (Anteilsgewichte
der Assets) in Bezug auf den jeweils verfolgten Zweck (operationalisiert
durch eine Zielfunktion \emph{ZF}) unter Beachtung eventueller
Nebenbedingungen.

\section{\texorpdfstring{\((\mu-\sigma)\)-effiziente Portfolios ohne
risikoloses
Asset}{(\textbackslash mu-\textbackslash sigma)-effiziente Portfolios ohne risikoloses Asset}}\label{mu-sigma-effiziente-portfolios-ohne-risikoloses-asset}

Was sind optimale Portfolios und wie lassen sie sich bestimmen?
Markowitz (1952, 1998) beschränkt zur Bewältigung dieses Problems seine
Lösung darauf, dass aus der Menge aller möglichen und zulässigen
Portfolios diejenigen ausgeschlosen werden, welche eindeutig schlechter
sind als andere. Die verbleibenden sogenannten
\((\mu-\sigma)\)-effizienten Portfolios bestimmen sich durch die Auswahl
derjenigen Portfolios aus der Menge aller möglichen Portfolios, welche
bei gegebenem Erwartungswert der Rendite \((\mu_{p})\) das minimale
Risiko (Varianz bzw. Standardabweichung der Rendite - \(\sigma^2_{p}\)
bzw. \(\sigma_{p}\)) aufweisen, oder bei gegebenem Risiko die maximale
Rendite erwarten lassen. Diese Portfolios werden auch als dominant
gegenüber den anderen Portfolios bezeichnet.

Diese Auswahl effizienter Portfolios basiert jedoch implizit auf zwei
zentralen Annahmen:

\begin{itemize}
\tightlist
\item
  Investoren bevorzugen mehr Rendite gegenüber weniger.
\item
  Weisen zwei Portfolios dieselbe erwartete Rendite auf, so wird
  dasjenige mit dem geringeren Risiko gewählt.
\end{itemize}

Damit unterstellt Markowitz implizit \emph{risikoaverse Anleger}. Das
\emph{Dominanzkriterium} lässt sich also hier in folgender Weise
formulieren:

Eine Anlage (Portfolio) dominiert eine andere, wenn sie

\begin{itemize}
\tightlist
\item
  bei gleicher Rendite ein geringeres Risiko aufweist, oder
\item
  bei gleichem Risiko eine höhere Rendite besitzt.
\end{itemize}

Ein Portfolio wird genau dann effizient genannt, wenn kein anderes
Portfolio existiert, welches

\begin{itemize}
\tightlist
\item
  bei gleicher erwarteter Rendite ein geringeres Risiko oder
\item
  bei gleichem Risiko eine höhere erwartete Rendite oder
\item
  bei höherer erwarteter Rendite gleichzeitig ein geringeres Risiko
\end{itemize}

besitzt. Damit ergibt sich unter Verwendung des Dominanzkriteriums die
äquivalente Aussage:

\begin{itemize}
\tightlist
\item
  Ein Portfolio ist genau dann effizient, wenn kein anderes Portfolio
  existiert, welches dieses dominiert.
\end{itemize}

Die Menge aller effizienten Portfolios wird auch als Rand oder Kurve
aller effizienten Portfolios oder kurz \emph{Effizienzkurve}
(``Efficient Frontier'') bezeichnet.

\subsection{Analytische Bestimmung der
Effizienzkurve}\label{analytische-bestimmung-der-effizienzkurve}

\subsubsection{Allgemein (mit
Leerverkaufsverbot)}\label{allgemein-mit-leerverkaufsverbot}

Um die Effizienzkurve rechnerisch zu bestimmen, ist das folgende
quadratische Optimierungsproblem (in Matrizenschreibweise) zu lösen:

\[
\begin{split}
 \\(OP1) \quad ZF(w) = \sigma_P^2=w^{T}\Sigma w \rightarrow \min_{w}!, \\
 \text{unter den Nebenbedingungen}\quad
 & \text{(a)}\ w^{T}\mu = \mu_{P}, \\
 & \text{(b)}\ w^{T}\iota = 1 \quad \text{(Budgetrestriktion)}, \\
 & \text{(c)}\ w\geqq 0 \quad \text{(Leerverkaufsverbot)},
\end{split}
\]

wobei

\[
 w = \begin{bmatrix} w_1 \\ \vdots \\ w_N \end{bmatrix}, \quad 
 \mu = \begin{bmatrix} \mu_1 \\ \vdots \\ \mu_N \end{bmatrix}, \quad 
 \Sigma = \begin{bmatrix} \sigma_1^2 & \cdots & \sigma_{1N} \\
 \vdots & \ddots & \vdots \\
 \sigma_{N1} & \cdots & \sigma_N^2 \\
 \end{bmatrix}.
\]

\begin{itemize}
\tightlist
\item
  \(\mu_{P}\) : beliebige, fest vorgegebene Portfoliorendite
\item
  \(\mu\) : Vektor der erwarteten Assetrenditen
\item
  \(\Sigma\) : zukünftige Varianz-Kovarianzmatrix der Assetrenditen
\item
  \(\iota\) : Einheitsvektor (alle Elemente des Vektors sind eins)
\end{itemize}

Für eine beliebige, fest vorgegebene Portfoliorendite \(\mu_{P}\) werden
also die Gewichte \(w_{i}\) (\(i=1, ..., N\), und \(N=\) Anzahl der
Assets im Anlageuniversum) derart bestimmt, dass das resultierende
Portfoliorisiko (Varianz der Portfoliorendite) minimal wird. Die Lösung
des Optimierungsproblems OP1 liefert einen Punkt
\((\mu_{P}, \sigma^2_{p})\) der Effizienzkurve. Löst man nun das
Optimierungsproblem für variierende \(\mu_{P}\), so lässt sich die
Effizienzkurve punktweise konstruieren.

\subsubsection{Diskussion der
Nebenbedingungen}\label{diskussion-der-nebenbedingungen}

Als mögliche Nebenbedingungen wurden die \textbf{Budgetrestriktion} und
die \textbf{Nichtnegativitätsbedingung (Leerverkaufsverbot)} eingeführt.
Die Budgetrestriktion beinhaltet die Anforderung, dass die Summe der
Einzelgewichte gleich eins ist. Der Investor kann also nicht mehr als
den zur Verfügung stehenden Anlagebetrag auf die Anlagen aufteilen. Dies
schließt die Aufnahme von Fremdkapital (Leverage) aus. Gleichzeitig wird
aber auch die Vollinvestition gefordert. Ist eventuell eine
Bargeldhaltung zu berücksichtigen, so ist Bargeld einfach als ein Asset
mit erwarteter Rendite und Risiko von Null in das Anlageuniversum zu
integrieren.

Die Möglichkeit von Leerverkäufen wird durch die
Nichtnegativitätsanforderung an die Einzelgewichte ausgeschlossen. Dies
ist eine in der Praxis übliche Beschränkung, die sich z.B. aufgrund
eines gesetzlichen oder satzungsmäßigen Verbots von Leerverkäufen
ergeben kann. Unter einem \emph{Leerverkauf} versteht man den Verkauf
eines Wertpapiers, welches sich nicht im Eigentum des Verkäufers
befindet, sondern von ihm mittels \emph{Wertpapierleihe} beschafft wird.
Ein Leerverkäufer zielt darauf ab, unter Erwartung fallender Kurse, das
Wertpapier später zu einem niedrigeren Kurs erwerben zu können, so dass
die Differenz zwischen Kaufkurs und Verkaufkurs die zu entrichtende
Leihgebühr überkompensiert.

Ferner kann es eine Vielzahl von weiteren Nebenbedingungen gesetzlicher,
statutarischer oder persönlicher Art geben. Hierzu zählen z.B. Mindest-
und Höchstbestandsgrenzen für einzelne Assets, zulässige
Höchstbestandsgrenzen für Gruppen von Assets, eine geforderte
Mindestdividendenrendite des Portfolios, Ausschluss bestimmter Assets
(z.B. ``Sin Stocks''), Beachtung von Nachhaltigkeitskriterien (z.B. ein
Portfoliominimum-ESG-Score), Beschränkung des Umschichtungsvolumens, der
Transaktionskosten usw.

Offensichtlich erschwert das Hinzufügen von (linearen und nichtlinearen)
Nebenbedingungen die Lösungsfindung bei der Optimierung. Da bei einem
Optimierungsproblem eine hochgradig nichtlineare Zielfunktion mit
ebenfalls hochgradig komplexen, nichtlinearen Nebenbedingungen vorliegen
kann, lässt sich schon intuitiv vermuten, dass kein gleichermaßen
allgemein gültiges wie zugleich effizientes Lösungsverfahren existiert.

\subsubsection{Sonderfall (ohne
Leerverkaufsverbot)}\label{sonderfall-ohne-leerverkaufsverbot}

\begin{enumerate}
\def\labelenumi{\arabic{enumi}.}
\tightlist
\item
  Geschlossene Lösung
\end{enumerate}

In der Regel lässt sich keine geschlossene Lösung des obigen
Optimierungsproblems angeben. Entfällt aber die Nebenbedingung (c), das
Leerverkaufsverbot (bzw. die Nichtnegativitätsbedingung), kann über
einen Lagrangeansatz die folgende geschlossene Lösung für den optimalen
(d.h., varianzminimalen) Portfoliogewichtsvektor \(w\) bei gegebener
erwarteter Portfoliorendite \(\mu_{P}\) ermittelt werden (siehe z.B.
Franzen und Schäfer, 2018, S. 182-189, für die Herleitung)

\[
 w = 
 \frac{C\mu_P-A}{D}\Sigma^{-1}\mu + \frac{B-A\mu_P}{D}\Sigma^{-1}\iota,
\]

wobei

\[
 A =\mu^{T}\Sigma^{-1}\iota,\ B=\mu^{T}\Sigma^{-1}\mu,\ 
 C =\iota^{T}\Sigma^{-1}\iota,\ D = B C - A^2.
\]

Die Beziehung zwischen der gegebenen Portfoliorendite \(\mu_P\) und dem
dazugehörigen minimalen Portfoliorisiko \(\sigma_P\) läßt sich dann
schreiben als:

\[
 \sigma_P = \sqrt{\frac{C\mu_P^2 - 2A\mu_P + B}{D}}
 = \sqrt{\frac{C}{D}\left(\mu_P-\frac{A}{C}\right)^2+\frac1{C}},
\]

und für die \((\mu_P-\sigma_{P})\)-Effizienzkurve gilt

\[ (1) \quad
\mu_P = \begin{cases}
\displaystyle
\frac{A + \sqrt{D(C\sigma_P^2 - 1)}}{C}, & \mu_P > \frac{A}{C}; \\
& \\
\displaystyle
\frac{A - \sqrt{D(C\sigma_P^2 - 1)}}{C}, & \mu_P < \frac{A}{C}.
\end{cases}
\]

\begin{enumerate}
\def\labelenumi{\arabic{enumi}.}
\setcounter{enumi}{1}
\tightlist
\item
  Two-Fund Theorem
\end{enumerate}

Im \((\mu-\sigma)\)-Koordinatensystem stellen die varianzminimalen
Portfolios, die für eine gegebene erwartete Rendite das Risiko (Varianz
oder Standardabweichung der Rendite) minimieren, eine Parabel
(Möglichkeitskurve - ``Envelop'') dar. Der effiziente obere Rand der
Parabel ist die Effizienzkurve. Der untere Rand (die
``Ineffizienzkurve'') enthält ineffiziente Portfolios, die von den
Portfolios des effizienten Rands dominiert werden (diese Portfolios
besitzen für ein gegebenes \(\sigma_{p}\) jeweils ein größeres
\(\mu_{p})\). Der Scheitelpunkt der Parabel, der den effizienten vom
ineffizienten Rand trennt, ist das sogenannte
\emph{Minimum-Varianz-Portfolio} (siehe unten).

Wir können nun zwei Theoreme, die auf Black (1972) und Merton (1972)
zurückgehen (siehe hierzu auch Benninga, 2014, Kapitel 9), verwenden, um
mit Hilfe zweier beliebiger Basisportfolios auf der Parabel die gesamte
Parabel aufzuspannen.

Das erste Theorem besagt, dass sich die Anteilsgewichte eines jeden
Portfolios der Parabel beschreiben lassen durch:

\[ (2) \quad w(c)=\frac{\Sigma^{-1}(\mu-c \iota)}{\iota^T[\Sigma^{-1}(\mu-c\iota)]},\]

wobei \(c\) eine beliebige Konstante und \(\iota\) den Einheitsvektor
der Dimension \(N\) darstellt. Durch Variation von \(c\) erhalten wir
unterschiedliche Punkte auf der Parabel. Nun wählen wir zwei beliebige
Basisportfolios \(X\) und \(Y\) auf der Parabel. Für \(X\) setzen wir
z.B. \(c=2\) und für \(Y\) \(c=4\).

Das zweite Theorem besagt, gegeben zwei Basisportfolios \(X\) und \(Y\)
auf der Parabel, jedes weitere Portfolio \(Z\) (Punkt auf der Parabel)
lässt sich als konvexe Kombination von \(X\) und \(Y\) darstellen:
\(Z=\alpha X+(1-\alpha)Y\), wobei \(\alpha\) eine beliebige Konstante
ist. Daraus folgt für die erwartete Rendite und das Risiko von \(Z\):

\[(3a) \quad \mu_{Z} = \alpha \mu_{X} + (1-\alpha) \mu_{Y} \]

\[(3b) \quad \sigma_{Z}=\sqrt{\alpha^2 \sigma^2_{X} + (1-\alpha^2) \sigma^2_{Y}+ 2\alpha (1-\alpha)Cov(X,Y)} \]

\(Cov(X,Y)\) bezeichnet hier die Renditekovarianz zwischen den beiden
Basisportfolios.

Die Bestimmung der Parabel der varianzminimalen Portfolios (effizienter
+ ineffizienter Rand) anhand des Two-Fund Theorems erfordert somit die
folgenden Schritte:

\begin{enumerate}
\def\labelenumi{\arabic{enumi}.}
\tightlist
\item
  Wahl der Basisportfolios \(X\) und \(Y\) durch die Festlegung zweier
  beliebiger Werte für \(c\).
\item
  Bestimmung der Anteilsgewichte von \(X\) und \(Y\) anhand Gleichung
  (2).
\item
  Ermittlung der erwarteten Rendite, Varianz der Rendite und der
  Renditekovarianz zwischen \(X\) und \(Y\).
\item
  Für eine große Zahl unterschiedlicher Werte für \(\alpha\): Berechnung
  der erwarteten Rendite und der Renditevarianz für \(Z\) gemäß der
  Gleichungen (3a) und (3b).
\end{enumerate}

\subsection{Zwei extreme optimale Portfolios: MVP und
MEP}\label{zwei-extreme-optimale-portfolios-mvp-und-mep}

Die Effizienzkurve optimaler, d.h. \((\mu-\sigma)\)-effizienter,
Portfolios wird begrenzt von zwei Extrempositionen, dem
\emph{Minimum-Varianz-Portfolio (MVP)} und dem
\emph{Maximum-Ertrag-Portfolio (MEP)}.

Das MVP ist dasjenige Portfolio, welches das \emph{global} geringste zu
erwartende Risiko aufweist. Das Optimierungsproblem (in
Matrizenschreibweise) lautet: \[
\begin{split}
 \\(OP2) \quad ZF(w) = \sigma_P^2=w^{T}\Sigma w \rightarrow \min_{w}!, \\
 \text{unter den Nebenbedingungen}\quad
 & \text{(a)}\ w^{T}\iota = 1 \quad \text{(Budgetrestriktion)}, \\
 & \text{(b)}\ w\geqq 0 \quad \text{(Leerverkaufsverbot)},
\end{split}
\]

Für das am anderen Ende der Effizienzkurve liegende MEP gilt:

\[ (OP3) \quad ZF(w) = \mu_{P}=w^{T}\mu  \rightarrow \max_{w}! \]

Die obigen Nebenbedingungen gelten hier analog. Für das MEP bildet das
Portfolio folglich die Lösung, welches zu 100\% aus dem Asset mit der
höchsten zu erwartenden Rendite besteht.

\subsection{Einführung eines risikolosen
Assets}\label{einfuxfchrung-eines-risikolosen-assets}

Das Problem der Bestimmung aller effizienten Portfolios kann unter
Anwendung der \emph{Tobin-Separation} vereinfacht werden. Diese
ermöglicht es, durch Einführung einer \emph{risikofreien} Anlage, nur
noch ein effizientes Portfolio bestimmen zu müssen. Die Einführung einer
risikolosen Anlagemöglichkeit mit Zinssatz \(r_{f}\) führt zu einem
modifizierten Problem: Der Anleger hat nun die Möglichkeit,
\emph{Mischportfolios} aus der risikolosen Anlagemöglichkeit und einem
beliebigen Portfolio auf der Effizienzkurve zu bilden, z.B. mit dem
Portfolio \emph{P}. Die sich dann ergebenden
\((\mu-\sigma)\)-Kombinationen für variierende Mischungsverhältnisse von
\(r_{f}\) und \emph{P} liegen auf einer Geraden. Ziel ist es nun,
dasjenige Portfolio \emph{TP} auf der Effizienzkurve für die Mischung
mit \(r_{f}\) zu verwenden, für dass die resultierenden Mischportfolios
nicht durch Kombinationen von \(r_{f}\) mit anderen effizienten
Portfolios \emph{P} dominiert werden.

Das Portfolio \emph{TP} wird dabei in folgenden Weise bestimmt:

\[
\begin{split}
 \\(OP4) \quad ZF(w) =  \frac{\mu_p-r_f}{\sigma_p} \rightarrow \max_{w}!, \\
 \text{unter den Nebenbedingungen}\quad
 & \text{(a)}\ w^{T}\iota = 1 \quad \text{(Budgetrestriktion)}, \\
 & \text{(b)}\ w\geqq 0 \quad \text{(Leerverkaufsverbot)},
\end{split}
\]

Das obige Optimierungsproblem OP4 entspricht der Maximierung der
Steigung der Geraden durch den Punkt \(r_{f}\) einerseits und durch
einen durch ein effizientes Portfolio bestimmten Punkt andererseits.
Dies ist gleichbedeutend mit der Bestimmung der Tangente an die
Effizienzkurve ausgehend vom Punkt \(r_{f}\). Der Ausdruck
\(\frac{\mu_p-r_f}{\sigma_p}\) wird auch als die \textbf{Sharpe-Ratio}
des Portfolios \(P\) bezeichnet.

Die sich für den risikoaversen Investor neu ergebende
\emph{Effizienzline} wird durch alle möglichen Kombinationen aus dem
\emph{Tangentialportfolio TP} und der risikofreien Anlage \(r_{f}\)
gebildet. Für dieses Mischportfolio ergeben sich die erwartete Rendite
und dessen Risiko wie folgt:

\[
\begin{split}
\ \mu_{Misch}=\alpha\mu_{TP}+(1-\alpha)r_{f} \
\text{und}\
\ \sigma_{Misch}=\alpha \sigma_{TP}, \
\end{split}
\]

mit \(\alpha=\) Anteil der Investitionssumme, der in das risikobehaftete
Portfolio \emph{TP} investiert wird.

\subsection{Bestimmung des anlegerindividuell-optimalen
Portfolios}\label{bestimmung-des-anlegerindividuell-optimalen-portfolios}

Mit der Bestimmung der effizienten Portfolios ist man aber noch nicht am
Ziel. Gewünscht ist schlussendlich die Bestimmung eines für den
individuellen Anleger optimalen Portfolios (Poddig et al., S. 84).

Im Rahmen dieser Vorlesung werden wir die Bestimmung
anlegerindividuell-optimaler Portfolios nicht behandeln!

\subsection{Beginn der Fallstudie}\label{beginn-der-fallstudie}

Wir starten mit dem Import der benötigten Pakete.

\begin{Shaded}
\begin{Highlighting}[]
\ImportTok{import}\NormalTok{ pandas }\ImportTok{as}\NormalTok{ pd}
\ImportTok{import}\NormalTok{ numpy }\ImportTok{as}\NormalTok{ np}
\ImportTok{from}\NormalTok{ scipy.optimize }\ImportTok{import}\NormalTok{ minimize}
\ImportTok{import}\NormalTok{ matplotlib.pyplot }\ImportTok{as}\NormalTok{ plt}
\ImportTok{import}\NormalTok{ numpy.linalg }\ImportTok{as}\NormalTok{ la}
\end{Highlighting}
\end{Shaded}

\subsubsection{Anmerkungen zur Umsetzung der numerischen Optimierung in
Python}\label{anmerkungen-zur-umsetzung-der-numerischen-optimierung-in-python}

Wir setzen im Folgenden die numerische Optimierung der jeweils
formulierten Zielfunktion (OP1-OP4) unter Nebenbedingungen mit Hilfe der
Funktion \texttt{minimize} aus dem Modul \texttt{scipy.optimize} um. Bei
der Umsetzung sind einige Besonderheiten zu beachten.

Erstens bietet Scipy eine ``Minimieren''-Funktionalität, aber keine
``Maximieren''-Funktion. Möchten wir beispielsweise die Sharpe-Ratio im
Optimierungsproblem 4 (OP4) maximieren, mag dies auf den ersten Blick
wie ein kleines Problem erscheinen, aber es lässt sich leicht lösen,
wenn man bedenkt, dass die Maximierung der Sharpe-Ratio analog zur
Minimierung der negativen Sharpe-Ratio ist - also buchstäblich nur der
Sharpe-Ratio-Wert mit einem vorangestellten Minuszeichen.

Das grundsätzliche Vorgehen bei der Optimierung ist immer identisch:
Zuerst schreiben wir die zu optimierende \texttt{Zielfunktion} als
Funktion und definieren dann über das Objekt \texttt{constraints} die
Struktur der Nebenbedingungen.

Lassen Sie uns die einzelnen Einträge durchgehen, um sie besser zu
verstehen:

Da wir die \texttt{SLSQP}-Methode in unserer ``Minimieren''-Funktion
verwenden werden (was für ``Sequential Least Squares Programming''
steht), muss das \texttt{constraints} Objekt das Format eines Tupels von
\texttt{Dictionaries} haben, das die Felder \texttt{type} und
\texttt{fun} mit den optionalen Feldern \texttt{jac} und \texttt{args}
enthält. Wir brauchen nur die Felder \texttt{type}, \texttt{fun} und
\texttt{args}. Ein Tupel ist eine unveränderliche Sequenz fester Länge
aus Python-Objekten, eingebettet in runde Klammern () wobei die
einzelnen Objekte kommasepariert sind.

Der \texttt{type} kann entweder \texttt{eq} oder \texttt{ineq} sein, was
sich auf \emph{equality} bzw. \emph{inequality} bezieht. Das
\texttt{fun} bezieht sich auf die Funktion, die die Beschränkung
definiert, in unserem Fall die Beschränkung, dass die Summe der
Anteilsgewichte Eins sein muss (Budgetrestriktion). Die Art und Weise,
wie dies eingegeben werden muss, ist etwas umständlich. Das \texttt{eq}
bedeutet, dass wir nach einer Funktion suchen, deren Ausgabewert gleich
Null ist (das ist es, worauf sich die Gleichheit bezieht - Gleichheit
mit Null). Der einfachste Weg, dies zu erreichen, besteht darin, eine
\texttt{Lambda}-Funktion zu erstellen, die die Summe der
Portfoliogewichte minus Eins liefert. Die Beschränkung, dass der
Ausgabewert dieser Funktion gleich Null sein muss, bedeutet per
Definition, dass die Gewichte zu Eins summiert werden müssen.

Über das Tuple \texttt{bounds}, das \(N\) identische
\texttt{bound}=(Mindestbestandsgrenze, Höchstbestandsgrenze) Tuple
enthält, legen wir fest, dass jedes einzelne Anteilsgewicht zwischen
Null und Eins liegen muss (Leerverkaufsverbot). Die \texttt{args} sind
die Argumente, die wir an die Funktion übergeben wollen, die wir zu
minimieren versuchen (\texttt{Zielfunktion}) - das sind alle Argumente
AUßER dem Anteilsvektor, der natürlich das Zielfunktionsargument ist,
das wir zur Optimierung der ZF-Ausgabe ändern.

Über die Liste \texttt{Startgewichte} legen wir den Ausgangspunkt für
die numerische Suche nach den optimalen Gewichten fest. Der grundlegende
Code für die Anwendung der \texttt{minimize}-Optimierungsfunktion sieht
demnach folgendermaßen aus:

\begin{Shaded}
\begin{Highlighting}[]
\CommentTok{\# basic structure of code for optimization}
\NormalTok{constraints }\OperatorTok{=}\NormalTok{ (\{}\StringTok{\textquotesingle{}type\textquotesingle{}}\NormalTok{: }\StringTok{\textquotesingle{}eq\textquotesingle{}}\NormalTok{, }\StringTok{\textquotesingle{}fun\textquotesingle{}}\NormalTok{: }\KeywordTok{lambda}\NormalTok{ x: np.}\BuiltInTok{sum}\NormalTok{(x) }\OperatorTok{{-}} \DecValTok{1}\NormalTok{\})}
\NormalTok{bound }\OperatorTok{=}\NormalTok{ (}\FloatTok{0.0}\NormalTok{,}\FloatTok{1.0}\NormalTok{)}
\NormalTok{bounds }\OperatorTok{=} \BuiltInTok{tuple}\NormalTok{(bound }\ControlFlowTok{for}\NormalTok{ i }\KeywordTok{in} \BuiltInTok{range}\NormalTok{(Anzahl\_Assets))}
\NormalTok{minimize(Zielfunktion, Startgewichte, args}\OperatorTok{=}\NormalTok{args,}
\NormalTok{                method}\OperatorTok{=}\StringTok{\textquotesingle{}SLSQP\textquotesingle{}}\NormalTok{, bounds}\OperatorTok{=}\NormalTok{bounds, constraints}\OperatorTok{=}\NormalTok{constraints)}
\end{Highlighting}
\end{Shaded}

\texttt{minimize} liefert als Ausgabe das Array \texttt{x} mit den
optimierten Anteilsgewichten.

\subsubsection{Laden und Beschreiben der
Datenbasis}\label{laden-und-beschreiben-der-datenbasis}

Das beispielhafte Anlageuniversum der Fallstudie umfasst zehn
Unternehmen aus den folgenden Branchen: Technologie, Gesundheit,
Nahrungsmittel, Pharma, Energie sowie Luft- und Raumfahrt. Die
Unternehmen sind: Abbott Laboratories (ABT), Boeing Industries (BA),
Costco Wholesale (COST), Cisco Systems (CSCO), IBM (IBM), Intel (INTC),
Merk (MRK), Microsoft (MSFT), AT\&T (T), und Exxon Mobil Corporation
(XOM).

Die Datengrundlage stellen Monatsanfangskurse (``Adjusted Close'') über
einen 5-Jahres-Zeitraum vom 1.12.2004 bis zum 1.12.2009 dar. Die
Stichprobe enthält somit 61 Zeitreihenbeobachtungen.

\begin{Shaded}
\begin{Highlighting}[]
\CommentTok{\# Hier Ihren lokalen Pfad zur Datei \textquotesingle{}Kapitel A1\textquotesingle{} eingeben!}
\CommentTok{\# cd "..."}
\end{Highlighting}
\end{Shaded}

\begin{Shaded}
\begin{Highlighting}[]
\NormalTok{frame }\OperatorTok{=}\NormalTok{ pd.read\_excel(}\StringTok{\textquotesingle{}Kapitel A1.xlsx\textquotesingle{}}\NormalTok{, }\StringTok{\textquotesingle{}Tabelle1\textquotesingle{}}\NormalTok{, index\_col}\OperatorTok{=}\DecValTok{0}\NormalTok{, parse\_dates}\OperatorTok{=}\VariableTok{True}\NormalTok{)}
\end{Highlighting}
\end{Shaded}

\begin{Shaded}
\begin{Highlighting}[]
\NormalTok{frame.head()}
\end{Highlighting}
\end{Shaded}

\begin{longtable}[]{@{}lllllllllll@{}}
\toprule\noalign{}
& ABT & BA & COST & CSCO & IBM & INTC & MRK & MSFT & T & XOM \\
\midrule\noalign{}
\endhead
\bottomrule\noalign{}
\endlastfoot
2004-12-01 & 46.65 & 51.77 & 48.41 & 19.32 & 98.58 & 23.39 & 32.14 &
26.72 & 25.77 & 51.26 \\
2005-01-03 & 45.02 & 50.60 & 47.27 & 18.04 & 93.42 & 22.45 & 28.05 &
26.28 & 23.76 & 51.60 \\
2005-02-01 & 45.99 & 54.97 & 46.59 & 17.42 & 92.58 & 23.99 & 31.70 &
25.16 & 24.06 & 63.31 \\
2005-03-01 & 46.62 & 58.46 & 44.18 & 17.89 & 91.38 & 23.23 & 32.37 &
24.17 & 23.69 & 59.60 \\
2005-04-01 & 49.16 & 59.52 & 40.63 & 17.27 & 76.38 & 23.52 & 33.90 &
25.30 & 23.80 & 57.03 \\
\end{longtable}

\subsubsection{1. Schritt: Schätzung der Inputdaten für die
Optimierung}\label{schritt-schuxe4tzung-der-inputdaten-fuxfcr-die-optimierung}

Alle oben formulierten Optimierungsprobleme benötigen als Ausgangsdaten
den Vektor der erwarteten, zukünftigen Assetrenditen \(\mu\) und die
zukünftige Varianz-Kovarianzmatrix \(\Sigma\). Unterschiedliche Methode
stehen zur Schätzung dieser Parameter zur Verfügung.

Die denkbar einfachste Vorgehensweise besteht darin, anhand einer
beobachteten Renditereihe ihren historischen Mittelwert und empirische
Varianz als Schätzer für den Erwartungswert der Rendite und die
zukünftige Varianz zu ermitteln. In gleicher Weise kann man die
empirische Kovarianz zweier Renditereihen als Schätzer der zukünftigen
Kovarianz berechnen. Diese Vorgehensweise wird auch als \textbf{einfache
historisch basierte Schätzung} bezeichnet. Sie wird im Folgenden wegen
ihrer leichten Umsetzbarkeit und Anschaulichkeit als Verfahren zur
Rendite- bzw. Risikoprognose eingesetzt, obwohl die damit erzielbaren
Ergebnisse im Regelfall eher schlecht sind. Die Verwendung begründet
sich hier ausschließlich mit der didaktischen Zielsetzung, die
verschiedenen Verfahren der Portfoliooptimierung verfahrenstechnisch zu
demonstrieren, wofür zwar Prognosen benötigt werden, deren Güte aber für
den hier verfolgten Zweck keine Rolle spielt.

Poddig et al.~(2009, S. 116-121) geben einen Überblick über komplexere
Methoden zur Prognose von Renditen und Risiken.

Wir berechnen zunächst diskrete Monatsrenditen über
\texttt{frame.pct\_change()}, erhöhen diese um Eins und berechnen davon
den Logarithmus (über die Funktion \texttt{log1p}) um stetige (Log-)
Renditen zu erhalten. Hierauf wenden wir die Schätzung an und übertragen
die historisch geschätzen (und annualiserten) Mittelwerte und die
empirische Varianz-Kovarianzmatrix in die Arrays \texttt{means} und
\texttt{Sigma}.

\begin{Shaded}
\begin{Highlighting}[]
\CommentTok{\# simple, historical estimation based on log returns}
\NormalTok{log\_returns }\OperatorTok{=}\NormalTok{ np.log1p(frame.pct\_change().dropna()) }
\NormalTok{means }\OperatorTok{=}\NormalTok{ log\_returns.mean().values}\OperatorTok{*}\DecValTok{100}\OperatorTok{*}\DecValTok{12} \CommentTok{\# annualised}
\NormalTok{Sigma }\OperatorTok{=}\NormalTok{ log\_returns.cov().values}\OperatorTok{*}\DecValTok{12} \CommentTok{\# annualised}
\end{Highlighting}
\end{Shaded}

\subsection{Anwendung der absoluten
Portfoliooptimierung}\label{anwendung-der-absoluten-portfoliooptimierung}

\subsubsection{1. Berechnung des Minimum-Varianz-Portfolios
(MVP)}\label{berechnung-des-minimum-varianz-portfolios-mvp}

\paragraph{Ohne Leerverkaufsverbot}\label{ohne-leerverkaufsverbot}

Wir beginnen mit der Formulierung der Zielfunktion des
Optimierungsproblems OP2 in der Funktion
\texttt{calculate\_portfolio\_var}. Über \texttt{matrix(w)}
transformieren wir das Array \texttt{w} in einen Zeilenvektor der
Dimension \(1xN\).

\begin{Shaded}
\begin{Highlighting}[]
\CommentTok{\# definition of target function to be minimized}
\KeywordTok{def}\NormalTok{ calculate\_portfolio\_var(w,Sigma):}
    \CommentTok{\# function that calculates portfolio risk}
\NormalTok{    w }\OperatorTok{=}\NormalTok{ np.matrix(w) }\CommentTok{\# w is a row (not column!) vector}
    \ControlFlowTok{return}\NormalTok{ (w}\OperatorTok{*}\NormalTok{Sigma}\OperatorTok{*}\NormalTok{w.T)[}\DecValTok{0}\NormalTok{,}\DecValTok{0}\NormalTok{]}
\end{Highlighting}
\end{Shaded}

Für die Startgewichte der Optimierung verwenden wir ein
gleichgewichtetes Portfolio. Durch Aufruf von \texttt{np.tile(A,x)} wird
ein Array generiert, welches x-Mal den Wert A enthält. Die Anzahl der
Assets in unserem Beispieluniversum erhalten wir über
\texttt{means.shape{[}0{]}}.

\begin{Shaded}
\begin{Highlighting}[]
\CommentTok{\# use equal weights "Weight\_1N" as starting values }
\NormalTok{Weight\_1N }\OperatorTok{=}\NormalTok{ np.tile(}\FloatTok{1.0}\OperatorTok{/}\NormalTok{means.shape[}\DecValTok{0}\NormalTok{], means.shape[}\DecValTok{0}\NormalTok{])}
\end{Highlighting}
\end{Shaded}

Nun sind wir bereit für die Optimierung, die optimierten Anteilsgewichte
speichern wir im Array \texttt{Weight\_MV1}. Durch setzen von
\texttt{options=\{\textquotesingle{}disp\textquotesingle{}:\ True\}}
werden allgemeine Informationen zum Ablauf der Optimierung angezeigt,
und \texttt{tol} legt die Genauigkeit der gefundenen Lösung fest.

\begin{Shaded}
\begin{Highlighting}[]
\CommentTok{\# unconstrained portfolio (only sum(w) = 1 )}
\NormalTok{cons }\OperatorTok{=}\NormalTok{ (\{}\StringTok{\textquotesingle{}type\textquotesingle{}}\NormalTok{: }\StringTok{\textquotesingle{}eq\textquotesingle{}}\NormalTok{, }\StringTok{\textquotesingle{}fun\textquotesingle{}}\NormalTok{: }\KeywordTok{lambda}\NormalTok{ x:  np.}\BuiltInTok{sum}\NormalTok{(x)}\OperatorTok{{-}}\FloatTok{1.0}\NormalTok{\})}
\NormalTok{res1}\OperatorTok{=}\NormalTok{ minimize(calculate\_portfolio\_var, Weight\_1N, args}\OperatorTok{=}\NormalTok{Sigma,}
\NormalTok{              method}\OperatorTok{=}\StringTok{\textquotesingle{}SLSQP\textquotesingle{}}\NormalTok{,constraints}\OperatorTok{=}\NormalTok{cons, tol}\OperatorTok{=}\FloatTok{1e{-}10}\NormalTok{,}
\NormalTok{              options}\OperatorTok{=}\NormalTok{\{}\StringTok{\textquotesingle{}disp\textquotesingle{}}\NormalTok{: }\VariableTok{True}\NormalTok{\})}
\NormalTok{Weight\_MV1 }\OperatorTok{=}\NormalTok{ res1.x}
\end{Highlighting}
\end{Shaded}

\begin{verbatim}
Optimization terminated successfully    (Exit mode 0)
            Current function value: 0.010828658442203366
            Iterations: 31
            Function evaluations: 341
            Gradient evaluations: 31
\end{verbatim}

Aus Gründen der besseren Übersicht transformieren wir
\texttt{Weight\_MV1} in ein DataFrame.

\begin{Shaded}
\begin{Highlighting}[]
\NormalTok{pd.DataFrame([}\BuiltInTok{round}\NormalTok{(x,}\DecValTok{4}\NormalTok{) }\ControlFlowTok{for}\NormalTok{ x }\KeywordTok{in}\NormalTok{ Weight\_MV1],index}\OperatorTok{=}\NormalTok{frame.columns).T}
\end{Highlighting}
\end{Shaded}

\begin{longtable}[]{@{}lllllllllll@{}}
\toprule\noalign{}
& ABT & BA & COST & CSCO & IBM & INTC & MRK & MSFT & T & XOM \\
\midrule\noalign{}
\endhead
\bottomrule\noalign{}
\endlastfoot
0 & 0.4359 & -0.0418 & 0.2083 & -0.0125 & 0.141 & 0.0902 & -0.1451 &
-0.0239 & 0.0445 & 0.3035 \\
\end{longtable}

Positive Gewichte stellen Long-Positionen und negative Gewichte
Short-Positionen dar. Assets mit negativen Gewichten werden somit
leerverkauft.

\paragraph{Leerverkaufsverbot}\label{leerverkaufsverbot}

Wir implementieren nun die Beschränkung auf positive Gewichte über eine
Liste von zehn (0, 1)-Tuplen.

\begin{Shaded}
\begin{Highlighting}[]
\CommentTok{\# positive weight portfolio}
\NormalTok{bnd}\OperatorTok{=}\NormalTok{[(}\DecValTok{0}\NormalTok{, }\DecValTok{1}\NormalTok{),(}\DecValTok{0}\NormalTok{, }\DecValTok{1}\NormalTok{),(}\DecValTok{0}\NormalTok{, }\DecValTok{1}\NormalTok{),(}\DecValTok{0}\NormalTok{, }\DecValTok{1}\NormalTok{),(}\DecValTok{0}\NormalTok{, }\DecValTok{1}\NormalTok{),}
\NormalTok{     (}\DecValTok{0}\NormalTok{, }\DecValTok{1}\NormalTok{),(}\DecValTok{0}\NormalTok{, }\DecValTok{1}\NormalTok{),(}\DecValTok{0}\NormalTok{, }\DecValTok{1}\NormalTok{),(}\DecValTok{0}\NormalTok{, }\DecValTok{1}\NormalTok{),(}\DecValTok{0}\NormalTok{, }\DecValTok{1}\NormalTok{)] }\CommentTok{\# only positive weights}
\NormalTok{cons }\OperatorTok{=}\NormalTok{ (\{}\StringTok{\textquotesingle{}type\textquotesingle{}}\NormalTok{: }\StringTok{\textquotesingle{}eq\textquotesingle{}}\NormalTok{, }\StringTok{\textquotesingle{}fun\textquotesingle{}}\NormalTok{: }\KeywordTok{lambda}\NormalTok{ x:  np.}\BuiltInTok{sum}\NormalTok{(x)}\OperatorTok{{-}}\FloatTok{1.0}\NormalTok{\})}

\NormalTok{res2}\OperatorTok{=}\NormalTok{ minimize(calculate\_portfolio\_var, Weight\_1N, args}\OperatorTok{=}\NormalTok{Sigma, }
\NormalTok{               bounds }\OperatorTok{=}\NormalTok{ bnd, method}\OperatorTok{=}\StringTok{\textquotesingle{}SLSQP\textquotesingle{}}\NormalTok{,constraints}\OperatorTok{=}\NormalTok{cons,tol}\OperatorTok{=}\FloatTok{1e{-}10}\NormalTok{, }
\NormalTok{               options}\OperatorTok{=}\NormalTok{\{}\StringTok{\textquotesingle{}disp\textquotesingle{}}\NormalTok{: }\VariableTok{True}\NormalTok{\})}
\NormalTok{Weight\_MV2 }\OperatorTok{=}\NormalTok{ res2.x}
\end{Highlighting}
\end{Shaded}

\begin{verbatim}
Optimization terminated successfully    (Exit mode 0)
            Current function value: 0.012112846573987465
            Iterations: 21
            Function evaluations: 231
            Gradient evaluations: 21
\end{verbatim}

\begin{Shaded}
\begin{Highlighting}[]
\NormalTok{pd.DataFrame([}\BuiltInTok{round}\NormalTok{(x,}\DecValTok{4}\NormalTok{) }\ControlFlowTok{for}\NormalTok{ x }\KeywordTok{in}\NormalTok{ Weight\_MV2],index}\OperatorTok{=}\NormalTok{frame.columns).T}
\end{Highlighting}
\end{Shaded}

\begin{longtable}[]{@{}lllllllllll@{}}
\toprule\noalign{}
& ABT & BA & COST & CSCO & IBM & INTC & MRK & MSFT & T & XOM \\
\midrule\noalign{}
\endhead
\bottomrule\noalign{}
\endlastfoot
0 & 0.3617 & 0.0 & 0.1652 & 0.0 & 0.1621 & 0.0137 & 0.0 & 0.0 & 0.0523 &
0.2449 \\
\end{longtable}

In der gefundenen Lösung fällt auf, das vier der zehn Aktien nicht im
optimierten Portfolio enthalten sind und mehr als 60\% des Portfolios
auf die beiden Aktien ABT und XOM entfällt. Um ein stärker
diversifiziertes Portfolio zu konstruieren, wird bei der Berechnung
eines dritten MVP eine zusätzliche Nebenbedingung eingeführt. Die
einzelnen Assets sollen mindestens mit einem Anteilsgewicht von 5\%,
aber maximal mit einem Höchstanteil von 35\% im Portfolio enthalten
sein.

\subsubsection{Zusätzliche Nebenbedingung: Bestandsgrenzen (min=5\%,
max=35\%)}\label{zusuxe4tzliche-nebenbedingung-bestandsgrenzen-min5-max35}

\begin{Shaded}
\begin{Highlighting}[]
\CommentTok{\# position constraints}
\NormalTok{bnd}\OperatorTok{=}\NormalTok{[(}\FloatTok{0.05}\NormalTok{, }\FloatTok{0.35}\NormalTok{),(}\FloatTok{0.05}\NormalTok{, }\FloatTok{0.35}\NormalTok{),(}\FloatTok{0.05}\NormalTok{, }\FloatTok{0.35}\NormalTok{),(}\FloatTok{0.05}\NormalTok{, }\FloatTok{0.35}\NormalTok{),(}\FloatTok{0.05}\NormalTok{, }\FloatTok{0.35}\NormalTok{),}
\NormalTok{     (}\FloatTok{0.05}\NormalTok{, }\FloatTok{0.35}\NormalTok{),(}\FloatTok{0.05}\NormalTok{, }\FloatTok{0.35}\NormalTok{),(}\FloatTok{0.05}\NormalTok{, }\FloatTok{0.35}\NormalTok{),(}\FloatTok{0.05}\NormalTok{, }\FloatTok{0.35}\NormalTok{),(}\FloatTok{0.05}\NormalTok{, }\FloatTok{0.35}\NormalTok{)]}
\NormalTok{cons }\OperatorTok{=}\NormalTok{ (\{}\StringTok{\textquotesingle{}type\textquotesingle{}}\NormalTok{: }\StringTok{\textquotesingle{}eq\textquotesingle{}}\NormalTok{, }\StringTok{\textquotesingle{}fun\textquotesingle{}}\NormalTok{: }\KeywordTok{lambda}\NormalTok{ x:  np.}\BuiltInTok{sum}\NormalTok{(x)}\OperatorTok{{-}}\FloatTok{1.0}\NormalTok{\})}

\NormalTok{res3}\OperatorTok{=}\NormalTok{ minimize(calculate\_portfolio\_var, Weight\_1N, args}\OperatorTok{=}\NormalTok{Sigma, }
\NormalTok{               bounds }\OperatorTok{=}\NormalTok{ bnd, method}\OperatorTok{=}\StringTok{\textquotesingle{}SLSQP\textquotesingle{}}\NormalTok{,constraints}\OperatorTok{=}\NormalTok{cons,tol}\OperatorTok{=}\FloatTok{1e{-}10}\NormalTok{, }
\NormalTok{               options}\OperatorTok{=}\NormalTok{\{}\StringTok{\textquotesingle{}disp\textquotesingle{}}\NormalTok{: }\VariableTok{True}\NormalTok{\})}
\NormalTok{Weight\_MV3 }\OperatorTok{=}\NormalTok{ res3.x}
\end{Highlighting}
\end{Shaded}

\begin{verbatim}
Optimization terminated successfully    (Exit mode 0)
            Current function value: 0.014612315837805704
            Iterations: 18
            Function evaluations: 198
            Gradient evaluations: 18
\end{verbatim}

\begin{Shaded}
\begin{Highlighting}[]
\NormalTok{pd.DataFrame([}\BuiltInTok{round}\NormalTok{(x,}\DecValTok{4}\NormalTok{) }\ControlFlowTok{for}\NormalTok{ x }\KeywordTok{in}\NormalTok{ Weight\_MV3],index}\OperatorTok{=}\NormalTok{frame.columns).T}
\end{Highlighting}
\end{Shaded}

\begin{longtable}[]{@{}lllllllllll@{}}
\toprule\noalign{}
& ABT & BA & COST & CSCO & IBM & INTC & MRK & MSFT & T & XOM \\
\midrule\noalign{}
\endhead
\bottomrule\noalign{}
\endlastfoot
0 & 0.3313 & 0.05 & 0.0689 & 0.05 & 0.1062 & 0.05 & 0.05 & 0.05 & 0.05 &
0.1935 \\
\end{longtable}

\subsubsection{2. Berechnung des Maximum-Ertrag-Portfolios
(MEP)}\label{berechnung-des-maximum-ertrag-portfolios-mep}

\paragraph{Mit Leerverkaufsverbot}\label{mit-leerverkaufsverbot}

Wir beginnen wieder mit der Formulierung der Zielfunktion des
Optimierungsproblems OP3 in der Funktion
\texttt{calculate\_negative\_portfolio\_ret}. \textbf{Beachte}:
Minimieren der mit -1 multiplizierten erwarteten Portfoliorendite ist
äquivalent zur Renditemaximierung.

\begin{Shaded}
\begin{Highlighting}[]
\CommentTok{\# definition of target function to be minimized}
\KeywordTok{def}\NormalTok{ calculate\_negative\_portfolio\_ret(w,means):}
    \CommentTok{\# function that calculates {-}1 times portfolio expected return}
\NormalTok{    w }\OperatorTok{=}\NormalTok{ np.matrix(w) }\CommentTok{\# w is a row (not column!) vector}
\NormalTok{    means }\OperatorTok{=}\NormalTok{ np.matrix(means)}
    \ControlFlowTok{return} \OperatorTok{{-}}\NormalTok{(w}\OperatorTok{*}\NormalTok{means.T)[}\DecValTok{0}\NormalTok{,}\DecValTok{0}\NormalTok{]}
\end{Highlighting}
\end{Shaded}

\begin{Shaded}
\begin{Highlighting}[]
\CommentTok{\# positive weight portfolio}
\NormalTok{bnd}\OperatorTok{=}\NormalTok{[(}\DecValTok{0}\NormalTok{, }\DecValTok{1}\NormalTok{),(}\DecValTok{0}\NormalTok{, }\DecValTok{1}\NormalTok{),(}\DecValTok{0}\NormalTok{, }\DecValTok{1}\NormalTok{),(}\DecValTok{0}\NormalTok{, }\DecValTok{1}\NormalTok{),(}\DecValTok{0}\NormalTok{, }\DecValTok{1}\NormalTok{),}
\NormalTok{     (}\DecValTok{0}\NormalTok{, }\DecValTok{1}\NormalTok{),(}\DecValTok{0}\NormalTok{, }\DecValTok{1}\NormalTok{),(}\DecValTok{0}\NormalTok{, }\DecValTok{1}\NormalTok{),(}\DecValTok{0}\NormalTok{, }\DecValTok{1}\NormalTok{),(}\DecValTok{0}\NormalTok{, }\DecValTok{1}\NormalTok{)] }\CommentTok{\# only positive weights}
\NormalTok{cons }\OperatorTok{=}\NormalTok{ (\{}\StringTok{\textquotesingle{}type\textquotesingle{}}\NormalTok{: }\StringTok{\textquotesingle{}eq\textquotesingle{}}\NormalTok{, }\StringTok{\textquotesingle{}fun\textquotesingle{}}\NormalTok{: }\KeywordTok{lambda}\NormalTok{ x:  np.}\BuiltInTok{sum}\NormalTok{(x)}\OperatorTok{{-}}\FloatTok{1.0}\NormalTok{\})}

\NormalTok{res4}\OperatorTok{=}\NormalTok{ minimize(calculate\_negative\_portfolio\_ret, Weight\_1N, args}\OperatorTok{=}\NormalTok{means, }
\NormalTok{               bounds }\OperatorTok{=}\NormalTok{ bnd, method}\OperatorTok{=}\StringTok{\textquotesingle{}SLSQP\textquotesingle{}}\NormalTok{,constraints}\OperatorTok{=}\NormalTok{cons,tol}\OperatorTok{=}\FloatTok{1e{-}10}\NormalTok{, }
\NormalTok{               options}\OperatorTok{=}\NormalTok{\{}\StringTok{\textquotesingle{}disp\textquotesingle{}}\NormalTok{: }\VariableTok{True}\NormalTok{\})}
\NormalTok{Weight\_MRP }\OperatorTok{=}\NormalTok{ res4.x}
\end{Highlighting}
\end{Shaded}

\begin{verbatim}
Optimization terminated successfully    (Exit mode 0)
            Current function value: -7.886975933686317
            Iterations: 2
            Function evaluations: 22
            Gradient evaluations: 2
\end{verbatim}

\begin{Shaded}
\begin{Highlighting}[]
\NormalTok{pd.DataFrame([}\BuiltInTok{round}\NormalTok{(x,}\DecValTok{4}\NormalTok{) }\ControlFlowTok{for}\NormalTok{ x }\KeywordTok{in}\NormalTok{ Weight\_MRP],index}\OperatorTok{=}\NormalTok{frame.columns).T}
\end{Highlighting}
\end{Shaded}

\begin{longtable}[]{@{}lllllllllll@{}}
\toprule\noalign{}
& ABT & BA & COST & CSCO & IBM & INTC & MRK & MSFT & T & XOM \\
\midrule\noalign{}
\endhead
\bottomrule\noalign{}
\endlastfoot
0 & 0.0 & 0.0 & 0.0 & 0.0 & 0.0 & 0.0 & 0.0 & 0.0 & 0.0 & 1.0 \\
\end{longtable}

Das Portfolio ist zu 100\% investiert in XOM, der Aktie mit der höchsten
mittleren Rendite im Datensatz. Um ein breiter diversifiziertes
Portfolio zu erhalten könnten wir wieder Bestandsgrenzen als zusätzliche
Nebenbedingung einführen.

\subsubsection{3. Berechnung eines Punktes der
Effizienzkurve}\label{berechnung-eines-punktes-der-effizienzkurve}

\paragraph{Mit Leerverkaufsverbot}\label{mit-leerverkaufsverbot-1}

Wir bestimmen nun einen Punkt der Effizienzkurve in dem wir im
Optimierungsproblem OP1 die Zielrendite \(\mu_P\) z.B. auf 6\% setzen.
Wir implementieren diese zusätzliche Nebenbedingung in zwei Schritten.
Zunächst schreiben wir die Funktion \texttt{calculate\_portfolio\_ret},
die die erwartete Portfoliorendite berechnet. Dann fügen wir diese
Funktion als \texttt{Lambda}-Funktion in das Tuple \texttt{cons} der
Nebenbedingungs-Dictionaries.

\begin{Shaded}
\begin{Highlighting}[]
\CommentTok{\# definition of function for portfolio expected return}
\KeywordTok{def}\NormalTok{ calculate\_portfolio\_ret(w,means):}
\NormalTok{    w }\OperatorTok{=}\NormalTok{ np.matrix(w) }\CommentTok{\# w is a row (not column!) vector}
\NormalTok{    means }\OperatorTok{=}\NormalTok{ np.matrix(means)}
    \ControlFlowTok{return}\NormalTok{ (w}\OperatorTok{*}\NormalTok{means.T)[}\DecValTok{0}\NormalTok{,}\DecValTok{0}\NormalTok{]}

\CommentTok{\# positive weight portfolio}
\NormalTok{bnd}\OperatorTok{=}\NormalTok{[(}\DecValTok{0}\NormalTok{, }\DecValTok{1}\NormalTok{),(}\DecValTok{0}\NormalTok{, }\DecValTok{1}\NormalTok{),(}\DecValTok{0}\NormalTok{, }\DecValTok{1}\NormalTok{),(}\DecValTok{0}\NormalTok{, }\DecValTok{1}\NormalTok{),(}\DecValTok{0}\NormalTok{, }\DecValTok{1}\NormalTok{),}
\NormalTok{     (}\DecValTok{0}\NormalTok{, }\DecValTok{1}\NormalTok{),(}\DecValTok{0}\NormalTok{, }\DecValTok{1}\NormalTok{),(}\DecValTok{0}\NormalTok{, }\DecValTok{1}\NormalTok{),(}\DecValTok{0}\NormalTok{, }\DecValTok{1}\NormalTok{),(}\DecValTok{0}\NormalTok{, }\DecValTok{1}\NormalTok{)] }\CommentTok{\# only positive weights}

\NormalTok{cons }\OperatorTok{=}\NormalTok{ (\{}\StringTok{\textquotesingle{}type\textquotesingle{}}\NormalTok{: }\StringTok{\textquotesingle{}eq\textquotesingle{}}\NormalTok{, }\StringTok{\textquotesingle{}fun\textquotesingle{}}\NormalTok{: }\KeywordTok{lambda}\NormalTok{ x:  np.}\BuiltInTok{sum}\NormalTok{(x)}\OperatorTok{{-}}\FloatTok{1.0}\NormalTok{\},}
\NormalTok{        \{}\StringTok{\textquotesingle{}type\textquotesingle{}}\NormalTok{: }\StringTok{\textquotesingle{}eq\textquotesingle{}}\NormalTok{, }\StringTok{\textquotesingle{}fun\textquotesingle{}}\NormalTok{: }\KeywordTok{lambda}\NormalTok{ x:  calculate\_portfolio\_ret(x,means)}\OperatorTok{{-}}\FloatTok{6.0}\NormalTok{\})}


\NormalTok{res4}\OperatorTok{=}\NormalTok{ minimize(calculate\_portfolio\_var, Weight\_1N, args}\OperatorTok{=}\NormalTok{Sigma, }
\NormalTok{               bounds }\OperatorTok{=}\NormalTok{ bnd, method}\OperatorTok{=}\StringTok{\textquotesingle{}SLSQP\textquotesingle{}}\NormalTok{,constraints}\OperatorTok{=}\NormalTok{cons,tol}\OperatorTok{=}\FloatTok{1e{-}10}\NormalTok{, }
\NormalTok{               options}\OperatorTok{=}\NormalTok{\{}\StringTok{\textquotesingle{}disp\textquotesingle{}}\NormalTok{: }\VariableTok{True}\NormalTok{\})}
\NormalTok{Weight\_tar6 }\OperatorTok{=}\NormalTok{ res4.x}
\end{Highlighting}
\end{Shaded}

\begin{verbatim}
Optimization terminated successfully    (Exit mode 0)
            Current function value: 0.0149725199855123
            Iterations: 13
            Function evaluations: 143
            Gradient evaluations: 13
\end{verbatim}

Wir überprüfen kurz ob die optimierten Portfoliogewichte wirklich zu
einer erwarteten Portfoliorendite von 6\% führen.

\begin{Shaded}
\begin{Highlighting}[]
\NormalTok{calculate\_portfolio\_ret(Weight\_tar6,means)}
\end{Highlighting}
\end{Shaded}

\begin{verbatim}
6.000000000000562
\end{verbatim}

\begin{Shaded}
\begin{Highlighting}[]
\NormalTok{pd.DataFrame([}\BuiltInTok{round}\NormalTok{(x,}\DecValTok{4}\NormalTok{) }\ControlFlowTok{for}\NormalTok{ x }\KeywordTok{in}\NormalTok{ Weight\_tar6],index}\OperatorTok{=}\NormalTok{frame.columns).T}
\end{Highlighting}
\end{Shaded}

\begin{longtable}[]{@{}lllllllllll@{}}
\toprule\noalign{}
& ABT & BA & COST & CSCO & IBM & INTC & MRK & MSFT & T & XOM \\
\midrule\noalign{}
\endhead
\bottomrule\noalign{}
\endlastfoot
0 & 0.2001 & 0.0 & 0.149 & 0.0 & 0.1609 & 0.0 & 0.0 & 0.0 & 0.0 &
0.49 \\
\end{longtable}

\subsubsection{4. Berechnung der vollständigen
Effizienkurve}\label{berechnung-der-vollstuxe4ndigen-effizienkurve}

\paragraph{Mit Leerverkaufsverbot}\label{mit-leerverkaufsverbot-2}

Zunächst berechnen wir das Intervall der erwarteten Renditen aller
Portfolios auf der Effizienzkurve. Dies ist gleichbedeutend mit dem
Intervall der Zielrenditen im Optimierungsproblem OP1. Hierbei gilt:
Minimum Zielrendite = Rendite des MVP, und Maximum Zielrendite = Rendite
der MEP.

\begin{Shaded}
\begin{Highlighting}[]
\CommentTok{\# calculation of min and max target return}
\CommentTok{\# min: expected return of GMVP, max: expected return of MRP}

\BuiltInTok{min} \OperatorTok{=}\NormalTok{ calculate\_portfolio\_ret(Weight\_MV2, means)}
\BuiltInTok{max} \OperatorTok{=}\NormalTok{ calculate\_portfolio\_ret(Weight\_MRP, means)}
\end{Highlighting}
\end{Shaded}

Über \texttt{np.linspace} generieren wir nun das Array
\texttt{V\_Target} mit 45 gleichmäßig verteilten Zielrenditen zwischen
den oben definierten Minimum und Maximum. Wir definieren die Arrays
\texttt{V\_Risk} und \texttt{V\_Return} in denen wir später die aus der
Optimierung resultierenden \(\sigma_P\) und \(\mu_P\) speichern. Zudem
definieren wir die Matrix \texttt{V\_Weight} der Dimension 45x10. In den
Zeilen dieser Matrix speichern wir die zehn Anteilsgewichte der zehn
Assets für die 45 optimierten Portfolios (eins für jede Zielrendite).

Nun iterieren wir über eine for-Schleife durch das Array der
Zielrenditen. Wir verwenden \texttt{enumerate} um den Iterationsindex
(hier als \texttt{idx} bezeichnet) bei den Iterationen ``mitzunehmen''
und damit die Positionen in den Ergebnis-Arrays bzw. der Gewichtsmatrix
zu indexieren.

\begin{Shaded}
\begin{Highlighting}[]
\NormalTok{V\_Target }\OperatorTok{=}\NormalTok{ np.linspace(}\BuiltInTok{min}\NormalTok{, }\BuiltInTok{max}\NormalTok{, num}\OperatorTok{=}\DecValTok{45}\NormalTok{)}
\NormalTok{V\_Risk }\OperatorTok{=}\NormalTok{ np.zeros(V\_Target.shape)}
\NormalTok{V\_Return }\OperatorTok{=}\NormalTok{ np.zeros(V\_Target.shape)}
\NormalTok{V\_Weight }\OperatorTok{=}\NormalTok{ np.zeros((V\_Target.shape[}\DecValTok{0}\NormalTok{], means.shape[}\DecValTok{0}\NormalTok{]))}
\ControlFlowTok{for}\NormalTok{ idx, Target\_Return }\KeywordTok{in} \BuiltInTok{enumerate}\NormalTok{(V\_Target):}
\NormalTok{    cons }\OperatorTok{=}\NormalTok{ (\{}\StringTok{\textquotesingle{}type\textquotesingle{}}\NormalTok{: }\StringTok{\textquotesingle{}eq\textquotesingle{}}\NormalTok{, }\StringTok{\textquotesingle{}fun\textquotesingle{}}\NormalTok{: }\KeywordTok{lambda}\NormalTok{ x:  np.}\BuiltInTok{sum}\NormalTok{(x)}\OperatorTok{{-}}\FloatTok{1.0}\NormalTok{\},}
\NormalTok{            \{}\StringTok{\textquotesingle{}type\textquotesingle{}}\NormalTok{: }\StringTok{\textquotesingle{}eq\textquotesingle{}}\NormalTok{, }\StringTok{\textquotesingle{}fun\textquotesingle{}}\NormalTok{: }\KeywordTok{lambda}\NormalTok{ x:  calculate\_portfolio\_ret(x,means)}\OperatorTok{{-}}\NormalTok{Target\_Return\})}
\NormalTok{    res}\OperatorTok{=}\NormalTok{ minimize(calculate\_portfolio\_var, Weight\_1N, args}\OperatorTok{=}\NormalTok{Sigma, }
\NormalTok{               bounds }\OperatorTok{=}\NormalTok{ bnd, method}\OperatorTok{=}\StringTok{\textquotesingle{}SLSQP\textquotesingle{}}\NormalTok{,constraints}\OperatorTok{=}\NormalTok{cons,tol}\OperatorTok{=}\FloatTok{1e{-}10}\NormalTok{)}
\NormalTok{    V\_Weight[idx, :] }\OperatorTok{=}\NormalTok{ res.x.T}
\NormalTok{    V\_Return[idx] }\OperatorTok{=}\NormalTok{ calculate\_portfolio\_ret(res.x,means)}
\NormalTok{    V\_Risk[idx] }\OperatorTok{=}\NormalTok{ np.sqrt(calculate\_portfolio\_var(res.x, Sigma))}
\end{Highlighting}
\end{Shaded}

Wir plotten die resultierende Effizienkurve und die Positionen der zehn
Assets.

\begin{Shaded}
\begin{Highlighting}[]
\NormalTok{fig1 }\OperatorTok{=}\NormalTok{ plt.figure(num}\OperatorTok{=}\DecValTok{1}\NormalTok{, facecolor}\OperatorTok{=}\StringTok{\textquotesingle{}w\textquotesingle{}}\NormalTok{, figsize}\OperatorTok{=}\NormalTok{(}\DecValTok{10}\NormalTok{, }\DecValTok{5}\NormalTok{))}
\NormalTok{plt.plot(V\_Risk, V\_Target, }\StringTok{\textquotesingle{}g:\textquotesingle{}}\NormalTok{, label}\OperatorTok{=}\StringTok{\textquotesingle{}Efficient frontier without short selling\textquotesingle{}}\NormalTok{)}
\NormalTok{plt.plot(np.sqrt(np.diagonal(Sigma)), means, }\StringTok{\textquotesingle{}rx\textquotesingle{}}\NormalTok{, label}\OperatorTok{=}\StringTok{\textquotesingle{}Asset\textquotesingle{}}\NormalTok{)}
\NormalTok{plt.legend(loc}\OperatorTok{=}\StringTok{\textquotesingle{}best\textquotesingle{}}\NormalTok{,  frameon}\OperatorTok{=}\VariableTok{False}\NormalTok{)}
\NormalTok{plt.xlabel(}\StringTok{\textquotesingle{}Standard deviation\textquotesingle{}}\NormalTok{)}
\NormalTok{plt.ylabel(}\StringTok{\textquotesingle{}Expected return (\%)\textquotesingle{}}\NormalTok{)}
\NormalTok{plt.show()}
\end{Highlighting}
\end{Shaded}

\pandocbounded{\includegraphics[keepaspectratio]{kapitel1_files/figure-pdf/cell-24-output-1.png}}

Hier plotten wir die optimalen Anteilsgewichte für die 45 Zielrenditen.

\begin{Shaded}
\begin{Highlighting}[]
\NormalTok{fig2 }\OperatorTok{=}\NormalTok{ plt.figure(num}\OperatorTok{=}\DecValTok{2}\NormalTok{, facecolor}\OperatorTok{=}\StringTok{\textquotesingle{}w\textquotesingle{}}\NormalTok{, figsize}\OperatorTok{=}\NormalTok{(}\DecValTok{10}\NormalTok{, }\DecValTok{5}\NormalTok{))}
\NormalTok{plt.stackplot(V\_Target, V\_Weight.T}\OperatorTok{*}\DecValTok{100}\NormalTok{)}
\CommentTok{\# colors=tuple([tuple(gray*np.ones(3)) for gray in np.linspace(0.4, 0.8, num=means.shape[0])]))}
\NormalTok{plt.axis([}\BuiltInTok{min}\NormalTok{, }\BuiltInTok{max}\NormalTok{, }\FloatTok{0.0}\NormalTok{, }\FloatTok{100.0}\NormalTok{])}
\NormalTok{plt.legend(}\BuiltInTok{list}\NormalTok{(frame.columns),}
\NormalTok{           loc}\OperatorTok{=}\StringTok{\textquotesingle{}upper left\textquotesingle{}}\NormalTok{, bbox\_to\_anchor}\OperatorTok{=}\NormalTok{(}\FloatTok{1.0}\NormalTok{, }\FloatTok{1.0}\NormalTok{), frameon}\OperatorTok{=}\VariableTok{False}\NormalTok{)}
\NormalTok{plt.xlabel(}\StringTok{\textquotesingle{}Target expected return (\%)\textquotesingle{}}\NormalTok{)}
\NormalTok{plt.ylabel(}\StringTok{\textquotesingle{}Allocation weight (\%)\textquotesingle{}}\NormalTok{)}
\NormalTok{plt.show()}
\end{Highlighting}
\end{Shaded}

\pandocbounded{\includegraphics[keepaspectratio]{kapitel1_files/figure-pdf/cell-25-output-1.png}}

\subsubsection{5. Berechnung der vollständigen Effizienzkurve ohne
Leerverkaufsverbot}\label{berechnung-der-vollstuxe4ndigen-effizienzkurve-ohne-leerverkaufsverbot}

Im Sonderfall ohne Leerverkaufsverbot (und nur Gültigkeit der
Budgetrestriktion) kann die Parabel varianzminimaler Portfolios auch
ohne numerische Optimierung über eine geschlossene Lösung ermittelt
werden (siehe oben). Wir berechnen nun die Parabel über Gleichung (1).\\
Hierfür benötigen wir das NumPy Module \texttt{numpy.linalg} für Lineare
Algebra, dass wir oben als \texttt{la} importiert hatten.
\textbf{Wichtig}: In Sonderfall ohne Leerverkaufsverbot gilt für die
Varianz des MVP: \(\sigma^2_{MVP}=\frac{1}{C}\), wobei
\(C=\iota^{T}\Sigma^{-1}\iota\) (siehe zur Herleitung z.B. Franzen und
Schäfer, 2018, S. 189).

\texttt{inv} bezeichnet die Funktion zur Berechnung der Inversen einer
Matrix, und \texttt{@} ist der Operator für Matrixmultiplikationen.
\textbf{Wichtig}: Beachten Sie, dass in Python \texttt{@} anwendbar ist,
solange die Länge eines Vektors mit der Länge der entsprechenden
Zeile/Spalte einer Matrix übereinstimmt. Wir müssen also den
Transponierungsoperator \texttt{T} nicht auf \texttt{means} oder
\texttt{iota} anwenden.

Zunächst generieren wir den Einheitsvektor der Dimension \(N\) und
extrahieren die Standardabweichungen der Assets.

\begin{Shaded}
\begin{Highlighting}[]
\NormalTok{iota }\OperatorTok{=}\NormalTok{ np.ones(means.shape)}
\NormalTok{Stdev }\OperatorTok{=}\NormalTok{ np.sqrt(np.diagonal(Sigma))}
\end{Highlighting}
\end{Shaded}

Dann berechnen wir die Inverse der Varianz-Kovarianzmatrix und die vier
Komponenten A, B, C, und D der Gleichung (1).

\begin{Shaded}
\begin{Highlighting}[]
\NormalTok{inv\_Sigma }\OperatorTok{=}\NormalTok{ la.inv(Sigma)}
\NormalTok{A }\OperatorTok{=}\NormalTok{ means }\OperatorTok{@}\NormalTok{ inv\_Sigma }\OperatorTok{@}\NormalTok{ iota}
\NormalTok{B }\OperatorTok{=}\NormalTok{ means }\OperatorTok{@}\NormalTok{ inv\_Sigma }\OperatorTok{@}\NormalTok{ means}
\NormalTok{C }\OperatorTok{=}\NormalTok{ iota }\OperatorTok{@}\NormalTok{ inv\_Sigma }\OperatorTok{@}\NormalTok{ iota}
\NormalTok{D }\OperatorTok{=}\NormalTok{ B }\OperatorTok{*}\NormalTok{ C }\OperatorTok{{-}}\NormalTok{ A }\OperatorTok{**} \DecValTok{2}
\end{Highlighting}
\end{Shaded}

Wir speichern die Standardabweichung des MVP als \texttt{sigma\_gmv}.
Diese dient als Minimum der \(\sigma_P\)'s für die wir die beiden
(effizienten und ineffizienten) \(\mu_P\)'s gemäß Gleichung (1)
berechnen. Das Maximum der \(\sigma_P\)'s legen wir in Form der
maximalen Standardabweichung (\texttt{np.max(Stdev)}) unter allen Assets
fest. Für 250 gleichmäßig verteilte \(\sigma_P\)'s innerhalb dieser
Grenzen berechnen wir den effizienten und ineffizienten Bereich der
Parabel varianzminimaler Portfolios (in den beiden Arrays
\texttt{mu\_p\_efficient} und \texttt{mu\_p\_inefficient}) .

\begin{Shaded}
\begin{Highlighting}[]
\NormalTok{sigma\_gmv }\OperatorTok{=} \FloatTok{1.0} \OperatorTok{/}\NormalTok{ np.sqrt(C)}
\NormalTok{sigma\_p }\OperatorTok{=}\NormalTok{ np.linspace(sigma\_gmv, np.}\BuiltInTok{max}\NormalTok{(Stdev), num}\OperatorTok{=}\DecValTok{250}\NormalTok{)}
\NormalTok{mu\_p\_efficient }\OperatorTok{=}\NormalTok{ (A }\OperatorTok{+}\NormalTok{ np.sqrt(np.}\BuiltInTok{abs}\NormalTok{(C }\OperatorTok{*}\NormalTok{ sigma\_p }\OperatorTok{**} \DecValTok{2} \OperatorTok{{-}} \FloatTok{1.0}\NormalTok{) }\OperatorTok{*}\NormalTok{ D)) }\OperatorTok{/}\NormalTok{ C}
\NormalTok{mu\_p\_inefficient }\OperatorTok{=}\NormalTok{ (A }\OperatorTok{{-}}\NormalTok{ np.sqrt(np.}\BuiltInTok{abs}\NormalTok{(C }\OperatorTok{*}\NormalTok{ sigma\_p }\OperatorTok{**} \DecValTok{2} \OperatorTok{{-}} \FloatTok{1.0}\NormalTok{) }\OperatorTok{*}\NormalTok{ D)) }\OperatorTok{/}\NormalTok{ C}
\end{Highlighting}
\end{Shaded}

Abschließend plotten wir die Effizienzkurve mit und ohne
Leerverkaufsverbot gemeinsam in einem Graph.

\begin{Shaded}
\begin{Highlighting}[]
\NormalTok{fig2 }\OperatorTok{=}\NormalTok{ plt.figure(num}\OperatorTok{=}\DecValTok{1}\NormalTok{, facecolor}\OperatorTok{=}\StringTok{\textquotesingle{}w\textquotesingle{}}\NormalTok{, figsize}\OperatorTok{=}\NormalTok{(}\DecValTok{10}\NormalTok{, }\DecValTok{5}\NormalTok{))}
\NormalTok{plt.plot(sigma\_p, mu\_p\_efficient, }\StringTok{\textquotesingle{}b{-}\textquotesingle{}}\NormalTok{, label}\OperatorTok{=}\StringTok{\textquotesingle{}Efficient frontier with short selling\textquotesingle{}}\NormalTok{)}
\NormalTok{plt.plot(V\_Risk, V\_Target, }\StringTok{\textquotesingle{}g:\textquotesingle{}}\NormalTok{, label}\OperatorTok{=}\StringTok{\textquotesingle{}Efficient frontier without short selling\textquotesingle{}}\NormalTok{)}
\NormalTok{plt.plot(np.sqrt(np.diagonal(Sigma)), means, }\StringTok{\textquotesingle{}rx\textquotesingle{}}\NormalTok{, label}\OperatorTok{=}\StringTok{\textquotesingle{}Asset\textquotesingle{}}\NormalTok{)}
\NormalTok{plt.legend(loc}\OperatorTok{=}\StringTok{\textquotesingle{}best\textquotesingle{}}\NormalTok{,  frameon}\OperatorTok{=}\VariableTok{False}\NormalTok{)}
\NormalTok{plt.xlabel(}\StringTok{\textquotesingle{}Standard deviation\textquotesingle{}}\NormalTok{)}
\NormalTok{plt.ylabel(}\StringTok{\textquotesingle{}Expected return (\%)\textquotesingle{}}\NormalTok{)}
\NormalTok{plt.show()}
\end{Highlighting}
\end{Shaded}

\pandocbounded{\includegraphics[keepaspectratio]{kapitel1_files/figure-pdf/cell-29-output-1.png}}

Wir erwartet liegt die Effizienzkurve ohne Leerverkaufsverbot oberhalb
derjenigen mit Leerverkaufsverbot. Zusätzliche Nebenbedingungen
schränken den Lösungsraum optimaler Portfolios immer weiter ein.

\subsubsection{6. Berechnung der Effizienzkurve auf Basis des Two-Fund
Theorems}\label{berechnung-der-effizienzkurve-auf-basis-des-two-fund-theorems}

\paragraph{Wichtig: Dieser Ansatz ist nur gültig wenn Leerverkäufe
erlaubt
sind!}\label{wichtig-dieser-ansatz-ist-nur-guxfcltig-wenn-leerverkuxe4ufe-erlaubt-sind}

Im ersten Schritt ermitteln wir die Gewichte (\texttt{port1} und
\texttt{port2}) von zwei Basisportfolios \(X\) und \(Y\) gemäß Gleichung
(2) oben. Wir wählen die Werte 2.0 und 4.0 für die Konstante \(c\).

Für beide Portfolios berechnen wir die erwartete Rendite, Varianz der
Rendite, und die Renditekovarianz zwischen \(X\) und \(Y\).

\textbf{Beachten Sie:} die Renditekovarianz ist gegeben durch
\(Cov(X,Y)=w^{T}_X\Sigma w_Y\)

\begin{Shaded}
\begin{Highlighting}[]
\CommentTok{\# use 2.0 and 4.0 as constant values}
\CommentTok{\# calculate the two basis portfolios 1 and 2}
\NormalTok{port1 }\OperatorTok{=}\NormalTok{ (inv\_Sigma}\OperatorTok{*}\NormalTok{np.matrix(means}\OperatorTok{{-}}\FloatTok{2.0}\NormalTok{).T)}\OperatorTok{/}\NormalTok{np.}\BuiltInTok{sum}\NormalTok{(inv\_Sigma}\OperatorTok{*}\NormalTok{np.matrix(means}\OperatorTok{{-}}\FloatTok{2.0}\NormalTok{).T)}
\NormalTok{port2 }\OperatorTok{=}\NormalTok{ (inv\_Sigma}\OperatorTok{*}\NormalTok{np.matrix(means}\OperatorTok{{-}}\FloatTok{4.0}\NormalTok{).T)}\OperatorTok{/}\NormalTok{np.}\BuiltInTok{sum}\NormalTok{(inv\_Sigma}\OperatorTok{*}\NormalTok{np.matrix(means}\OperatorTok{{-}}\FloatTok{4.0}\NormalTok{).T)}
\NormalTok{covariance}\OperatorTok{=}\NormalTok{(port1.T}\OperatorTok{*}\NormalTok{Sigma}\OperatorTok{*}\NormalTok{port2)[}\DecValTok{0}\NormalTok{,}\DecValTok{0}\NormalTok{]}
\NormalTok{ret1 }\OperatorTok{=}\NormalTok{ (np.matrix(means) }\OperatorTok{*}\NormalTok{ port1)[}\DecValTok{0}\NormalTok{,}\DecValTok{0}\NormalTok{] }
\NormalTok{ret2 }\OperatorTok{=}\NormalTok{ (np.matrix(means) }\OperatorTok{*}\NormalTok{ port2)[}\DecValTok{0}\NormalTok{,}\DecValTok{0}\NormalTok{] }
\NormalTok{var1 }\OperatorTok{=}\NormalTok{ (port1.T}\OperatorTok{*}\NormalTok{Sigma}\OperatorTok{*}\NormalTok{port1)[}\DecValTok{0}\NormalTok{,}\DecValTok{0}\NormalTok{]}
\NormalTok{var2 }\OperatorTok{=}\NormalTok{ (port2.T}\OperatorTok{*}\NormalTok{Sigma}\OperatorTok{*}\NormalTok{port2)[}\DecValTok{0}\NormalTok{,}\DecValTok{0}\NormalTok{]}
\end{Highlighting}
\end{Shaded}

Danach berechnen wir gemäß der Formeln (3a) und (3b) die erwartete
Rendite und die Renditevarianz für ein Portfolio
\(Z=\alpha X+(1-\alpha)Y\). Über \texttt{np.linspace} wählen wir dabei
250 gleichmäßig verteilte Werte für \(\alpha\) zwischen 0 und 1.5. Die
resultierende erwartete Rendite und das Risiko der 250 \(Z\)-Portfolios
speichern wir dann in den Arrays \texttt{Risk} und \texttt{Return}. Wir
verwenden wieder die \texttt{enumerate} Funktion um die Arrays zu
indexieren.

\begin{Shaded}
\begin{Highlighting}[]
\CommentTok{\# construction of portfolios from the two basis portfolios for }
\CommentTok{\# different weights w}
\NormalTok{weight }\OperatorTok{=}\NormalTok{ np.linspace(}\DecValTok{0}\NormalTok{, }\FloatTok{1.5}\NormalTok{, num}\OperatorTok{=}\DecValTok{250}\NormalTok{)}
\NormalTok{Risk }\OperatorTok{=}\NormalTok{ np.zeros(weight.shape)}
\NormalTok{Return }\OperatorTok{=}\NormalTok{ np.zeros(weight.shape)}
\ControlFlowTok{for}\NormalTok{ idx, w }\KeywordTok{in} \BuiltInTok{enumerate}\NormalTok{(weight):}
\NormalTok{    Return[idx] }\OperatorTok{=}\NormalTok{ w}\OperatorTok{*}\NormalTok{ret1 }\OperatorTok{+}\NormalTok{ (}\DecValTok{1}\OperatorTok{{-}}\NormalTok{w)}\OperatorTok{*}\NormalTok{ret2}
\NormalTok{    Risk[idx] }\OperatorTok{=}\NormalTok{ np.sqrt(w}\OperatorTok{**}\DecValTok{2} \OperatorTok{*}\NormalTok{ var1 }\OperatorTok{+}\NormalTok{ (}\DecValTok{1}\OperatorTok{{-}}\NormalTok{w)}\OperatorTok{**}\DecValTok{2} \OperatorTok{*}\NormalTok{ var2 }\OperatorTok{+} \DecValTok{2}\OperatorTok{*}\NormalTok{w}\OperatorTok{*}\NormalTok{(}\DecValTok{1}\OperatorTok{{-}}\NormalTok{w)}\OperatorTok{*}\NormalTok{covariance)}
    
\end{Highlighting}
\end{Shaded}

Final plotten wir die sich ergebende Effizienzkurve zusammen mit der
oben numerisch bestimmten Kurve für den Fall dass Leerverkäufe
unzulässig sind.

\begin{Shaded}
\begin{Highlighting}[]
\NormalTok{fig3 }\OperatorTok{=}\NormalTok{ plt.figure(num}\OperatorTok{=}\DecValTok{1}\NormalTok{, facecolor}\OperatorTok{=}\StringTok{\textquotesingle{}w\textquotesingle{}}\NormalTok{, figsize}\OperatorTok{=}\NormalTok{(}\DecValTok{10}\NormalTok{, }\DecValTok{5}\NormalTok{))}
\NormalTok{plt.plot(Risk, Return, }\StringTok{\textquotesingle{}b{-}\textquotesingle{}}\NormalTok{, label}\OperatorTok{=}\StringTok{\textquotesingle{}Efficient frontier with short selling\textquotesingle{}}\NormalTok{)}
\NormalTok{plt.plot(V\_Risk, V\_Target, }\StringTok{\textquotesingle{}g:\textquotesingle{}}\NormalTok{, label}\OperatorTok{=}\StringTok{\textquotesingle{}Efficient frontier without short selling\textquotesingle{}}\NormalTok{)}
\NormalTok{plt.plot(np.sqrt(np.diagonal(Sigma)), means, }\StringTok{\textquotesingle{}rx\textquotesingle{}}\NormalTok{, label}\OperatorTok{=}\StringTok{\textquotesingle{}Asset\textquotesingle{}}\NormalTok{)}
\NormalTok{plt.legend(loc}\OperatorTok{=}\StringTok{\textquotesingle{}best\textquotesingle{}}\NormalTok{,  frameon}\OperatorTok{=}\VariableTok{False}\NormalTok{)}
\NormalTok{plt.xlabel(}\StringTok{\textquotesingle{}Standard deviation\textquotesingle{}}\NormalTok{)}
\NormalTok{plt.ylabel(}\StringTok{\textquotesingle{}Expected return (\%)\textquotesingle{}}\NormalTok{)}
\NormalTok{plt.show()}
\end{Highlighting}
\end{Shaded}

\pandocbounded{\includegraphics[keepaspectratio]{kapitel1_files/figure-pdf/cell-32-output-1.png}}

\subsubsection{7. Bestimmung des Tangentialportfolios
(TP)}\label{bestimmung-des-tangentialportfolios-tp}

\paragraph{Mit Leerverkaufsverbot}\label{mit-leerverkaufsverbot-3}

Wie schon oben geschildert, ergibt sich unter der Annahme der Existenz
einer risikofreien Anlage eine neue Effizienzline. Die Bestimmung des
Tangentialportfolios TP (das Portfolio mit der maximalen
\emph{Sharpe-Ratio}) beinhaltet die Frage: Welches Portfolio maximiert
die Überschussrendite (über den risikolosen Zins hinaus) relativ zum
Portfoliorisiko?

Die Zielfunktion des Optimierungsproblems OP4 implementieren wir in der
Funktion \texttt{calc\_neg\_sharpe}. Diese Funktion erfordert als
Eingabe \(w\), \(\mu\), \(\Sigma\), \(r_f\) und die Zeitfrequenz
\texttt{freq} der Renditebeobachtungen. Liegen beispielsweise monatliche
Renditen vor, ist \texttt{freq=12}. Um als Ausgabe der Funktion die
(negative) \emph{annualisierte} Sharpe-Ratio zu erhalten, wird die
erwartete Portfoliorendite (\texttt{portfolio\_return}) mit
\texttt{freq} multipliziert, und die Standardabweichung der Rendite
(\texttt{portfolio\_std}) mit der Wurzel aus \texttt{freq}. \(r_f\)
sollte als Jahreszins angegeben werden. Die Annualisierung entfällt
natürlich (und \texttt{freq} ist auf den Wert Eins zu setzen) falls die
Schätzungen von \(\mu\) und \(\Sigma\) bereits annualisiert wurden (wie
in unserem Fall oben).

Beachten Sie, dass die Portfoliovarianz in Matrizenschreibweise durch
\(\sigma_P^2=w^{T}\Sigma w\) gegeben ist. In der Funktion führen wir die
Matrizenmultiplikation zweimal hintereinander über \texttt{np.dot}
durch.

\begin{Shaded}
\begin{Highlighting}[]
\CommentTok{\# definition of target function for maximum Sharpe portfolio}
\CommentTok{\# "freq" denotes the return frequency (daily=252, monthly=12, annual=1, etc.)}
\KeywordTok{def}\NormalTok{ calc\_neg\_sharpe(weights, mean\_returns, cov, rf, freq):}
\NormalTok{    portfolio\_return }\OperatorTok{=}\NormalTok{ np.}\BuiltInTok{sum}\NormalTok{(mean\_returns }\OperatorTok{*}\NormalTok{ weights) }\OperatorTok{*}\NormalTok{ freq}
\NormalTok{    portfolio\_std }\OperatorTok{=}\NormalTok{ np.sqrt(np.dot(weights.T, np.dot(cov, weights))) }\OperatorTok{*}\NormalTok{ np.sqrt(freq)}
\NormalTok{    sharpe\_ratio }\OperatorTok{=}\NormalTok{ (portfolio\_return }\OperatorTok{{-}}\NormalTok{ rf) }\OperatorTok{/}\NormalTok{ portfolio\_std}
    \ControlFlowTok{return} \OperatorTok{{-}}\NormalTok{sharpe\_ratio}
\end{Highlighting}
\end{Shaded}

Das vollständige Optimierungsproblem schreiben wir in die Funktion
\texttt{max\_sharpe\_ratio}. Es gilt hier zwei Besonderheiten zu
beachten:

\begin{enumerate}
\def\labelenumi{\arabic{enumi}.}
\tightlist
\item
  Das Tuple \texttt{bounds} mit den identischen Positionsgrenzen für
  alle Assets generieren wir elegant über eine for-Schleife.
\item
  Das gleichgewichtete Portfolio als Startlösung erzeugen wir als Liste
  durch \texttt{num\_assets*{[}1./num\_assets,{]}}.
\end{enumerate}

\begin{Shaded}
\begin{Highlighting}[]
\CommentTok{\# function that implements the Sharpe portfolio optimization}
\KeywordTok{def}\NormalTok{ max\_sharpe\_ratio(mean\_returns, cov, rf, freq):}
\NormalTok{    num\_assets }\OperatorTok{=} \BuiltInTok{len}\NormalTok{(mean\_returns)}
\NormalTok{    args }\OperatorTok{=}\NormalTok{ (mean\_returns, cov, rf, freq)}
\NormalTok{    constraints }\OperatorTok{=}\NormalTok{ (\{}\StringTok{\textquotesingle{}type\textquotesingle{}}\NormalTok{: }\StringTok{\textquotesingle{}eq\textquotesingle{}}\NormalTok{, }\StringTok{\textquotesingle{}fun\textquotesingle{}}\NormalTok{: }\KeywordTok{lambda}\NormalTok{ x: np.}\BuiltInTok{sum}\NormalTok{(x) }\OperatorTok{{-}} \DecValTok{1}\NormalTok{\})}
\NormalTok{    bound }\OperatorTok{=}\NormalTok{ (}\FloatTok{0.0}\NormalTok{,}\FloatTok{1.0}\NormalTok{)}
\NormalTok{    bounds }\OperatorTok{=} \BuiltInTok{tuple}\NormalTok{(bound }\ControlFlowTok{for}\NormalTok{ asset }\KeywordTok{in} \BuiltInTok{range}\NormalTok{(num\_assets))}
\NormalTok{    result }\OperatorTok{=}\NormalTok{ minimize(calc\_neg\_sharpe, num\_assets}\OperatorTok{*}\NormalTok{[}\FloatTok{1.}\OperatorTok{/}\NormalTok{num\_assets,], args}\OperatorTok{=}\NormalTok{args,}
\NormalTok{                        method}\OperatorTok{=}\StringTok{\textquotesingle{}SLSQP\textquotesingle{}}\NormalTok{, bounds}\OperatorTok{=}\NormalTok{bounds, constraints}\OperatorTok{=}\NormalTok{constraints,tol}\OperatorTok{=}\FloatTok{1e{-}10}\NormalTok{)}
    \ControlFlowTok{return}\NormalTok{ result}
\end{Highlighting}
\end{Shaded}

Für einen jährlichen risikolosen Zins von 3\% berechnen für nun für
unser Beispiel-Anlageuniversum die Gewichte des Tangentialportfolios.

\begin{Shaded}
\begin{Highlighting}[]
\CommentTok{\# set annual risk{-}free rate to 3\% and calculate maximum Sharpe portfolio}
\NormalTok{rf }\OperatorTok{=} \FloatTok{3.0}
\NormalTok{optimal\_port\_sharpe }\OperatorTok{=}\NormalTok{ max\_sharpe\_ratio(means, Sigma, rf, }\DecValTok{1}\NormalTok{)}
\end{Highlighting}
\end{Shaded}

\begin{Shaded}
\begin{Highlighting}[]
\NormalTok{pd.DataFrame([}\BuiltInTok{round}\NormalTok{(x,}\DecValTok{4}\NormalTok{) }\ControlFlowTok{for}\NormalTok{ x }\KeywordTok{in}\NormalTok{ optimal\_port\_sharpe[}\StringTok{\textquotesingle{}x\textquotesingle{}}\NormalTok{]],index}\OperatorTok{=}\NormalTok{frame.columns).T}
\end{Highlighting}
\end{Shaded}

\begin{longtable}[]{@{}lllllllllll@{}}
\toprule\noalign{}
& ABT & BA & COST & CSCO & IBM & INTC & MRK & MSFT & T & XOM \\
\midrule\noalign{}
\endhead
\bottomrule\noalign{}
\endlastfoot
0 & 0.0 & 0.0 & 0.0979 & 0.0 & 0.1474 & 0.0 & 0.0 & 0.0 & 0.0 &
0.7547 \\
\end{longtable}

Augenscheinlich besteht des Tangentialportfolio zu 75\% aus XOM. Ein
breiter diversifiziertes Portfolio würde Bestandshöchstgrenzen (z.B.
35\%) erfordern.

\subsection{Literatur}\label{literatur}

Benninga, S., (2014). Financial Modeling, 4. Auflage, MIT Press, London.

Franzen, D., Schäfer, K. (2018). Assetmanagement, 1. Auflage,
Schäffer-Poeschel, Stuttgart.

Poddig, T., Brinkmann, U., Seiler, K. (2009). Portfolio Management:
Konzepte und Strategien, 2. Auflage, Uhlenbruch Verlag, Wiesbaden.

\bookmarksetup{startatroot}

\chapter{Relative Optimierung}\label{relative-optimierung}

Die folgenden Ausführungen zur relativen Portfoliooptimierung sind
angelehnt an Poddig et al.~(2009), S. 197-216. Der grundlegende Ansatz
der relativen Optimierung im Rahmen der Portfoliokonstruktion geht dabei
zurück auf Grinold und Kahn (2000).

\subsection{Motivation}\label{motivation}

Ein möglicher Ausgangspunkt für einen relativen Optimierungsansatz
ergibt sich durch die häufige Trennung der Ergebnisverantwortlichkeit
beim aktiven Management. So ist bei Fremdverwaltung des Portfolios
streng genommen eine Separierung der Rendite- und Risikoverantwortung
notwendig. Der Investor überträgt hier die Vermögensverwaltung auf einen
(Fremd-) Manager unter Vorgabe einer für den Portfoliomanager bindenden
Benchmark. Der Investor akzeptiert also die Rendite und das Risiko der
Benchmark. Er ist bereit, dieses zu tragen. Der Manager trägt demzufolge
nicht die Verantwortung für die Benchmarkrendite und das
Benchmarkrisiko, sondern nur für die \emph{zusätzliche Rendite} und das
\emph{zusätzliche Risiko} des aktiven Portfolios. Konsequenz für den
(Portfolio-) Manager ist, dass die absolute Optimierung eines Portfolios
für ihn gar nicht sinnvoll ist. Relevant ist hier nur die zusätzliche
Rendite (aktive Rendite) gegenüber dem dabei entstehenden zusätzlichen
Risiko (aktives Risiko). Das optimale ``relative'' Portfolio ist
gekennzeichnet durch den bestmöglichen Trade-off zwischen aktiver
Rendite und aktivem Risiko relativ zur Benchmark. Gesucht ist also das
optimale Portfolio \emph{relativ} zur Benchmark. Eine solche Trennung
der Ergebnisverantwortlichkeit ergibt sich beispielsweise auch bei der
Verwaltung von Publikumsfonds, deren stratgische Ausrichtung für den
Portfoliomanager zumeist gegeben ist und durch eine Benchmark fixiert
wird. Eine aktive Portfoliostrategie kann dann mit der relativen
Optimierung umgesetzt werden.

Ausgehend von einigen Kritikpunkten zur Vorgehensweise bei der absoluten
Optimierung kann der Einsatz der relativen Optimierung ebenfalls
begründet werden. Zentrale Voraussetzungen der absoluten Optimierung
sind die ``genaue'' Prognostizierbarkeit der benötigten Inputparameter
(im Sinne der ersten beiden Momente der Renditeverteilung) und die
Tatsache, dass der Investor seine Nutzenfunktion kennt und diese auch
genau beschreiben kann. Unter diesen Voraussetzungen liefert die
absolute Optimierung konsequent das optimale Portfolio.

Ein Hauptkritikpunkt setzt an den Voraussetzungen der
Erwartungsnutzentheorie an. Im Wesentlichen geht es um die
Identifikation der Nutzenfunktion, welche sich in der Praxis als
schwierig erweist.

Zudem ist die Prognose der Inputparameter aus der Sache heraus mit
Unsicherheit behaftet. Finanzmarktprognosen sind nicht in der eigentlich
benötigten Präzision bereitstellbar, selbst wenn es ``nur'' um die
Momente der Verteilung von Zufallsvariablen geht. Aber gerade dies führt
bei der absoluten Portfoliooptimierung zu Problemen. Die Sensitivität
der Portfoliooptimierung in Bezug auf relativ kleine Veränderungen der
geschätzten Inputparameter, gerade bei den prognostizierten
Renditeerwartungswerten, ist im Allgemeinen vergleichsweise hoch.
Aufgrund dessen ergibt sich schon bei kleinen Veränderungen der
Renditeschätzungen ein relativ großer Umschichtungsbedarf bei der
Portfoliobildung und somit auch eine gewisse ``Instabilität'' der
Portfoliostruktur.

Neben der Instabilität ist bei der praktischen Portfoliooptimierung
ferner eine Neigung zu ``extremen'' Positionen beim optimalen Portfolio
und sogar die Bildung ökonomisch unplausibler Portfoliostrukturen zu
beobachten. Dies ist zwar auch ein Folge der Prognoseproblematik, nicht
der Optimierung an sich, aber die absolute Optimierung liefert aufgrund
dieser Problematik oftmals Portfolios, welche der Investor wegen
extremer Positionen in den einzelnen Assets nicht halten möchte. Eine
ausführlichere Diskussion um die Schächen des Mean-Variance-Ansatzes mit
entsprechenden Beispielen findet sich in Drobetz (2003, S. 206 ff.) und
in Kapitel A3.1.

Damit sind weder die Effizienzkurve (wegen der Prognoseproblematik) noch
die Nutzenfunktion (wegen der Identifikationsproblematik) hinreichend
präzise bestimmbar. Nur mit beiden zusammen ist jedoch streng genommen
das optimale Portfolio auffindbar.

Zur Lösung dieses Problems sind in Wissenschaft und Praxis verschiedene
Ansätze entwickelt worden:

\begin{itemize}
\tightlist
\item
  als Konsequenz aus der Unmöglichkeit präziser Finanzmarktprognosen,
  Verzicht auf das aktive Portfoliomanagement und stattdessen die
  Durchführung eines \textbf{passiven Managements mit Index Tracking}
  (siehe Kapitel B1.);
\item
  Ansätze der \textbf{robusten Portfoliooptimierung} (siehe Kapitel
  A3.): Einsatz verschiedenster Nebenbedingungen, robuste Schätzung der
  Inputparameter, Black-Litterman Portfoliooptimierungsansatz, Portfolio
  Resampling, risikogesteuerte Ansätze;
\end{itemize}

Ein weiterer Ansatz ist die \textbf{relative Optimierung}. Hier werden
die Inputparameter der Assets des Anlageuniversums in Relation zu einer
Benchmark betrachtet und die Optimierung der Portfoliostruktur wird mit
diesen relativen Größen durchgeführt. So bleibt eine Bindung des
optimalen Portfolios an die Benchmark erhalten, womit inakzeptable
Lösungen tendenziell vermieden werden.

\subsection{Grundlagen und Grundbegriffe der relativen
Optimierung}\label{grundlagen-und-grundbegriffe-der-relativen-optimierung}

Bei der relativen Optimierung ist es üblich, die Rendite \(r_i\) eines
Assets \(i\) in Form einer \emph{Überschussrendite} \(r^´_i\) über den
risikofreien Zinssatz \(r_f\) hinaus aufzufassen: \(r^´_i=r_i-r_f\).
Wird der risikolose Zins als zeitkonstant angenommen, ist die
Varianz-Kovarianzmatrix der Überschussrenditen gleich der
Varianz-Kovarianzmatrix der totalen (absoluten) Renditen, d.h.
\(\Sigma^´=\Sigma\).

\subsubsection{Linearer
Renditegenerierungsprozess}\label{linearer-renditegenerierungsprozess}

Die relative Optimierung unterstellt einen univariaten
Renditegenerierungsprozess für jedes Asset \(i\):

\[ (1) \quad r^´_i=\alpha_i + \beta_i*r^´_B+\epsilon_i, \]

mit:

\begin{itemize}
\tightlist
\item
  \(r^´_i\): Überschussrendite des i-ten Assets
\item
  \(r^´_B\): Überschussrendite des Benchmarkportfolios B
\item
  \(\alpha_i\): autonome Eigenrendite des i-ten Assets, Konstante
\item
  \(\beta_i\): Sensitivität gegenüber dem Benchmarkportfolio
\item
  \(\epsilon_i\): unsystematische, zufällige, nicht erklärbare Restgröße
  (auch \emph{Residualrendite} oder \emph{residuale Rendite} genannt).
\end{itemize}

Dabei wird ferner für die Residualrenditen angenommen:

\$ (2) \quad E(\epsilon\_i)=0 \$ \(\qquad\) für alle \(i\), mit \(E()\):
Erwartungswertoperator;

\$ (3) \quad Var(\epsilon\emph{i)=\sigma\^{}2}\{\epsilon\_i\}\$
\(\quad\) endlich und konstant für alle \(i\)

\$ (4) \quad Cov(\epsilon\_i,
r\^{}´\_B)=E((\epsilon\_i-0)(r\^{}´\_B-\mu\^{}´\_B))=0 \$ \(\quad\) mit
\(\mu^´_B=E(r^´_B)\): Erwartungswert der Benchmarkrendite

\$ (5) \quad Cov(\epsilon\_i,
\epsilon\_j)=E((\epsilon\_i-0)(\epsilon\_j-0))=0 \$ \(\qquad\) für alle
\(i, j, i \neq j.\)

Damit lassen sich leicht die Erwartungswerte der Assetrenditen, deren
Varianzen und die Kovarianzen zwischen den Assetrenditen bestimmen.

Für die erwartete (Überschuss-) Rendite, Varianz und Kovarianz des
Assets \(i\) gilt damit unter den gesetzten Annahmen:

\$ (6) \quad E(r\^{}´\_i)=\mu\^{}´\_i=\alpha\_i+\beta\_i*\mu\^{}´\_B \$

\$ (7)
\quad \sigma\textsuperscript{2\_i=\beta}2\_i*\sigma\textsuperscript{2\_B+\sigma}2\_\{\epsilon\_i\}
\$

\$ (8) \quad \sigma\_\{ij\}=\beta\_i*\beta\_j*\sigma\^{}2\_B \$

Die Varianz, also das mit Asset \(i\) verbundene Risiko, kann nach (7)
in zwei Komponenten zerlegt werden. Ein Teil des Risikos
\((\beta^2_i\sigma^2_B)\) wird durch die Benchmark erklärt. Der zweite
Teil \((\sigma^2_{\epsilon_i})\) ist das sogenannte Restrisiko, auch
\emph{residuales Risiko} (``residual risk'') genannt. Aus den bisherigen
Annahmen und Definitionen ergeben sich nun die für die relative
Portfoliooptimierung relevanten Formeln.

\subsubsection{Portfolio-Alpha und Beta}\label{portfolio-alpha-und-beta}

Für die Portfolioüberschussrendite gilt:

\$ (1)
\quad r\^{}´\_p=\alpha\_p+\beta\_p*r\^{}´\_B+\epsilon\_P=\alpha\^{}*\_P+\beta\_P*r\^{}´\_B
\$ \(\qquad\) mit \(\alpha^*_P=\alpha_P+\epsilon_P\)

Das Portfolio-Alpha und Beta wird durch die Gewichtung der Assets im
Portfolio und deren Alpha- und Beta-Werte determiniert:

\$ (2) \quad \alpha\_P=w\^{}T\_P\alpha \$

mit \(\qquad\) \(\alpha:\) \(\quad\) Vektor der autonomen Eigenrenditen
der Assets

\$ (3) \quad \beta\_P=w\^{}T\_P\beta \$

mit \(\qquad\) \(\beta:\) \(\quad\) Vektor der Sensitivitäten der
enthaltenen Assets gegenüber dem Benchmarkportfolio \(B\).

Das Risiko (die Varianz) eines Portfolios kann ebenso wie nach Gleichung
(7) oben in zwei Komponenten zerlegt werden. Ein Teil des Risikos
\((\beta^2_P\sigma^2_B)\) wird durch die Benchmark erklärt. Der zweite
Teil \((\sigma^2_{\epsilon_P})\) stellt das Restrisiko (residuales
Risiko) dar:

\$ (4)
\quad \sigma\textsuperscript{2\_P=\beta}2\_P*\sigma\textsuperscript{2\_B+\sigma}2\_\{\epsilon\_P\}
\$

bzw. ergibt sich das residuale Risiko des Portfolios als:

\$ (5)
\quad \sigma\^{}2\_\{\epsilon\_P\}=\sigma\textsuperscript{2\_P-\beta}2\_P*\sigma\^{}2\_B.
\$

Das Risiko von Portfolio \(P\) und Benchmark \(B\) berechnen sich wie
bekannt:

\$ (6) \quad \sigma\textsuperscript{2\_P=w}T\_P\Sigma w\_P \$

\$ (7) \quad \sigma\textsuperscript{2\_B=w}T\_B\Sigma w\_B \$

Eingesetzt in (5) folgt:

\$ (8)
\quad \sigma\^{}2\_\{\epsilon\_P\}=w\textsuperscript{T\_P\Sigma w\_P-(w}T\_P\beta)\^{}2*w\^{}T\_B\Sigma w\_B.
\$

\subsubsection{Aktive Position und aktives
Risiko}\label{aktive-position-und-aktives-risiko}

Die relative Optimierung zeichnet sich im Gegensatz zur absoluten
Optimierung durch den Bezug zur Benchmark aus, ohne jedoch zum passiven
Portfoliomanagement zu zählen. Die Abweichung der Assetgewichte im
Portfolio zu den jeweiligen Assetgewichten in der Benchmark wird auch
als \emph{aktive Position} bezeichnet. Die aktive Position ist folglich
definiert als Differenzgewichte zwischen gehaltenem Portfolio \(P\) und
Benchmark \(B\):

\$ (1) \quad w\_A=w\_P-w\_B. \$

Das auf die aktive Position zurückzuführende Risiko wird auch
\emph{aktives Risiko} (aktive Varianz) bezeichnet und ergibt sich als:

\$ (2) \quad \sigma\textsuperscript{2\_\{AP\}=w}T\_A\Sigma w\_A. \$

\subsubsection{Aktives Beta}\label{aktives-beta}

Analog zur Bestimmung des aktiven Risikos ergibt sich auch ein
\emph{aktives Beta}. Das aktive Beta ist die Differenz zwischen
Portfolio- und Benchmark-Beta (welches definitionsgemäß eins beträgt):

\$ (1) \quad \beta\_\{AP\}=\beta\_P-\beta\_B=\beta\_P-1 \$

mit \(\qquad\) \(\beta_{AP}:\) \(\quad\) aktives Beta.

Für die aktive Varianz (das aktive Risiko) gilt damit (siehe zur
Herleitung der Formel Poddig et al., 2009, S. 240-241):

\$ (2)
\quad \sigma\textsuperscript{2\_\{AP\}=\beta}2\_\{AP\}\sigma\^{}2\_B +
\sigma\^{}2\_\{\epsilon\_P\} \$

\subsubsection{Selektions- und
Timingrisiko}\label{selektions--und-timingrisiko}

Für die Interpretation der Gleichung (2) sind zwei verschiedene
Risikoarten zu definieren. Man unterscheidet Risiko aufgrund von
Selektion und Risiko aufgrund von Timing. Selektionsfähigkeit beschreibt
das Können eines Portfoliomanagers, überdurchschnittlich
renditeträchtige Wertpapiere zu identifizieren. Dies bedeutet eine gute
Portfolioperformance aufgrund der Auswahl der einzelnen Assets. Einem
Portfoliomanager wird Selektionsfähigkeit beispielsweise dann
zugeschrieben, wenn das von ihm zusammengestellte Portfolio ein
signifikant positives Alpha aufweist. Timingfähigkeit beschreibt
hingegen die Fähigkeit des Portfoliomanagers in Zeiträumen, in denen die
Benchmark eine positive (Überschuss-) Rendite aufweist, mit dem
gemanagten Portfolio eine zur Benchmarkrendite stärker steigende
(Überschuss-) Rendite durch aktive Gestaltung der Sensitivität \(\beta\)
zu erzielen. Umgekehrt sollte das aktiv gemanagte Portfolio bei
Timingfähigkeit in fallenden Marktphasen eine zur Benchmark weniger
stark fallende (Überschuss-) Rendite erzielen.

Das aktive Risiko (vgl. Formel (2)) ist infolgedessen in das Risiko
aufgrund von Timing \(\beta^2_{AP}\sigma^2_B\) und das durch
\(\sigma^2_{\epsilon_P}\) repräsentierte Risiko aufgrund von Selektion
zu trennen.

Sofern bei der relativen Optimierung die Timingkomponente bewusst
ausgeschlossen wird, also:

\$ (1) \quad \beta\_\{P\}=\beta\emph{B = 1
\Leftrightarrow \beta}\{AP\}=0 \$

explizit gefordert wird, so vereinfacht sich (2) zu:

\$ (2\^{}´) \quad \sigma\^{}2\_\{AP\}= \sigma\^{}2\_\{\epsilon\_P\} \$

Dies impliziert die Gleichsetzung des aktiven Risikos mit dem residualen
Risiko, welches das ``Selektionsrisiko'' widerspiegelt. Die Forderung
nach \((2^´)\) wird im Folgenden stets gesetzt (vgl. Grinold und Kahn,
2000, S. 102, und Poddig et al., 2009, S. 213f., zur Begründung des
Ausschlusses der Timingkomponente bei der relativen Optimierung).

\subsection{Zielfunktion der relativen
Optimierung}\label{zielfunktion-der-relativen-optimierung}

Aus den bisherigen Überlegungen und unter Vernachlässigung der
Timingkomponente des aktiven Risikos lässt sich folgende Zielfunktion
der relativen Optimierung aus der Zielfunktion der absoluten Optimierung
herleiten (siehe Poddig et al., 2009, S. 209-216):

\[
\begin{split}
 \\(1) \quad ZF(w) = \alpha_P-\lambda\sigma^2_{\epsilon_P} \rightarrow \max_{w}!, \\
\end{split}
\]

wobei \(\lambda\) den anlegerindividuellen Risikoaversionskoeffizienten
angibt. In Matrizenschreibweise erhält man:

\[ (2) \quad ZF(w) = w_p ^T\alpha-\lambda(w_P^T\Sigma w_P-(w_P^T\beta)^2*w_B^T\Sigma w_B) \]

bzw. alternativ (da \(\beta_P=1\) bzw. \(\beta_{AP}=0\) im Folgenden
gesetzt werden)

\$ (3) \quad ZF(w) = w\_p
\textsuperscript{T\alpha-\lambda(w\_A}T\Sigma w\_A). \$

Dies entspricht der Maximierung der Differenz von Portfolio-Alpha und
dem mit dem Risikoaversionsparameter gewichteten Selektionsrisiko (oder
residualen Risiko), welches unter der Annahme einer nicht existenten
Timingkomponente zugleich dem aktiven Risiko entspricht.

Die zentralen Nebenbedingungen (Budgetrestriktion und Verbot von
Leerverkäufen) werden um den Ausschluss der Timingkomponente ergänzt:

\[
\begin{split}
 & \text{(a)}\ w^{T}\iota = 1 \quad bzw. \quad \ w_A^{T}\iota = 0 \qquad \text{(Budgetrestriktion)}, \\
 & \text{(b)}\ w\geqq 0 \qquad \text{(Leerverkaufsverbot)}, \\
 & \text{(c)}\ \beta_P=1 \quad \Leftrightarrow \quad \beta_{AP}=0 \quad \text{(kein Timing)}.
\end{split}
\]

Damit ist das Optimierungsproblem für die relative Optimierung
formuliert. Es wird im nächsten Kapitel im Rahmen einer Fallstudie näher
veranschaulicht.

\subsection{Beginn der Fallstudie}\label{beginn-der-fallstudie-1}

Wir starten mit dem Import der benötigten Pakete.

\begin{Shaded}
\begin{Highlighting}[]
\ImportTok{import}\NormalTok{ pandas }\ImportTok{as}\NormalTok{ pd}
\ImportTok{import}\NormalTok{ numpy }\ImportTok{as}\NormalTok{ np}
\ImportTok{from}\NormalTok{ scipy.optimize }\ImportTok{import}\NormalTok{ minimize}
\end{Highlighting}
\end{Shaded}

\subsubsection{Laden und Beschreiben der
Datenbasis}\label{laden-und-beschreiben-der-datenbasis-1}

Das beispielhafte Anlageuniversum der Fallstudie umfasst acht
Unternehmen aus den folgenden Branchen: Technologie, Gesundheit,
Nahrungsmittel, Pharma, Energie sowie Luft- und Raumfahrt. Die
Unternehmen sind: Costco Wholesale (COST), Cisco Systems (CSCO), IBM
(IBM), Intel (INTC), Merk (MRK), Microsoft (MSFT), AT\&T (T), und Exxon
Mobil Corporation (XOM).

Die Datengrundlage stellen Monatsanfangskurse (``Adjusted Close'') über
einen 5-Jahres-Zeitraum vom 1.12.2004 bis zum 1.12.2009 dar. Die
Stichprobe enthält somit 61 Zeitreihenbeobachtungen.

\begin{Shaded}
\begin{Highlighting}[]
\CommentTok{\# hier den Pfad zur Datei eingeben}
\CommentTok{\# cd "..."}
\end{Highlighting}
\end{Shaded}

\begin{Shaded}
\begin{Highlighting}[]
\NormalTok{frame }\OperatorTok{=}\NormalTok{ pd.read\_excel(}\StringTok{\textquotesingle{}Kapitel A1.xlsx\textquotesingle{}}\NormalTok{, }\StringTok{\textquotesingle{}Tabelle1\textquotesingle{}}\NormalTok{, index\_col}\OperatorTok{=}\DecValTok{0}\NormalTok{, parse\_dates}\OperatorTok{=}\VariableTok{True}\NormalTok{)}
\end{Highlighting}
\end{Shaded}

\begin{Shaded}
\begin{Highlighting}[]
\NormalTok{frame.head()}
\end{Highlighting}
\end{Shaded}

\begin{longtable}[]{@{}lllllllllll@{}}
\toprule\noalign{}
& ABT & BA & COST & CSCO & IBM & INTC & MRK & MSFT & T & XOM \\
\midrule\noalign{}
\endhead
\bottomrule\noalign{}
\endlastfoot
2004-12-01 & 46.65 & 51.77 & 48.41 & 19.32 & 98.58 & 23.39 & 32.14 &
26.72 & 25.77 & 51.26 \\
2005-01-03 & 45.02 & 50.60 & 47.27 & 18.04 & 93.42 & 22.45 & 28.05 &
26.28 & 23.76 & 51.60 \\
2005-02-01 & 45.99 & 54.97 & 46.59 & 17.42 & 92.58 & 23.99 & 31.70 &
25.16 & 24.06 & 63.31 \\
2005-03-01 & 46.62 & 58.46 & 44.18 & 17.89 & 91.38 & 23.23 & 32.37 &
24.17 & 23.69 & 59.60 \\
2005-04-01 & 49.16 & 59.52 & 40.63 & 17.27 & 76.38 & 23.52 & 33.90 &
25.30 & 23.80 & 57.03 \\
\end{longtable}

\subsection{Fall A: Das aktive Portfolio und die Benchmark entspringen
demselben
Anlageuniversum}\label{fall-a-das-aktive-portfolio-und-die-benchmark-entspringen-demselben-anlageuniversum}

Wir berechnen zunächst diskrete Monatsrenditen über
\texttt{frame.pct\_change()}. Annahmegemäß beträgt der zeitkonstante
annualisierte risikolose Zins 4\%, was einem Monatszinssatz von
0,327374\% entspricht. Die absoluten Renditen werden dann in
Überschussrenditen überführt (DataFrame \texttt{ex\_returns}). Auf Basis
der Überschussrenditen ermitteln wir den Vektor der zukünftigen
erwarteten Assetrenditen \(\mu\) und die zukünftige
Varianz-Kovarianzmatrix \(\Sigma\) nach der Methode der
\textbf{historisch basierten Schätzung}.

\begin{Shaded}
\begin{Highlighting}[]
\CommentTok{\# calculation based on discrete returns}
\NormalTok{returns }\OperatorTok{=}\NormalTok{ frame.pct\_change().dropna()}
\NormalTok{returns\_new }\OperatorTok{=}\NormalTok{ returns.drop([}\StringTok{\textquotesingle{}ABT\textquotesingle{}}\NormalTok{, }\StringTok{\textquotesingle{}BA\textquotesingle{}}\NormalTok{], }\DecValTok{1}\NormalTok{) }\CommentTok{\# we drop first two tickers}

\NormalTok{rf }\OperatorTok{=}\NormalTok{ (}\DecValTok{1}\OperatorTok{+}\FloatTok{0.04}\NormalTok{)}\OperatorTok{**}\NormalTok{(}\DecValTok{1}\OperatorTok{/}\DecValTok{12}\NormalTok{)}\OperatorTok{{-}}\DecValTok{1} \CommentTok{\# risk{-}free rate assumed 4\% p.a.}
\NormalTok{ex\_returns }\OperatorTok{=}\NormalTok{ returns\_new }\OperatorTok{{-}}\NormalTok{ rf}
\NormalTok{means }\OperatorTok{=}\NormalTok{ ex\_returns.mean().values}\OperatorTok{*}\DecValTok{100} \CommentTok{\# non{-}annualised!}
\NormalTok{Sigma }\OperatorTok{=}\NormalTok{ ex\_returns.cov().values }\CommentTok{\# non{-}annualised!}
\end{Highlighting}
\end{Shaded}

Das Benchmarkportfolio wird hier willkürlich als gleich gewichtetes
Portfolio aller Assets gewählt. Durch Aufruf von \texttt{np.tile(A,x)}
wird ein Array generiert, welches x-Mal den Wert A enthält. Die Anzahl
der Assets in unserem Beispieluniversum erhalten wir über
\texttt{means.shape{[}0{]}}.

Im Folgenden werden wir für die Matrizenmultiplikation häufig Arrays der
Dimension \(N\) über \texttt{np.matrix} in \(1xN\) Zeilenvektoren
transformieren.

Die Zeitreihe \texttt{benchmark} der Überschussrenditen \(r^´_B\) der
Benchmark erhalten wir indem zunächst die Assetrenditen über
\texttt{ex\_returns.multiply(Weight\_1N)} zeilenweise mit den
Benchmarkgewichten \texttt{Weight\_1N} multipliziert werden und dann die
gewichtete Summe der Zeilen gebildet wird.

\begin{Shaded}
\begin{Highlighting}[]
\CommentTok{\# using an equally{-}weighted benchmark}
\NormalTok{Weight\_1N }\OperatorTok{=}\NormalTok{ np.tile(}\FloatTok{1.0}\OperatorTok{/}\NormalTok{means.shape[}\DecValTok{0}\NormalTok{], means.shape[}\DecValTok{0}\NormalTok{])}
\NormalTok{bench\_w }\OperatorTok{=}\NormalTok{ np.matrix(Weight\_1N) }\CommentTok{\# benchmark portfolio weights as row vector}
\NormalTok{benchmark }\OperatorTok{=}\NormalTok{ ex\_returns.multiply(Weight\_1N).}\BuiltInTok{sum}\NormalTok{(axis}\OperatorTok{=}\DecValTok{1}\NormalTok{)}
\end{Highlighting}
\end{Shaded}

\subsubsection{Schätzung weiterer Inputparameter: Alpha und
Beta}\label{schuxe4tzung-weiterer-inputparameter-alpha-und-beta}

Als zentrale Inputparameter für eine relative Optimierung gehen die
Alpha- und Beta-Parameter in die Berechnungen ein. Diese erklären die
Überschussrenditen der einzelnen Assets als Summe der autonomen
Eigenrendite des jeweiligen Assets und aus einem von der Benchmark
abhängigen Renditeanteil. Wird ein linearer Renditegenerierungsprozess
unterstellt, so ist eine einfache historisch basierte Schätzung in
diesem Fall über eine Regression der Renditen der Assets auf die
Benchmark möglich.

Grundsätzlich gelten für Alpha- und Beta-Parameter dieselben
Überlegungen wie hinsichtlich der erwarteten Renditen und zukünftigen
Risiken bei der absoluten Optimierung. Die Alpha- und Beta-Werte sind ex
ante Größen, womit Prognosen (Schätzungen) dieser Größen unvermeidbar
sind. Ebenso ist hier zu bedenken, dass die Güte der Alpha- und
Beta-Prognosen die spätere Performance des aktiven Portfolios bestimmt.
Auch hier ist die Prognose der wertgenerierende Prozess.

Zum Zwecke der Fallstudie wird nachfolgend der pragmatische Ansatz der
einfachen historisch basierten Schätzung mittels univariater linearer
Regression umgesetzt. Poddig et al.~(2009, S. 220-223) geben einige
Hinweise auf weitere Verfahren speziell zur Prognose von Alpha- und
Beta-Parametern.

Zur Umsetzung der linearen Kleinste-Quadrate (Ordinary Least Squares -
OLS) Regression laden wir zunächst die Module \texttt{linear\_model} und
\texttt{tools} aus dem Paket \texttt{statsmodels}.

\begin{Shaded}
\begin{Highlighting}[]
\ImportTok{import}\NormalTok{ statsmodels.regression.linear\_model }\ImportTok{as}\NormalTok{ sm}
\ImportTok{import}\NormalTok{ statsmodels.tools.tools }\ImportTok{as}\NormalTok{ sm2}
\end{Highlighting}
\end{Shaded}

Die Matrix \texttt{x2} unten enthält die exogenen Variablen der
Regression. In unserem Fall stellt die erste Spalte einen Vektor mit
Einsen dar, die die Konstante (Achsenabschnitt=Alpha) der Regression
modellieren. Diese Spalte wird über \texttt{add\_constant} erzeugt. Die
zweite Spalte ist der Vektor der Benchmarkrenditen. Die y-Variable der
Regression ist die entsprechende Spalte des DataFrames der
Überschussrenditen der Assets.

Die Schätzung eines OLS Models erfolgt über
\texttt{sm.OLS(y-Variable,\ (Konstante,x-Variablen)).fit()}. Die
geschätzen Regressionskoeffizienten sind im Array \texttt{params}
enthalten. Die Regressionen werden in Form einer for-Schleife über die
Spalten von \texttt{ex\_returns} iteriert, wobei wiederum die
\texttt{enumerate} Methode verwendet wird, um den Interationsindex für
die Indixierung der Ergebnis-Arrays \texttt{alpha} und \texttt{beta} zu
verwenden.

\begin{Shaded}
\begin{Highlighting}[]
\CommentTok{\# calculating vectors of alphas and betas}
\NormalTok{alpha }\OperatorTok{=}\NormalTok{ np.zeros(ex\_returns.columns.shape)}
\NormalTok{beta }\OperatorTok{=}\NormalTok{ np.zeros(ex\_returns.columns.shape)}

\NormalTok{x1 }\OperatorTok{=}\NormalTok{ benchmark}
\NormalTok{x2 }\OperatorTok{=}\NormalTok{ sm2.add\_constant(x1)}

\ControlFlowTok{for}\NormalTok{ idx, ticker }\KeywordTok{in} \BuiltInTok{enumerate}\NormalTok{(ex\_returns.columns):}
\NormalTok{    reg }\OperatorTok{=}\NormalTok{ sm.OLS(ex\_returns[ticker], x2).fit()}
\NormalTok{    parameter }\OperatorTok{=}\NormalTok{ np.asarray(reg.params)}
\NormalTok{    alpha[idx] }\OperatorTok{=}\NormalTok{ parameter[}\DecValTok{0}\NormalTok{]}
\NormalTok{    beta[idx] }\OperatorTok{=}\NormalTok{ parameter[}\DecValTok{1}\NormalTok{]}
\NormalTok{df }\OperatorTok{=}\NormalTok{pd.DataFrame(\{}\StringTok{\textquotesingle{}Alpha\textquotesingle{}}\NormalTok{: alpha, }\StringTok{\textquotesingle{}Beta\textquotesingle{}}\NormalTok{: beta\}, index}\OperatorTok{=}\NormalTok{ex\_returns.columns)}
\end{Highlighting}
\end{Shaded}

Aus Gründen der besseren Übersicht geben wir die Schätzergebnisse in
Form eines DataFrames wieder.

\begin{Shaded}
\begin{Highlighting}[]
\NormalTok{df}
\end{Highlighting}
\end{Shaded}

\begin{longtable}[]{@{}lll@{}}
\toprule\noalign{}
& Alpha & Beta \\
\midrule\noalign{}
\endhead
\bottomrule\noalign{}
\endlastfoot
COST & 0.000870 & 0.900610 \\
CSCO & 0.001147 & 1.285882 \\
IBM & 0.001738 & 0.798834 \\
INTC & -0.005285 & 1.366525 \\
MRK & 0.000367 & 1.158739 \\
MSFT & -0.000665 & 1.146965 \\
T & -0.002151 & 0.812048 \\
XOM & 0.003978 & 0.530397 \\
\end{longtable}

Für die nachfolgende Matrizenmultiplikation transformieren wir die
beiden Ergebnis-Arrays in Zeilenvektoren.

\begin{Shaded}
\begin{Highlighting}[]
\NormalTok{alpha }\OperatorTok{=}\NormalTok{ np.matrix(alpha)}
\NormalTok{beta }\OperatorTok{=}\NormalTok{ np.matrix(beta)}
\end{Highlighting}
\end{Shaded}

Nun schreiben wird die Zielfunktion der Optimierung in Form der Funktion
\texttt{relative\_opt1}. \textbf{Wichtig}: Diese Funktion erfordert,
dass die Arrays der Alpha's, Beta's, und der Benchmark-Gewichte
vorher(!) in Zeilenvektoren überführt wurden.

Die Funktion berechnet auf Basis eines Eingabe-Arrays von
Portfolio-Anteilsgewichten \(w_P\) das Portfolio Alpha
\((w_p ^T\alpha)\), Beta \((w_P^T\beta)\), die Varianz der
Portfoliorendite \((w_P^T\Sigma w_P)\) und der Benchmarkrendite
\((w_B^T\Sigma w_B)\), und das residuale Risiko
\((w_P^T\Sigma w_P-(w_P^T\beta)^2*w_B^T\Sigma w_B)\). Als Ausgabe
liefert die Funktion den mit -1 multiplizierten Zielfunktionswert, da es
sich um ein Maximierungsproblem handelt:

\$ \quad ZF(w) = w\_p
\textsuperscript{T\alpha-\lambda(w\_P}T\Sigma w\_P-(w\_P\textsuperscript{T\beta)}2*w\_B\^{}T\Sigma w\_B)
\$

\begin{Shaded}
\begin{Highlighting}[]
\CommentTok{\# target function: active alpha {-} lambda * residual risk}
\CommentTok{\# alpha, beta, bench\_w have to be defined prior as vectors}
\KeywordTok{def}\NormalTok{ relative\_opt1(w, lambda0):}
\NormalTok{    port\_w }\OperatorTok{=}\NormalTok{ np.matrix(w) }\CommentTok{\# w is a row (not column!) vector}
\NormalTok{    port\_alpha }\OperatorTok{=}\NormalTok{ (port\_w}\OperatorTok{*}\NormalTok{alpha.T)[}\DecValTok{0}\NormalTok{,}\DecValTok{0}\NormalTok{]}
\NormalTok{    port\_beta }\OperatorTok{=}\NormalTok{ (port\_w}\OperatorTok{*}\NormalTok{beta.T)[}\DecValTok{0}\NormalTok{,}\DecValTok{0}\NormalTok{]}
\NormalTok{    port\_var }\OperatorTok{=}\NormalTok{ (port\_w }\OperatorTok{*}\NormalTok{ Sigma}\OperatorTok{*}\NormalTok{port\_w.T)[}\DecValTok{0}\NormalTok{,}\DecValTok{0}\NormalTok{]}
\NormalTok{    bench\_var }\OperatorTok{=}\NormalTok{ (bench\_w }\OperatorTok{*}\NormalTok{ Sigma }\OperatorTok{*}\NormalTok{ bench\_w.T)[}\DecValTok{0}\NormalTok{,}\DecValTok{0}\NormalTok{]}
\NormalTok{    resid\_var }\OperatorTok{=}\NormalTok{ port\_var }\OperatorTok{{-}}\NormalTok{ port\_beta}\OperatorTok{**}\DecValTok{2}\OperatorTok{*}\NormalTok{bench\_var}
    \ControlFlowTok{return} \OperatorTok{{-}}\NormalTok{(port\_alpha }\OperatorTok{{-}}\NormalTok{ lambda0 }\OperatorTok{*}\NormalTok{ resid\_var) }\CommentTok{\# {-}1 for minimization problem}
\end{Highlighting}
\end{Shaded}

Die zusätzliche Nebenbedingung (kein Timing), dass das Portfolio-Beta
eins bzw. das aktive Beta null betragen muss, implementieren wir ähnlich
wie die Budgetrestriktion über eine \texttt{lambda}-Funktion, die wir
null setzen und dem Tuple \texttt{cons} der Nebenbedingungs-Dictionaries
hinzufügen.

\begin{Shaded}
\begin{Highlighting}[]
\NormalTok{cons }\OperatorTok{=}\NormalTok{ (\{}\StringTok{\textquotesingle{}type\textquotesingle{}}\NormalTok{: }\StringTok{\textquotesingle{}eq\textquotesingle{}}\NormalTok{, }\StringTok{\textquotesingle{}fun\textquotesingle{}}\NormalTok{: }\KeywordTok{lambda}\NormalTok{ x: np.}\BuiltInTok{sum}\NormalTok{(x) }\OperatorTok{{-}} \DecValTok{1}\NormalTok{\},}
\NormalTok{        \{}\StringTok{\textquotesingle{}type\textquotesingle{}}\NormalTok{: }\StringTok{\textquotesingle{}eq\textquotesingle{}}\NormalTok{, }\StringTok{\textquotesingle{}fun\textquotesingle{}}\NormalTok{: }\KeywordTok{lambda}\NormalTok{ x: (np.matrix(x)}\OperatorTok{*}\NormalTok{beta.T)[}\DecValTok{0}\NormalTok{,}\DecValTok{0}\NormalTok{] }\OperatorTok{{-}} \DecValTok{1}\NormalTok{\})}
\end{Highlighting}
\end{Shaded}

Wir setzen den Risikoaversionskoeffizienten auf den Wert
\(\lambda=-0.3705867\), verwenden ein gleich gewichtetes Portfolio als
Startlösung für die numerische Optimierung und implementieren
Bestandsgenzen (min=5\%, max=40\%). Die Anwendung von \texttt{minimize}
ergibt die folgenden optimierten Portfoliogewichte.

\begin{Shaded}
\begin{Highlighting}[]
\NormalTok{lambda0 }\OperatorTok{=} \OperatorTok{{-}}\FloatTok{0.3705867}
\NormalTok{bound }\OperatorTok{=}\NormalTok{ (}\FloatTok{0.05}\NormalTok{,}\FloatTok{0.40}\NormalTok{)}
\NormalTok{bounds }\OperatorTok{=} \BuiltInTok{tuple}\NormalTok{(bound }\ControlFlowTok{for}\NormalTok{ asset }\KeywordTok{in} \BuiltInTok{range}\NormalTok{(alpha.shape[}\DecValTok{1}\NormalTok{]))}
\NormalTok{res }\OperatorTok{=}\NormalTok{ minimize(relative\_opt1, Weight\_1N, args}\OperatorTok{=}\NormalTok{lambda0,}
\NormalTok{                        method}\OperatorTok{=}\StringTok{\textquotesingle{}SLSQP\textquotesingle{}}\NormalTok{, bounds}\OperatorTok{=}\NormalTok{bounds, constraints}\OperatorTok{=}\NormalTok{cons,tol}\OperatorTok{=}\FloatTok{1e{-}10}\NormalTok{)}
\NormalTok{pd.DataFrame([}\BuiltInTok{round}\NormalTok{(x,}\DecValTok{4}\NormalTok{) }\ControlFlowTok{for}\NormalTok{ x }\KeywordTok{in}\NormalTok{ res.x],index}\OperatorTok{=}\NormalTok{ex\_returns.columns).T }
\end{Highlighting}
\end{Shaded}

\begin{longtable}[]{@{}lllllllll@{}}
\toprule\noalign{}
& COST & CSCO & IBM & INTC & MRK & MSFT & T & XOM \\
\midrule\noalign{}
\endhead
\bottomrule\noalign{}
\endlastfoot
0 & 0.05 & 0.4 & 0.05 & 0.05 & 0.0776 & 0.05 & 0.05 & 0.2724 \\
\end{longtable}

Wir überprüfen kurz ob dieser Gewichtsvektor die No-Timing-Bedingung
auch wirklich einhält:

\begin{Shaded}
\begin{Highlighting}[]
\CommentTok{\# function to calculate the portfolio beta}
\KeywordTok{def}\NormalTok{ port\_beta(w,beta):}
\NormalTok{    port\_w }\OperatorTok{=}\NormalTok{ np.matrix(w) }\CommentTok{\# w is a row (not column!) vector}
\NormalTok{    port\_beta }\OperatorTok{=}\NormalTok{ (port\_w}\OperatorTok{*}\NormalTok{beta.T)[}\DecValTok{0}\NormalTok{,}\DecValTok{0}\NormalTok{]}
    \ControlFlowTok{return}\NormalTok{ port\_beta}


\NormalTok{port\_beta(res.x,beta)}
\end{Highlighting}
\end{Shaded}

\begin{verbatim}
0.9999999999038441
\end{verbatim}

Dies ist offensichtlich der Fall! Das Portfolio liefert eine erwartete
annualisierte Rendite von:

\begin{Shaded}
\begin{Highlighting}[]
\NormalTok{(np.matrix(means)}\OperatorTok{*}\NormalTok{ np.matrix(res.x).T)[}\DecValTok{0}\NormalTok{,}\DecValTok{0}\NormalTok{]}\OperatorTok{*}\DecValTok{12}
\end{Highlighting}
\end{Shaded}

\begin{verbatim}
3.6127468232671593
\end{verbatim}

\subsection{Fall B: Relative Optimierung bei unterschiedlichen
Anlageuniversen}\label{fall-b-relative-optimierung-bei-unterschiedlichen-anlageuniversen}

In der obigen Fallstudie ist das Anlageuniversum für das aktive
Portfolio und die Benchmark \emph{identisch}. Dies ist jedoch ein
Idealfall, der in der Praxis so nicht immer vorliegt. In der Praxis
ergibt sich häufig das Problem unterschiedlicher Anlageuniversen.
Oftmals stehen also nicht alle Anlagen des Benchmarkportfolios für eine
Aufnahme in das aktive Portfolio zur Verfügung. Im Extremfall ist das
Benchmarkportfolio ein synthetischer Index, gegen das ein aus nicht im
Index enthaltenen Einzeltiteln bestehendes aktives Portfolio zu
optimieren ist. Aus diesen Umständen resultiert die Frage, wie eine
relative Optimierung durchgeführt wird, wenn sich die Anlageuniversen
unterscheiden.

Zur Illustration des geschilderten Problems verwenden wir nun den
S\&P500 Index als Benchmark. Die Aufgabenstellung besteht darin, die
relative Optimierung des aktiven Portfolios gegen diese Benchmark
durchzuführen. Dabei wird im Folgenden bewusst so getan, als ob der
S\&P500 ein synthetischer Index sei.

\begin{Shaded}
\begin{Highlighting}[]
\CommentTok{\# benchmark is now the S\&P500 index, therefore we need a new dataframe}
\NormalTok{frame1 }\OperatorTok{=}\NormalTok{ pd.read\_excel(}\StringTok{\textquotesingle{}Kapitel A2\_1.xlsx\textquotesingle{}}\NormalTok{, }\StringTok{\textquotesingle{}Tabelle1\textquotesingle{}}\NormalTok{, index\_col}\OperatorTok{=}\DecValTok{0}\NormalTok{, parse\_dates}\OperatorTok{=}\VariableTok{True}\NormalTok{)}
\end{Highlighting}
\end{Shaded}

\begin{Shaded}
\begin{Highlighting}[]
\NormalTok{frame1.head()}
\end{Highlighting}
\end{Shaded}

\begin{longtable}[]{@{}llllllllll@{}}
\toprule\noalign{}
& COST & CSCO & IBM & INTC & MRK & MSFT & T & XOM & S\&P500 \\
\midrule\noalign{}
\endhead
\bottomrule\noalign{}
\endlastfoot
2004-12-01 & 48.41 & 19.32 & 98.58 & 23.39 & 32.14 & 26.72 & 25.77 &
51.26 & 1211.92 \\
2005-01-03 & 47.27 & 18.04 & 93.42 & 22.45 & 28.05 & 26.28 & 23.76 &
51.60 & 1181.27 \\
2005-02-01 & 46.59 & 17.42 & 92.58 & 23.99 & 31.70 & 25.16 & 24.06 &
63.31 & 1203.60 \\
2005-03-01 & 44.18 & 17.89 & 91.38 & 23.23 & 32.37 & 24.17 & 23.69 &
59.60 & 1180.59 \\
2005-04-01 & 40.63 & 17.27 & 76.38 & 23.52 & 33.90 & 25.30 & 23.80 &
57.03 & 1156.85 \\
\end{longtable}

Die grundsätzliche Vorgehensweise ändert sich im Vergleich zu oben
nicht. Zunächst erfolgt eine Umrechnung der Kursreihen in diskrete
Monatsrenditen, die als Überschussrenditen über den risiklosen Zins
formuliert werden.

\begin{Shaded}
\begin{Highlighting}[]
\CommentTok{\# calculation based on discrete returns}
\NormalTok{returns }\OperatorTok{=}\NormalTok{ frame1.pct\_change().dropna()}

\NormalTok{rf }\OperatorTok{=}\NormalTok{ (}\DecValTok{1}\OperatorTok{+}\FloatTok{0.04}\NormalTok{)}\OperatorTok{**}\NormalTok{(}\DecValTok{1}\OperatorTok{/}\DecValTok{12}\NormalTok{)}\OperatorTok{{-}}\DecValTok{1} \CommentTok{\# risk{-}free rate assumed 4\% p.a.}
\NormalTok{ex\_returns }\OperatorTok{=}\NormalTok{ returns }\OperatorTok{{-}}\NormalTok{ rf}

\CommentTok{\# important: we use only the first 60 returns!}
\NormalTok{means }\OperatorTok{=}\NormalTok{ ex\_returns.iloc[:}\DecValTok{60}\NormalTok{,:].mean().values}\OperatorTok{*}\DecValTok{100}\OperatorTok{*}\DecValTok{12} \CommentTok{\# annualised!}
\NormalTok{Sigma }\OperatorTok{=}\NormalTok{ ex\_returns.iloc[:}\DecValTok{60}\NormalTok{,:].cov().values }\CommentTok{\# non{-}annualised!}
\end{Highlighting}
\end{Shaded}

Diese bilden die Ausgangsbasis für eine historisch basierte Schätzung
der Alpha- und Beta-Parameter, welche mittels univariater linearer
Regression der jeweiligen Assetrenditen auf die Benchmarkrendite
ermittelt werden.

\begin{Shaded}
\begin{Highlighting}[]
\CommentTok{\# calculating vectors of alphas and betas}
\NormalTok{alpha }\OperatorTok{=}\NormalTok{ np.zeros(ex\_returns.columns.shape)}
\NormalTok{beta }\OperatorTok{=}\NormalTok{ np.zeros(ex\_returns.columns.shape)}

\NormalTok{x1 }\OperatorTok{=}\NormalTok{ ex\_returns[}\StringTok{\textquotesingle{}S\&P500\textquotesingle{}}\NormalTok{]}
\NormalTok{x2 }\OperatorTok{=}\NormalTok{ sm2.add\_constant(x1)}

\ControlFlowTok{for}\NormalTok{ idx, ticker }\KeywordTok{in} \BuiltInTok{enumerate}\NormalTok{(ex\_returns.columns):}
\NormalTok{    reg }\OperatorTok{=}\NormalTok{ sm.OLS(ex\_returns[ticker].iloc[:}\DecValTok{60}\NormalTok{], x2.iloc[:}\DecValTok{60}\NormalTok{]).fit()}
\NormalTok{    parameter }\OperatorTok{=}\NormalTok{ np.asarray(reg.params)}
\NormalTok{    alpha[idx] }\OperatorTok{=}\NormalTok{ parameter[}\DecValTok{0}\NormalTok{]}
\NormalTok{    beta[idx] }\OperatorTok{=}\NormalTok{ parameter[}\DecValTok{1}\NormalTok{]}
\NormalTok{df }\OperatorTok{=}\NormalTok{pd.DataFrame(\{}\StringTok{\textquotesingle{}Alpha\textquotesingle{}}\NormalTok{: alpha, }\StringTok{\textquotesingle{}Beta\textquotesingle{}}\NormalTok{: beta\}, index}\OperatorTok{=}\NormalTok{ex\_returns.columns)}
\end{Highlighting}
\end{Shaded}

\begin{Shaded}
\begin{Highlighting}[]
\NormalTok{df}
\end{Highlighting}
\end{Shaded}

\begin{longtable}[]{@{}lll@{}}
\toprule\noalign{}
& Alpha & Beta \\
\midrule\noalign{}
\endhead
\bottomrule\noalign{}
\endlastfoot
COST & 5.317110e-03 & 0.793253 \\
CSCO & 7.774840e-03 & 1.208840 \\
IBM & 6.029992e-03 & 0.798688 \\
INTC & 1.309308e-03 & 1.161773 \\
MRK & 5.708572e-03 & 0.916956 \\
MSFT & 4.811190e-03 & 0.959172 \\
T & 1.748128e-03 & 0.684994 \\
XOM & 6.592676e-03 & 0.465999 \\
S\&P500 & -8.673617e-19 & 1.000000 \\
\end{longtable}

\begin{Shaded}
\begin{Highlighting}[]
\NormalTok{alpha }\OperatorTok{=}\NormalTok{ np.matrix(alpha)}
\NormalTok{beta }\OperatorTok{=}\NormalTok{ np.matrix(beta)}
\end{Highlighting}
\end{Shaded}

Die Formulierung des relativen Optimierungsansatzes und die Bestimmung
des aktiven Portfolios erfolgt wie bisher. Die Modifikation der
Fallstudie liegt in der Wahl unterschiedlicher Anlageuniversen. Ein
genereller Lösungsansatz zum Umgang mit unterschiedlichen
Anlageuniversen besteht darin, bei zwei gegebenen Portfolios \(P\) und
\(B\) mit Anlageuniversen \(X\) und \(Y\) ein gemeinsames
Anlageuniversum \(Z\) als Vereinigungsmenge von \(X\) und \(Y\), d.h.
\(Z=X \cup Y\), zu bilden. Damit sind die Anlagen beider Portfolios
\(P\) und \(B\) nun Elemente des identischen Anlageuniversums \(Z\).
Assets, die in \(P\), aber nicht in \(B\) gehalten werden, haben in
\(B\) ein fixiertes Gewicht von null und entsprechend erhalten Assets,
die in \(B\), aber nicht in \(P\) gehalten werden, in \(P\) ein
fixiertes Gewicht von null.

Das Anlageuniversum in der Fallstudie hat nunmehr neun Assets, nämlich
die betrachteten acht Einzeltitel sowie den S\&P500 als ``virtuelle''
Anlage. Die historisch basierte Berechnung von \(\mu\), \(\Sigma\),
\(\alpha\), und \(\beta\) erfolgte oben bereits für diese neun Anlagen.

Im nächsten Schritt muss der Vektor der Benchmark-Gewichte derart
ausgestaltet sein, dass die ersten acht Gewichte (für die Einzeltitel)
den Wert null und das letzte Gewicht (für den S\&P500) den Wert eins
zugewiesen bekommen.

\begin{Shaded}
\begin{Highlighting}[]
\CommentTok{\# calculation of weight vector for active portfolio and benchmark}
\CommentTok{\# using S\&P500 as a benchmark implies that all benchmark weights (first N{-}1)}
\CommentTok{\# elements are zero and 1 for S\&P500 (last element)}
\NormalTok{bench\_w }\OperatorTok{=}\NormalTok{ np.zeros(means.shape)}
\NormalTok{bench\_w[}\OperatorTok{{-}}\DecValTok{1}\NormalTok{]}\OperatorTok{=}\FloatTok{1.0}
\NormalTok{bench\_w }\OperatorTok{=}\NormalTok{ np.matrix(bench\_w)}
\end{Highlighting}
\end{Shaded}

Zudem muss der Vektor der Startgewichte für das aktive Portfolio jeweils
den Wert \(1/8\) (\texttt{1/(means.shape{[}0{]}-1)}) auf den ersten acht
Positionen enthalten, und den Wert null auf der letzten Position.

\begin{Shaded}
\begin{Highlighting}[]
\CommentTok{\# vector of starting weights is equally{-}weighted for the first N{-}1 }
\CommentTok{\# elements and zero for the last element}
\NormalTok{Weight\_start }\OperatorTok{=}\NormalTok{ np.zeros(means.shape)}
\NormalTok{Weight\_start[}\DecValTok{0}\NormalTok{:}\OperatorTok{{-}}\DecValTok{1}\NormalTok{]}\OperatorTok{=}\DecValTok{1}\OperatorTok{/}\NormalTok{(means.shape[}\DecValTok{0}\NormalTok{]}\OperatorTok{{-}}\DecValTok{1}\NormalTok{)}
\NormalTok{Weight\_start }\OperatorTok{=}\NormalTok{ np.matrix(Weight\_start)}
\end{Highlighting}
\end{Shaded}

Die Definition der Zielfunktion \texttt{relative\_opt1} bleibt
unverändert. Ebenso das Tuple \texttt{cons} der Nebenbedingungen.

Beachten Sie, dass nur die ersten acht Gewichte in \(w_P\) die
änderbaren Parameter der Optimierung darstellen, das neunte Gewicht in
\(w_P\) (für den S\&P500) ist fest auf null fixiert und nicht Gegenstand
der Optimierung. Dieses Prinzip gilt immer für alle Gewichte in \(P\),
die explizit auf null gesetzt und fixiert werden, weil die zugehörigen
Anlagen nicht Elemente des Anlageuniversums von \(P\) sind.

Wir implementieren die Nullgewichtung der Benchmark in \(w_p\) über das
Tuple \texttt{bounds} der Bestandsgrenzen indem wir
\texttt{bounds\ +=\ ((0.0,\ 1e-10),\ )} setzen. Hierdurch fügen wir am
Ende von \texttt{bounds} ein weiteres Tuple der Form
\texttt{(0.0,\ 1e-10)} hinzu. \textbf{Wichtig:} die Obergrenze des
Tuples muss zwingend größer sein als die Untergenze (daher die Wahl von
1e-10). Dies hat aber keinen Einfluss auf die Optimierungsergebnisse.

Mit \(\lambda=-0.1075445\) ergibt sich der folgende optimierte
Anteilsvektor.

\begin{Shaded}
\begin{Highlighting}[]
\NormalTok{bound }\OperatorTok{=}\NormalTok{ (}\FloatTok{0.05}\NormalTok{,}\FloatTok{0.40}\NormalTok{)}
\NormalTok{bounds }\OperatorTok{=} \BuiltInTok{tuple}\NormalTok{(bound }\ControlFlowTok{for}\NormalTok{ asset }\KeywordTok{in} \BuiltInTok{range}\NormalTok{(alpha.shape[}\DecValTok{1}\NormalTok{]}\OperatorTok{{-}}\DecValTok{1}\NormalTok{))}
\CommentTok{\# weight of the benchmark is fixed to zero; Important: lb==ub is no longer allowed!}
\NormalTok{bounds }\OperatorTok{+=}\NormalTok{ ((}\FloatTok{0.0}\NormalTok{, }\FloatTok{1e{-}10}\NormalTok{), ) }
\NormalTok{lambda0 }\OperatorTok{=} \OperatorTok{{-}}\FloatTok{0.1075445}

\NormalTok{cons }\OperatorTok{=}\NormalTok{ (\{}\StringTok{\textquotesingle{}type\textquotesingle{}}\NormalTok{: }\StringTok{\textquotesingle{}eq\textquotesingle{}}\NormalTok{, }\StringTok{\textquotesingle{}fun\textquotesingle{}}\NormalTok{: }\KeywordTok{lambda}\NormalTok{ x: np.}\BuiltInTok{sum}\NormalTok{(x) }\OperatorTok{{-}} \DecValTok{1}\NormalTok{\},}
\NormalTok{        \{}\StringTok{\textquotesingle{}type\textquotesingle{}}\NormalTok{: }\StringTok{\textquotesingle{}eq\textquotesingle{}}\NormalTok{, }\StringTok{\textquotesingle{}fun\textquotesingle{}}\NormalTok{: }\KeywordTok{lambda}\NormalTok{ x: (np.matrix(x)}\OperatorTok{*}\NormalTok{beta.T)[}\DecValTok{0}\NormalTok{,}\DecValTok{0}\NormalTok{] }\OperatorTok{{-}} \DecValTok{1}\NormalTok{\})}

\NormalTok{res1 }\OperatorTok{=}\NormalTok{ minimize(relative\_opt1, Weight\_start, args}\OperatorTok{=}\NormalTok{lambda0,}
\NormalTok{                        method}\OperatorTok{=}\StringTok{\textquotesingle{}SLSQP\textquotesingle{}}\NormalTok{, bounds}\OperatorTok{=}\NormalTok{bounds, constraints}\OperatorTok{=}\NormalTok{cons,tol}\OperatorTok{=}\FloatTok{1e{-}10}\NormalTok{)}
\NormalTok{pd.DataFrame([}\BuiltInTok{round}\NormalTok{(x,}\DecValTok{4}\NormalTok{) }\ControlFlowTok{for}\NormalTok{ x }\KeywordTok{in}\NormalTok{ res1.x],index}\OperatorTok{=}\NormalTok{ex\_returns.columns).T}
\end{Highlighting}
\end{Shaded}

\begin{longtable}[]{@{}llllllllll@{}}
\toprule\noalign{}
& COST & CSCO & IBM & INTC & MRK & MSFT & T & XOM & S\&P500 \\
\midrule\noalign{}
\endhead
\bottomrule\noalign{}
\endlastfoot
0 & 0.05 & 0.4 & 0.05 & 0.05 & 0.296 & 0.05 & 0.05 & 0.054 & 0.0 \\
\end{longtable}

Kurzer Check ob Timing Bedingung (Portfolio-Beta gleich eins, aktives
Beta gleich null) eingehalten wird.

\begin{Shaded}
\begin{Highlighting}[]
\NormalTok{port\_beta(res1.x,beta)}
\end{Highlighting}
\end{Shaded}

\begin{verbatim}
1.0000000000585922
\end{verbatim}

\begin{Shaded}
\begin{Highlighting}[]
\CommentTok{\# portfolio variance}
\NormalTok{(np.matrix(res1.x)}\OperatorTok{*}\NormalTok{ Sigma}\OperatorTok{*}\NormalTok{ np.matrix(res1.x).T)[}\DecValTok{0}\NormalTok{,}\DecValTok{0}\NormalTok{]}
\end{Highlighting}
\end{Shaded}

\begin{verbatim}
0.0030267295902247117
\end{verbatim}

\begin{Shaded}
\begin{Highlighting}[]
\CommentTok{\# portfolio alpha}
\NormalTok{(np.matrix(res1.x)}\OperatorTok{*}\NormalTok{alpha.T)[}\DecValTok{0}\NormalTok{,}\DecValTok{0}\NormalTok{]}
\end{Highlighting}
\end{Shaded}

\begin{verbatim}
0.006116489277362495
\end{verbatim}

\subsection{Literatur}\label{literatur-1}

Drobetz, W. (2003). Einsatz des Black-Litterman-Verfahrens in der Asset
Allocation, in: Dichtl, H., Kleeberg, J., und C. Schlenger (Hrsg.),
Handbuch Asset Allocation, Uhlenbruch Verlag: Bad Soden/Ts.

Grinold, R. C., Kahn, R. N. (2000). Active Portfoliomanagement,
Quantitative Theory and Applications, 2. Auflage, New York u.a..

Poddig, T., Brinkmann, U., Seiler, K. (2009). Portfolio Management:
Konzepte und Strategien, 2. Auflage, Uhlenbruch Verlag, Bad Soden/Ts..

\bookmarksetup{startatroot}

\chapter{Die Bedeutung von Schätzrisiken in der
Portfoliotheorie}\label{die-bedeutung-von-schuxe4tzrisiken-in-der-portfoliotheorie}

Die folgende Darstellung ist angelehnt an:

\begin{itemize}
\tightlist
\item
  Kempf, A./Memmel, C. \emph{Schätzrisiken in der Portfoliotheorie}. In:
  Kleeberg, J.M. \& Rehkugler, H. (Hrsg.): Handbuch Portfoliomanagement,
  2. Auflage, Bad Soden/Ts. 2002, S. 895-919.
\item
  Memmel, C. \emph{Schätzrisiken in der Portfoliotheorie: Auswirkungen
  und Möglichkeiten der Reduktion,} Eul Verlag, Lohmar, 2004.
\end{itemize}

\section{Einleitung}\label{einleitung}

In seinem bahnbrechenden \emph{Journal of Finance} Artikel ``Portfolio
Selection'' aus dem Jahr 1952 legt \textbf{Harry Markowitz} dar, dass
risikoaverse Investoren sich bei ihrer Anlageentscheidung an
Erwartungswert und Varianz der Rendite ihres Gesamtportfolios
orientieren sollten. Der Anleger strebt danach, eine vorgegebene
erwartete Portfoliorendite mit dem geringsten Risiko zu erreichen. Die
mathematische Formalisierung dieser Idee führt zu einem quadratischen
Optimierungsproblem mit N-1 Entscheidungsparametern, wobei N die Anzahl
der zulässigen Anlageinstrumente darstellt.

Die praktische Implementierung des Portfolioansatzes von Markowitz ist
mit einem zentralen Problem verbunden. Der Anleger kennt die Parameter
der Renditeverteilungen (Erwartungswerte, Varianzen, Kovarianzen) nicht.
Diese Parameter können beispielsweise aus den fundamentalen
Unternehmensdaten ermittelt werden oder - und darauf werden wir uns
konzentrieren - aus historischen Zeitreihen geschätzt werden. Diese
Schätzung stellt für große Wertpapierportfolios zunächst in
quantitativer Hinsicht eine Hausforderung dar, da N erwartete Renditen
und Varianzen bzw. 0,5 · N · (N-1) Kovarianzen zu schätzen sind. Selbst
wenn das Anlageuniversum eines Investors nur aus 500 Aktien besteht,
sind bereits 125.750 Parameter zu schätzen. Neben diesem reinen
Mengenproblem besteht allerdings ein zweites gravierenderes Problem, das
im Zentrum der folgenden Ausführungen stehen wird. Jede Schätzung ist
mit einem Schätzrisiko verbunden, d.h., der geschätzte Parameter wird im
allgemeinen nicht dem (unbeobachtbaren) wahren Parameter der
Renditeverteilung entsprechen. Dies kann zu suboptimalen
Portfoliozusammensetzungen und damit zu mangelndem Anlageerfolg führen.
Das Ziel dieses Notebooks besteht darin, die Bedeutung des Schätzrisikos
für die Portfoliozusammensetzung aufzuzeigen und anschließend Wege
darzustellen, wie der Einfluss des Schätzrisikos auf die
Portfoliozusammensetzung verringert werden kann.

\subsection{2. Einfluß des Schätzfehlers auf die
Portfoliozusammensetzung}\label{einfluuxdf-des-schuxe4tzfehlers-auf-die-portfoliozusammensetzung}

Zunächst werden wir in einer kleinen Simulationsstudie aufzeigen, wie
stark das Schätzrisiko die Portfoliozusammensetzung beeinflusst. Hierzu
simulieren wir für vier Aktien unabhängig identisch normalverteilte
Wochenrenditen über einen Zeitraum von zwei Jahren. Für jede Aktie
unterstellen wir einen Erwartungswert der Rendite von 11\% p.a. und eine
Standardabweichung von 25\% p.a. Außerdem nehmen wir an, dass die
Renditen der verschiedenen Aktien paarweise eine Korrelation von 0,3
aufweisen. Daneben unterstellen wir, dass es ein risikoloses Instrument
mit einer Rendite \(r_f\) = 6\% p.a. gibt. Dieses risikolose Instrument
berücksichtigen wir, damit die optimale Zusammensetzung des
Aktienportfolios unabhangig vom Ausmaß der Risikoaversion des Anlegers
ist (Tobin-Separation). Tobin (1958) hat gezeigt, dass alle Anleger in
diesem Fall unabhängig von dem Ausmaß ihrer Risikoaversion das gleiche
Aktienportfolio halten, das sogenannte Tangentialportfolio.

Um die Größenordnung der Schätzrisiken zu verdeutlichen, analysieren wir
die optimale Zusammensetzung des Tangentialportfolios eines Anlegers in
zwei Fällen. Im ersten Fall kennt der Anleger die obigen
Verteilungsparameter, während er sie im zweiten Fall aus den simulierten
Renditerealisationen schätzen muss. Kennt der Anleger die
Verteilungsparameter, so ist es für ihn optimal, sein in Aktien
anzulegendes Vermögen gleichmäßig auf die vier Aktien zu verteilen.
Diese Gleichverteilungsstrategie führt zu einer erwarteten
Aktienportfoliorendite von 11\% p.a. bei einer Standardabweichung von
17,23\% p.a. Betrachten wir nun den Fall, dass der Anleger die
Verteilungsparameter aus den realisierten Renditen schätzt.

\begin{Shaded}
\begin{Highlighting}[]
\CommentTok{\# import of necessary libaries}
\ImportTok{import}\NormalTok{ pandas }\ImportTok{as}\NormalTok{ pd}
\ImportTok{import}\NormalTok{ numpy }\ImportTok{as}\NormalTok{ np}
\ImportTok{from}\NormalTok{ scipy.optimize }\ImportTok{import}\NormalTok{ minimize}
\ImportTok{import}\NormalTok{ matplotlib.pyplot }\ImportTok{as}\NormalTok{ plt}
\end{Highlighting}
\end{Shaded}

\textbf{Ausgangssituation für die Simulation}:

\begin{itemize}
\item
  wahre Parameter (annualisiert):
  \(\mu_i=0,11, \sigma_i=0,25, \rho_{ij}=0,3\), für \(i,j =1,..., 4,\)
  und \(i \neq j\).
\item
  Die Varianz auf Wochenebene beträgt: \(0,25^{2}/52=0,00120192\).
\item
  Für die wöchentliche erwartete Rendite gilt: \(0,11/52=0,00211538\).
\item
  Der konstante Korrelationskoeffizient ist 0,3. Damit folgt für die
  wöchentliche Kovarianz: \$ 0,25\^{}\{2\}/52 * 0,3=0,00036058\$.
\end{itemize}

\begin{Shaded}
\begin{Highlighting}[]
\CommentTok{\# Die wöchentliche Varianz{-}Kovarianzmatrix nennen wir *Sigma\_true*:}
\NormalTok{Sigma\_true}\OperatorTok{=}\NormalTok{[[}\FloatTok{0.00120192}\NormalTok{, }\FloatTok{0.00036058}\NormalTok{, }\FloatTok{0.00036058}\NormalTok{, }\FloatTok{0.00036058}\NormalTok{],}\OperatorTok{\textbackslash{}}
\NormalTok{            [}\FloatTok{0.00036058}\NormalTok{, }\FloatTok{0.00120192}\NormalTok{, }\FloatTok{0.00036058}\NormalTok{, }\FloatTok{0.00036058}\NormalTok{],}\OperatorTok{\textbackslash{}}
\NormalTok{            [}\FloatTok{0.00036058}\NormalTok{, }\FloatTok{0.00036058}\NormalTok{, }\FloatTok{0.00120192}\NormalTok{, }\FloatTok{0.00036058}\NormalTok{],}\OperatorTok{\textbackslash{}}
\NormalTok{            [}\FloatTok{0.00036058}\NormalTok{, }\FloatTok{0.00036058}\NormalTok{, }\FloatTok{0.00036058}\NormalTok{, }\FloatTok{0.00120192}\NormalTok{]]}

\CommentTok{\# Der Vektor der erwarteten, wöchentlichen Renditen:}
\NormalTok{means\_true}\OperatorTok{=}\NormalTok{[}\FloatTok{0.00211538}\NormalTok{, }\FloatTok{0.00211538}\NormalTok{, }\FloatTok{0.00211538}\NormalTok{, }\FloatTok{0.00211538}\NormalTok{]}
\end{Highlighting}
\end{Shaded}

\begin{Shaded}
\begin{Highlighting}[]
\CommentTok{\# eine Monte Carlo Ziehung mit jeweils einer 2{-}jährigen Zeitreihe }
\CommentTok{\# (104 Wochen) für jede Aktie}
\NormalTok{np.random.seed(}\DecValTok{42}\NormalTok{)}
\NormalTok{df }\OperatorTok{=}\NormalTok{ pd.DataFrame(np.asarray(np.random.multivariate\_normal(means\_true,}\OperatorTok{\textbackslash{}}
\NormalTok{                Sigma\_true, size }\OperatorTok{=} \DecValTok{104}\NormalTok{)), columns}\OperatorTok{=}\NormalTok{[}\StringTok{\textquotesingle{}A1\textquotesingle{}}\NormalTok{, }\StringTok{\textquotesingle{}A2\textquotesingle{}}\NormalTok{, }\StringTok{\textquotesingle{}A3\textquotesingle{}}\NormalTok{, }\StringTok{\textquotesingle{}A4\textquotesingle{}}\NormalTok{])}
\NormalTok{means\_est }\OperatorTok{=}\NormalTok{ df.mean().values }\OperatorTok{*} \DecValTok{52} \CommentTok{\# annualsiert}
\NormalTok{Sigma\_est }\OperatorTok{=}\NormalTok{ df.cov().values }\OperatorTok{*} \DecValTok{52} \CommentTok{\# annualisiert}
\NormalTok{std\_est}\OperatorTok{=}\NormalTok{np.sqrt(np.diag(Sigma\_est))}
\end{Highlighting}
\end{Shaded}

In den folgenden zwei Tabellen sind die geschätzten Parameter eines
exemplarischen Zufallspfades der Simulation sowie die wahren Werte
angegeben. Zunächst ein Vergleich der wahren und geschätzen erwarteten
Renditen:

\begin{Shaded}
\begin{Highlighting}[]
\NormalTok{pd.DataFrame(\{}\StringTok{\textquotesingle{}wahr\textquotesingle{}}\NormalTok{: }\FloatTok{0.11}\NormalTok{, }\StringTok{\textquotesingle{}geschätzt\textquotesingle{}}\NormalTok{: means\_est, }\OperatorTok{\textbackslash{}}
              \StringTok{\textquotesingle{}Schätzfehler (\%)\textquotesingle{}}\NormalTok{: (}\FloatTok{0.11}\OperatorTok{{-}}\NormalTok{means\_est)}\OperatorTok{*}\DecValTok{100}\NormalTok{\}, index}\OperatorTok{=}\NormalTok{df.columns)}
\end{Highlighting}
\end{Shaded}

\begin{longtable}[]{@{}llll@{}}
\toprule\noalign{}
& wahr & geschätzt & Schätzfehler (\%) \\
\midrule\noalign{}
\endhead
\bottomrule\noalign{}
\endlastfoot
A1 & 0.11 & 0.193280 & -8.328017 \\
A2 & 0.11 & 0.214416 & -10.441623 \\
A3 & 0.11 & 0.099346 & 1.065403 \\
A4 & 0.11 & 0.145107 & -3.510683 \\
\end{longtable}

Und nun der Vergleich der wahren (0,25) und der geschätzten
Standardabweichung:

\begin{Shaded}
\begin{Highlighting}[]
\NormalTok{pd.DataFrame(\{}\StringTok{\textquotesingle{}wahr\textquotesingle{}}\NormalTok{: }\FloatTok{0.25}\NormalTok{, }\StringTok{\textquotesingle{}geschätzt\textquotesingle{}}\NormalTok{: std\_est, }\OperatorTok{\textbackslash{}}
              \StringTok{\textquotesingle{}Schätzfehler (\%)\textquotesingle{}}\NormalTok{: (}\FloatTok{0.25}\OperatorTok{{-}}\NormalTok{std\_est)}\OperatorTok{*}\DecValTok{100}\NormalTok{\}}\OperatorTok{\textbackslash{}}
\NormalTok{             , index}\OperatorTok{=}\NormalTok{df.columns)}
\end{Highlighting}
\end{Shaded}

\begin{longtable}[]{@{}llll@{}}
\toprule\noalign{}
& wahr & geschätzt & Schätzfehler (\%) \\
\midrule\noalign{}
\endhead
\bottomrule\noalign{}
\endlastfoot
A1 & 0.25 & 0.233158 & 1.684243 \\
A2 & 0.25 & 0.226982 & 2.301796 \\
A3 & 0.25 & 0.236087 & 1.391323 \\
A4 & 0.25 & 0.239289 & 1.071104 \\
\end{longtable}

Berechnen wir nun die optimalen Gewichte für das Tangentialportfolio,
d.h., das Portfolio mit der maximalen Sharpe Ratio. Wir setzen als
Nebenbedingung nur die Budget-Restriktion: kein Leverage
(Fremdkapitalaufnahme), Summe der Portfoliogewichte entspricht 1.

\begin{Shaded}
\begin{Highlighting}[]
\CommentTok{\# function that implements the Sharpe portfolio optimization}
\CommentTok{\# with no short sale constraint!}

\CommentTok{\# definition of target function for maximum Sharpe portfolio}
\KeywordTok{def}\NormalTok{ calc\_neg\_sharpe(weights, mean\_returns, cov, rf):}
\NormalTok{    portfolio\_return }\OperatorTok{=}\NormalTok{ np.}\BuiltInTok{sum}\NormalTok{(mean\_returns }\OperatorTok{*}\NormalTok{ weights)}
\NormalTok{    portfolio\_std }\OperatorTok{=}\NormalTok{ np.sqrt(np.dot(weights.T, np.dot(cov, weights)))}
\NormalTok{    sharpe\_ratio }\OperatorTok{=}\NormalTok{ (portfolio\_return }\OperatorTok{{-}}\NormalTok{ rf) }\OperatorTok{/}\NormalTok{ portfolio\_std}
    \ControlFlowTok{return} \OperatorTok{{-}}\NormalTok{sharpe\_ratio}

\KeywordTok{def}\NormalTok{ max\_sharpe\_ratio(mean\_returns, cov, rf):}
\NormalTok{    num\_assets }\OperatorTok{=} \BuiltInTok{len}\NormalTok{(mean\_returns)}
\NormalTok{    args }\OperatorTok{=}\NormalTok{ (mean\_returns, cov, rf)}
\NormalTok{    constraints }\OperatorTok{=}\NormalTok{ (\{}\StringTok{\textquotesingle{}type\textquotesingle{}}\NormalTok{: }\StringTok{\textquotesingle{}eq\textquotesingle{}}\NormalTok{, }\StringTok{\textquotesingle{}fun\textquotesingle{}}\NormalTok{: }\KeywordTok{lambda}\NormalTok{ x: np.}\BuiltInTok{sum}\NormalTok{(x) }\OperatorTok{{-}} \DecValTok{1}\NormalTok{\})}
\NormalTok{    result }\OperatorTok{=}\NormalTok{ minimize(calc\_neg\_sharpe, num\_assets}\OperatorTok{*}\NormalTok{[}\FloatTok{1.}\OperatorTok{/}\NormalTok{num\_assets,], args}\OperatorTok{=}\NormalTok{args,}
\NormalTok{                        method}\OperatorTok{=}\StringTok{\textquotesingle{}SLSQP\textquotesingle{}}\NormalTok{, constraints}\OperatorTok{=}\NormalTok{constraints,tol}\OperatorTok{=}\FloatTok{1e{-}10}\NormalTok{)}
\NormalTok{    weights}\OperatorTok{=}\NormalTok{[}\BuiltInTok{round}\NormalTok{(x,}\DecValTok{4}\NormalTok{) }\ControlFlowTok{for}\NormalTok{ x }\KeywordTok{in}\NormalTok{ result[}\StringTok{\textquotesingle{}x\textquotesingle{}}\NormalTok{]]}
    \ControlFlowTok{return}\NormalTok{ weights}
\end{Highlighting}
\end{Shaded}

\begin{Shaded}
\begin{Highlighting}[]
\NormalTok{weights}\OperatorTok{=}\NormalTok{max\_sharpe\_ratio(means\_est, Sigma\_est, }\FloatTok{0.06}\NormalTok{)}
\end{Highlighting}
\end{Shaded}

Graphischer Vergleich der optimalen Portfoliogewichte: tatsächlich (1/N)
versus ermittelt auf Basis geschätzter Input-Parameter:

\begin{Shaded}
\begin{Highlighting}[]
\NormalTok{pd.DataFrame(\{}\StringTok{\textquotesingle{}wahres Gewicht\textquotesingle{}}\NormalTok{: }\FloatTok{0.25}\NormalTok{, }\StringTok{\textquotesingle{}geschätztes Gewicht\textquotesingle{}}\NormalTok{: weights\},index}\OperatorTok{=}\NormalTok{df.columns). }\OperatorTok{\textbackslash{}}
\NormalTok{plot.bar(stacked}\OperatorTok{=}\VariableTok{False}\NormalTok{, alpha}\OperatorTok{=}\FloatTok{0.5}\NormalTok{, figsize}\OperatorTok{=}\NormalTok{(}\DecValTok{10}\NormalTok{,}\DecValTok{5}\NormalTok{))}\OperatorTok{;}
\end{Highlighting}
\end{Shaded}

\pandocbounded{\includegraphics[keepaspectratio]{kapitel3_files/figure-pdf/cell-9-output-1.png}}

Zwei Ergebnisse fallen auf. Das erste bemerkenswerte Ergebnis besteht
darin, dass sich die optimalen Gewichte auf Basis der geschätzten
Parameter sehr stark von den wahren optimalen Werten unterscheiden.
Offensichtlich schlagen sich Schätzfehler gravierend in der
Portfoliozusammensetzung nieder. Daneben weichen die Schätzwerte für die
erwartete Rendite sehr stark von deren wahren Wert ab, während die
Schätzwerte der Standardabweichung wesentlich näher am wahren Parameter
liegen. Dies ist ein erster Hinweis darauf, dass die Schätzung der
erwarteten Rendite die größten Schwierigkeiten verursacht. Diesem
Hinweis gehen wir im folgenden Abschnitt weiter nach.

\subsection{3. Komponenten des
Schätzfehlers}\label{komponenten-des-schuxe4tzfehlers}

Bisher haben wir den Einfluss des gesamten Schätzfehlers auf die
Portfoliozusammensetzung herausgearbeitet, ohne allerdings der Frage
nachzugehen, welche Quelle des Schätzfehlers (Fehlschätzung bei
erwarteten Renditen, Fehlschätzung bei Varianzen, Fehlschätzung bei
Korrelationen) sich besonders stark auf die Portfoliogewichte
durchschlägt. Die Stärke des Einflusses hängt hierbei von zwei Größen
ab, der Sensitivität der Portfoliozusammensetzung auf die Fehlschätzung
und dem Ausmaß der Fehlschätzung. Im folgenden werden wir zunächst die
Sensitivität der Portfoliozusammensetzung für die verschiedenen
Komponenten des Schätzfehlers analysieren und anschließend Fragen nach
der erreichbaren Schätzgenauigkeit für die verschiedenen Parameter
diskutieren.

\subsubsection{Sensitivitat des optimalen Portfolios bezüglich
Schätzfehlern in den verschiedenen
Parametern}\label{sensitivitat-des-optimalen-portfolios-bezuxfcglich-schuxe4tzfehlern-in-den-verschiedenen-parametern}

Zur Analyse der Sensitivitat der Portfoliogewichte in bezug auf
Fehlschätzungen in den einzelnen Parametern verwenden wir wiederum die
vier Aktien des Abschnittes 1 mit den wahren Parametern \(\mu_i\) = 11\%
(erwartete Rendite), \(\sigma_i\) = 25\% (Standardabweichung) und
\(rho_{ij}\) = 30\% (Korrelationskoeffizient). Als Vergleichsmaßstab
betrachten wir den Fall, das der Anleger sämtliche Parameter exakt kennt
und sein optimales Aktienportfolio deshalb aus gleichen Anteilen in
allen vier Aktien besteht. Hiermit werden drei Fälle verglichen, in
denen der Anleger einen Schätzfehler bezüglich der Parameter der Aktie 1
begeht. Im ersten Fall täuscht er sich bei der erwarteten Rendite der
Aktie 1, im zweiten Fall bei der Standardabweichung der Aktie 1 und im
dritten Fall bei der Korrelation zwischen den Renditen von Aktie 1 und
Aktie 2. Die einzelnen Fälle zur Bestimmung der Komponenten des
Schatzrisikos sind in der nachfolgenden Tabelle dargestellt.

\pandocbounded{\includegraphics[keepaspectratio]{figure31.png}}

Wir untersuchen den Einfluß des Schätzfehlers, indem wir in einer
komparativ-statischen Analyse den jeweiligen mit Schätzfehler versehenen
Parameter ausgehend vom wahren Wert um ±10 Prozentpunkten in Schritten
der Größe 2,0 Prozentpunkte variieren. Die erwartete Rendite bewegt sich
somit im Wertebereich zwischen 1\% und 21\%, die Volatilitat zwischen
15\% und 35\% und die Korrelation zwischen 20\% und 40\%. Für die
verschiedenen Parameterkonstellationen bestimmen wir die optimale
Portfoliozusammensetzung und vergleichen diese mit der optimalen
Portfoliozusammensetzung bei Kenntnis aller Parameter.

\textbf{Fall 1: Schätzfehler in der erwarteten Rendite von Aktie 1,
zwischen 1\% und 21\% um den wahren Wert 11\%.}

\begin{Shaded}
\begin{Highlighting}[]
\CommentTok{\# Berechnung der Fehlgewichtung in A1 für Fehler in der geschätzten}
\CommentTok{\# erwarteten Rendite}
\NormalTok{est\_error\_mean}\OperatorTok{=}\NormalTok{np.linspace(}\FloatTok{0.01}\NormalTok{, }\FloatTok{0.21}\NormalTok{, }\DecValTok{11}\NormalTok{) }\CommentTok{\# symmetrische Fehler um wahren Wert 11\%}
\NormalTok{wa1\_means}\OperatorTok{=}\NormalTok{np.zeros(}\DecValTok{11}\NormalTok{) }\CommentTok{\# enthält die optimierten Gewichte}
\ControlFlowTok{for}\NormalTok{ i }\KeywordTok{in} \BuiltInTok{range}\NormalTok{(}\DecValTok{11}\NormalTok{):}
\NormalTok{    means\_mod}\OperatorTok{=}\NormalTok{np.multiply(means\_true,}\DecValTok{52}\NormalTok{) }\CommentTok{\# annualisieren nicht vergessen!}
\NormalTok{    Sigma\_true\_ann}\OperatorTok{=}\NormalTok{np.multiply(Sigma\_true,}\DecValTok{52}\NormalTok{) }\CommentTok{\# annualisieren nicht vergessen!}
\NormalTok{    means\_mod[}\DecValTok{0}\NormalTok{]}\OperatorTok{=}\NormalTok{est\_error\_mean[i]}
\NormalTok{    weights}\OperatorTok{=}\NormalTok{max\_sharpe\_ratio(means\_mod, Sigma\_true\_ann, }\FloatTok{0.06}\NormalTok{)}
\NormalTok{    wa1\_means[i]}\OperatorTok{=}\NormalTok{weights[}\DecValTok{0}\NormalTok{]}
    
\end{Highlighting}
\end{Shaded}

\textbf{Fall 2: Schätzfehler in der Standardabweichung der Rendite von
Aktie 1, zwischen 15\% und 35\% um den wahren Wert 25\%.}

\begin{Shaded}
\begin{Highlighting}[]
\CommentTok{\# Berechnung der Fehlgewichtung in A1 für Fehler in der geschätzten}
\CommentTok{\# (annualisierten) Standardabweichung}
\NormalTok{est\_error\_std}\OperatorTok{=}\NormalTok{np.linspace(}\FloatTok{0.15}\NormalTok{, }\FloatTok{0.35}\NormalTok{, }\DecValTok{11}\NormalTok{) }\CommentTok{\# symmetrische Fehler um wahren Wert 25\%}
\CommentTok{\# Berechnung der neuen Varianz und der Kovarianz}
\NormalTok{a1\_var}\OperatorTok{=}\NormalTok{np.zeros(}\DecValTok{11}\NormalTok{) }\CommentTok{\# enthält die neue Varianz}
\NormalTok{a1\_cov}\OperatorTok{=}\NormalTok{np.zeros(}\DecValTok{11}\NormalTok{) }\CommentTok{\# enthält die neue Kovarianz}
\NormalTok{wa1\_std}\OperatorTok{=}\NormalTok{np.zeros(}\DecValTok{11}\NormalTok{) }\CommentTok{\# enthält die optimierten Gewichte}

\ControlFlowTok{for}\NormalTok{ i }\KeywordTok{in} \BuiltInTok{range}\NormalTok{(}\DecValTok{11}\NormalTok{):}
\NormalTok{    a1\_var[i]}\OperatorTok{=}\NormalTok{est\_error\_std[i]}\OperatorTok{**}\DecValTok{2}
\NormalTok{    a1\_cov[i]}\OperatorTok{=}\NormalTok{est\_error\_std[i]}\OperatorTok{*}\FloatTok{0.25}\OperatorTok{*}\FloatTok{0.3}
    \CommentTok{\# Berechnung der neuen Varanz{-}Kovarianzmatrix; 7 Änderungen vornehmen}
\NormalTok{    Sigma\_true\_ann}\OperatorTok{=}\NormalTok{np.multiply(Sigma\_true,}\DecValTok{52}\NormalTok{) }\CommentTok{\# annualisieren nicht vergessen!}
\NormalTok{    Sigma\_mod}\OperatorTok{=}\NormalTok{Sigma\_true\_ann}
\NormalTok{    Sigma\_mod[}\DecValTok{0}\NormalTok{,}\DecValTok{0}\NormalTok{]}\OperatorTok{=}\NormalTok{a1\_var[i]}
\NormalTok{    Sigma\_mod[}\DecValTok{0}\NormalTok{,}\DecValTok{1}\NormalTok{]}\OperatorTok{=}\NormalTok{a1\_cov[i]}
\NormalTok{    Sigma\_mod[}\DecValTok{0}\NormalTok{,}\DecValTok{2}\NormalTok{]}\OperatorTok{=}\NormalTok{a1\_cov[i]}
\NormalTok{    Sigma\_mod[}\DecValTok{0}\NormalTok{,}\DecValTok{3}\NormalTok{]}\OperatorTok{=}\NormalTok{a1\_cov[i]}
\NormalTok{    Sigma\_mod[}\DecValTok{1}\NormalTok{,}\DecValTok{0}\NormalTok{]}\OperatorTok{=}\NormalTok{a1\_cov[i]}
\NormalTok{    Sigma\_mod[}\DecValTok{2}\NormalTok{,}\DecValTok{0}\NormalTok{]}\OperatorTok{=}\NormalTok{a1\_cov[i]}
\NormalTok{    Sigma\_mod[}\DecValTok{3}\NormalTok{,}\DecValTok{0}\NormalTok{]}\OperatorTok{=}\NormalTok{a1\_cov[i]}
\NormalTok{    means\_ann}\OperatorTok{=}\NormalTok{np.multiply(means\_true,}\DecValTok{52}\NormalTok{) }\CommentTok{\# annualisieren nicht vergessen!}
\NormalTok{    weights}\OperatorTok{=}\NormalTok{max\_sharpe\_ratio(means\_ann, Sigma\_mod, }\FloatTok{0.06}\NormalTok{)}
\NormalTok{    wa1\_std[i]}\OperatorTok{=}\NormalTok{weights[}\DecValTok{0}\NormalTok{]}
   
\end{Highlighting}
\end{Shaded}

\textbf{Fall 3: Schätzfehler in der Korrelation der Rendite von Aktie 1,
zwischen 20\% und 40\% um den wahren Wert 30\%.}

\begin{Shaded}
\begin{Highlighting}[]
\CommentTok{\# Berechnung der Fehlgewichtung in A1 für Fehler in der geschätzten}
\CommentTok{\# Korrelation zwischen Aktie 1 und 2}
\NormalTok{est\_error\_corr}\OperatorTok{=}\NormalTok{np.linspace(}\FloatTok{0.20}\NormalTok{, }\FloatTok{0.40}\NormalTok{, }\DecValTok{11}\NormalTok{) }\CommentTok{\# symmetrische Fehler um wahren Wert 25\%}
\CommentTok{\# Berechnung der neuen Kovarianz}
\NormalTok{a1\_cov}\OperatorTok{=}\NormalTok{np.zeros(}\DecValTok{11}\NormalTok{) }\CommentTok{\# enthält die neue Kovarianz}
\NormalTok{wa1\_corr}\OperatorTok{=}\NormalTok{np.zeros(}\DecValTok{11}\NormalTok{) }\CommentTok{\# enthält die optimierten Gewichte}

\ControlFlowTok{for}\NormalTok{ i }\KeywordTok{in} \BuiltInTok{range}\NormalTok{(}\DecValTok{11}\NormalTok{):}
\NormalTok{    a1\_cov[i]}\OperatorTok{=}\NormalTok{est\_error\_corr[i]}\OperatorTok{*}\FloatTok{0.25}\OperatorTok{*}\FloatTok{0.25}
    \CommentTok{\# Berechnung der neuen Varanz{-}Kovarianzmatrix; 2 Änderungen vornehmen}
\NormalTok{    Sigma\_true\_ann}\OperatorTok{=}\NormalTok{np.multiply(Sigma\_true,}\DecValTok{52}\NormalTok{) }\CommentTok{\# annualisieren nicht vergessen!}
\NormalTok{    Sigma\_mod}\OperatorTok{=}\NormalTok{Sigma\_true\_ann}
\NormalTok{    Sigma\_mod[}\DecValTok{0}\NormalTok{,}\DecValTok{1}\NormalTok{]}\OperatorTok{=}\NormalTok{a1\_cov[i]}
\NormalTok{    Sigma\_mod[}\DecValTok{1}\NormalTok{,}\DecValTok{0}\NormalTok{]}\OperatorTok{=}\NormalTok{a1\_cov[i]}
\NormalTok{    means\_ann}\OperatorTok{=}\NormalTok{np.multiply(means\_true,}\DecValTok{52}\NormalTok{) }\CommentTok{\# annualisieren nicht vergessen!}
\NormalTok{    weights}\OperatorTok{=}\NormalTok{max\_sharpe\_ratio(means\_ann, Sigma\_mod, }\FloatTok{0.06}\NormalTok{)}
\NormalTok{    wa1\_corr[i]}\OperatorTok{=}\NormalTok{weights[}\DecValTok{0}\NormalTok{]}
   
\end{Highlighting}
\end{Shaded}

Die folgende Tabelle gibt die optimalen Gewichte für Aktie 1 für die
verschiedenen Fälle an.

\begin{Shaded}
\begin{Highlighting}[]
\NormalTok{df}\OperatorTok{=}\NormalTok{pd.DataFrame(\{}\StringTok{\textquotesingle{}1. Fall\textquotesingle{}}\NormalTok{: np.}\BuiltInTok{round}\NormalTok{(wa1\_means,}\DecValTok{2}\NormalTok{), }\StringTok{\textquotesingle{}2. Fall\textquotesingle{}}\NormalTok{: np.}\BuiltInTok{round}\NormalTok{(wa1\_std,}\DecValTok{2}\NormalTok{), }\OperatorTok{\textbackslash{}}
              \StringTok{\textquotesingle{}3. Fall\textquotesingle{}}\NormalTok{: np.}\BuiltInTok{round}\NormalTok{(wa1\_corr,}\DecValTok{2}\NormalTok{)\}}\OperatorTok{\textbackslash{}}
\NormalTok{             , index}\OperatorTok{=}\NormalTok{np.linspace(}\OperatorTok{{-}}\DecValTok{10}\NormalTok{, }\DecValTok{10}\NormalTok{, }\DecValTok{11}\NormalTok{)).T}
\NormalTok{df.columns.name}\OperatorTok{=}\StringTok{\textquotesingle{}Abweichungen vom wahren Parameterwert in (\%)\textquotesingle{}}
\NormalTok{df}
\end{Highlighting}
\end{Shaded}

\begin{longtable}[]{@{}llllllllllll@{}}
\toprule\noalign{}
Abweichungen vom wahren Parameterwert in (\%) & -10.0 & -8.0 & -6.0 &
-4.0 & -2.0 & 0.0 & 2.0 & 4.0 & 6.0 & 8.0 & 10.0 \\
\midrule\noalign{}
\endhead
\bottomrule\noalign{}
\endlastfoot
1. Fall & -1.79 & -1.11 & -0.62 & -0.26 & 0.02 & 0.25 & 0.44 & 0.59 &
0.72 & 0.83 & 0.93 \\
2. Fall & 0.66 & 0.56 & 0.47 & 0.38 & 0.31 & 0.25 & 0.20 & 0.16 & 0.12 &
0.09 & 0.07 \\
3. Fall & 0.27 & 0.27 & 0.26 & 0.26 & 0.25 & 0.25 & 0.25 & 0.24 & 0.24 &
0.24 & 0.23 \\
\end{longtable}

In der Tabelle ist klar zu erkennen, dass sich Änderungen der erwarteten
Rendite am stärksten auf die Portfoliozusammensetzung durchschlagen.
Wird beispielsweise die erwartete Rendite von Aktie 1 um 6 Prozentpunkte
überschätzt, so ergibt sich hierdurch ein optimales Gewicht für Aktie 1
in Höhe von 0,72. Im Vergleich zum optimalen Gewicht ohne Schätzfehler
bedeutet dies eine Übergewichtung der Aktie 1 um 47 Prozentpunkte. Eine
Fehlschätzung der Standardabweichung in gleicher Höhe schlägt sich
demgegenüber wesentlich weniger stark (-0,13 = 0,12 - 0,25) auf die
Portfoliozusammensetzung durch, und eine Fehlschätzung der Korrelation
besitzt keinen nennenswerten Einfluss (-0,01 = 0,24 - 0,25). Dies wird
anhand der nachfolgenden Abbildung nochmals verdeutlicht. Dort ist die
Über- oder Untergewichtung von Aktie 1 in Abhängigkeit der Stärke des
Schätzfehlers graphisch dargestellt.

\begin{Shaded}
\begin{Highlighting}[]
\NormalTok{fig1 }\OperatorTok{=}\NormalTok{ plt.figure(num}\OperatorTok{=}\DecValTok{1}\NormalTok{, facecolor}\OperatorTok{=}\StringTok{\textquotesingle{}w\textquotesingle{}}\NormalTok{, figsize}\OperatorTok{=}\NormalTok{(}\DecValTok{10}\NormalTok{, }\DecValTok{5}\NormalTok{))}
\NormalTok{ax }\OperatorTok{=}\NormalTok{ fig1.add\_subplot(}\DecValTok{111}\NormalTok{)}
\NormalTok{ax.spines[}\StringTok{\textquotesingle{}left\textquotesingle{}}\NormalTok{].set\_position(}\StringTok{\textquotesingle{}zero\textquotesingle{}}\NormalTok{)}
\NormalTok{ax.spines[}\StringTok{\textquotesingle{}right\textquotesingle{}}\NormalTok{].set\_position(}\StringTok{\textquotesingle{}zero\textquotesingle{}}\NormalTok{)}
\NormalTok{ax.spines[}\StringTok{\textquotesingle{}top\textquotesingle{}}\NormalTok{].set\_position(}\StringTok{\textquotesingle{}zero\textquotesingle{}}\NormalTok{)}

\NormalTok{ax.spines[}\StringTok{\textquotesingle{}bottom\textquotesingle{}}\NormalTok{].set\_position(}\StringTok{\textquotesingle{}zero\textquotesingle{}}\NormalTok{)}
\NormalTok{plt.plot(np.linspace(}\OperatorTok{{-}}\DecValTok{10}\NormalTok{, }\DecValTok{10}\NormalTok{, }\DecValTok{11}\NormalTok{), (wa1\_means}\OperatorTok{{-}}\FloatTok{0.25}\NormalTok{)}\OperatorTok{*}\DecValTok{100}\NormalTok{, }\StringTok{\textquotesingle{}r{-}\textquotesingle{}}\NormalTok{, label}\OperatorTok{=}\StringTok{\textquotesingle{}Fehler in der erwarteten Rendite\textquotesingle{}}\NormalTok{)}
\NormalTok{plt.plot(np.linspace(}\OperatorTok{{-}}\DecValTok{10}\NormalTok{, }\DecValTok{10}\NormalTok{, }\DecValTok{11}\NormalTok{), (wa1\_std}\OperatorTok{{-}}\FloatTok{0.25}\NormalTok{)}\OperatorTok{*}\DecValTok{100}\NormalTok{, }\StringTok{\textquotesingle{}g{-}\textquotesingle{}}\NormalTok{, label}\OperatorTok{=}\StringTok{\textquotesingle{}Fehler in der Standardabweichung\textquotesingle{}}\NormalTok{)}
\NormalTok{plt.plot(np.linspace(}\OperatorTok{{-}}\DecValTok{10}\NormalTok{, }\DecValTok{10}\NormalTok{, }\DecValTok{11}\NormalTok{), (wa1\_corr}\OperatorTok{{-}}\FloatTok{0.25}\NormalTok{)}\OperatorTok{*}\DecValTok{100}\NormalTok{, }\StringTok{\textquotesingle{}b{-}\textquotesingle{}}\NormalTok{, label}\OperatorTok{=}\StringTok{\textquotesingle{}Fehler im Korrelationskoeffizienten\textquotesingle{}}\NormalTok{)}

\NormalTok{plt.legend(loc}\OperatorTok{=}\DecValTok{4}\NormalTok{,  frameon}\OperatorTok{=}\VariableTok{True}\NormalTok{)}
\NormalTok{plt.xlabel(}\StringTok{\textquotesingle{}Ausmaß der Fehlschätzung (in \%)\textquotesingle{}}\NormalTok{)}
\NormalTok{ax.yaxis.set\_label\_coords(}\OperatorTok{{-}}\FloatTok{0.01}\NormalTok{,}\FloatTok{0.75}\NormalTok{)}
\NormalTok{ax.xaxis.set\_label\_coords(}\FloatTok{0.5}\NormalTok{,}\OperatorTok{{-}}\FloatTok{0.01}\NormalTok{)}

\NormalTok{plt.ylabel(}\StringTok{\textquotesingle{}Fehlgewichtung in Aktie 1 (in \%)\textquotesingle{}}\NormalTok{)}
\NormalTok{plt.title(}\StringTok{\textquotesingle{}Auswirkung von Schätzfehlern auf die optimalen Gewichte in Aktie 1\textquotesingle{}}\NormalTok{)}
\NormalTok{plt.show()}
\end{Highlighting}
\end{Shaded}

\pandocbounded{\includegraphics[keepaspectratio]{kapitel3_files/figure-pdf/cell-14-output-1.png}}

Die Abbildung zeigt deutlich, dass bereits kleine Fehler in den
Parameterschätzungen zu erheblichen Änderungen bei den optimierten
Gewichten führen. Dies gilt im besonderen für Schätzfehler bei den
erwarteten Renditen. Die Sensitivität der Portfoliozusammensetzung auf
Schätzfehler ist somit ein zentrales Problem bei der Anwendung des
Portfolioansatzes von Markowitz. Wie stark ein Anleger allerdings
hierdurch berührt ist, hängt außer von der Sensitivität von der
erzielbaren Genauigkeit bei der Schätzung der verschiedenen Parameter
ab. Wie groß die Schätzgenauigkeit für die verschiedenen Parameter ist,
wird im folgenden analysiert.

\subsubsection{Größe der Schätzfehler für die verschiedenen
Parameter}\label{gruxf6uxdfe-der-schuxe4tzfehler-fuxfcr-die-verschiedenen-parameter}

Aus historischen Daten sollen möglichst verläßliche Schätzwerte für die
erwartete annualisierte Rendite \((\mu_i)\) und die annualisierte
Standardabweichung \((\sigma_i)\) ermittelt werden. (Auf eine Analyse
für die Korrelation verzichten wir im folgenden, da - wie oben berichtet
- die Sensitivität der Portfoliozusammensetzung auf Fehlschätzungen der
Korrelation sehr gering ist.) Hierzu stehe eine Zeitreihe zur Verfügung,
die n (nicht annualisierte) Periodenrenditen \(r_{t,i}\) enthält, welche
unabhängig, identisch und normalverteilt seien. Ihre Verteilung ist
durch den Erwartungswert \(\mu_i\Delta t\) und die Varianz
\(\sigma^2_i\Delta t\) charakterisiert. Um den Schätzfehler zu
reduzieren, wird der Anleger versuchen, die Anzahl der Beobachtungen,
die in seine Schätzung eingehen, zu erhöhen. Dies kann er dadurch
erreichen, dass er den Schätzzeitraum \(T\) (gemessen in Jahren)
verlängert oder den Schätzzeitraum in kürzere Teilintervalle einteilt,
also \(\Delta t\) verkürzt.

Aus verschiedenen Gründen ist einem Anleger jedoch die Verlängerung des
Beobachtungszeitraums \(T\) häufig nicht möglich. Bei jungen Unternehmen
liegt keine lange Kurshistorie vor. Andere Unternehmen weisen zwar eine
längere Kurshistorie auf, jedoch sind diese durch Brüche gekennzeichnet,
die durch Umstrukturierungen oder Unternehmenszukäufe verursacht sind.
Schließlich ist gegen die Verwendung einer sehr langen Kurshistorie
einzuwenden, dass die Renditeparameter nicht über Jahrzehnte konstant
bleiben. Dagegen genießt der Anleger bei der Wahl der Länge der
Teilintervalle größere Freiheiten. Er kann von Jahresdaten auf
Quartals-, Monats-, Wochen-, Tagesdaten oder gar innertägliche Daten
wechseln, um so die Anzahl der Beobachtungen zu erhöhen. Im folgenden
wollen wir prüfen, wie sich dies auf die Qualität des Schätzers für die
erwartete Rendite und die Varianz niederschlägt. Hierzu müssen wir
zunächst eine Annahme treffen, welche Schätzer der Anleger verwendet.
Wir unterstellen, dass er wiederum die einfache historisch basierte
Schätzung verwendet, also das arithmetische Mittel für das erste
Verteilungsmoment. Unter Verwendung der Tatsache, dass die Schätzperiode
\(T\) sich ergibt als die Anzahl der Subperioden n multipliziert mit
deren Länge \(\Delta t\), lässt sich der Mittelwert-Schätzer schreiben
als:

\[ (1)\quad \hat{\mu_i}=\frac{1}{\Delta t}\frac{1}{n}\sum_{t=1}^{n}r_{t,i}=\frac{1}{T}\sum_{t=1}^{n}r_{t,i}.\]

Dieser Schätzer ist erwartungstreu, d.h., er trifft im Mittel den wahren
Parameterwert \(\mu_i\). Die Varianz des Schätzers ist ein Maß für
dessen Güte. Je geringer die Varianz ist, um so weniger streut der
Schätzer um den wahren Parameterwert. Die Varianz des Schätzers beträgt:

\[ (2)\quad var(\hat{\mu_i})=\frac{\sigma^2_i}{T}.\]

Bemerkenswerterweise hängt dieses Gütemaß bei gleichbleibendem
Beobachtungszeitraum \(T\) nicht von der Datenfrequenz ab. Dies
bedeutet, dass eine Erhöhung der Beobachtungsfrequenz nicht zu genaueren
Schätzergebnissen führt. Daneben fällt auf, dass der Schätzfehler im
Verhältnis zu der zu schätzenden Größe sehr groß ist. Dies kann man sich
verdeutlichen, wenn man erneut die Daten aus unserem Beispiel in
Abschnitt 1 verwendet. Die erwartete Rendite einer Aktie beträgt 11\%
p.a. und die Standardabweichung der Rendite 25\% p.a. In der folgenden
Tabelle wird die Breite eines 95\%-Konfidenzintervalls in Abhängigkeit
der Länge des Schätzzeitraums \(T\) dargestellt. Allgemein lautet die
Formel zur Längenberechnung des (asymptotischen)
\((1-\alpha)\)-Konfidenzintervalls:

\[ (3)\quad 2\cdot[Q_{1-\frac{\alpha}{2}}\cdot\frac{\sigma_i}{\sqrt{T}}].\]

Hierbei bezeichnet \(Q_{1-\frac{\alpha}{2}}\) das
\((1-\frac{\alpha}{2})\)-Quantil der Standardnormalverteilung. Für
\(\alpha=5\%\) gilt: \(Q_{97,5}=1,96\)

\begin{Shaded}
\begin{Highlighting}[]
\CommentTok{\# Länge des Schätzzeitraums T in Jahren}
\NormalTok{est\_period}\OperatorTok{=}\NormalTok{[}\DecValTok{1}\NormalTok{, }\DecValTok{5}\NormalTok{, }\DecValTok{10}\NormalTok{, }\DecValTok{20}\NormalTok{, }\DecValTok{50}\NormalTok{]}
\CommentTok{\# Breite des Konfidenzintervalls}
\NormalTok{confidence}\OperatorTok{=}\NormalTok{np.zeros(}\DecValTok{5}\NormalTok{)}
\ControlFlowTok{for}\NormalTok{ i }\KeywordTok{in} \BuiltInTok{range}\NormalTok{(}\DecValTok{5}\NormalTok{):}
\NormalTok{    confidence[i]}\OperatorTok{=}\BuiltInTok{round}\NormalTok{((}\DecValTok{2}\OperatorTok{*}\FloatTok{1.96}\OperatorTok{*}\FloatTok{0.25}\OperatorTok{/}\NormalTok{np.sqrt(est\_period[i]))}\OperatorTok{*}\DecValTok{100}\NormalTok{, }\DecValTok{2}\NormalTok{)}
\NormalTok{pd.DataFrame(\{}\StringTok{\textquotesingle{}Schätzzeitraum T in Jahren\textquotesingle{}}\NormalTok{: est\_period, }\OperatorTok{\textbackslash{}}
              \StringTok{\textquotesingle{}Breite des Konfidenzintervalls (\%)\textquotesingle{}}\NormalTok{: confidence\})    }
\end{Highlighting}
\end{Shaded}

\begin{longtable}[]{@{}lll@{}}
\toprule\noalign{}
& Schätzzeitraum T in Jahren & Breite des Konfidenzintervalls (\%) \\
\midrule\noalign{}
\endhead
\bottomrule\noalign{}
\endlastfoot
0 & 1 & 98.00 \\
1 & 5 & 43.83 \\
2 & 10 & 30.99 \\
3 & 20 & 21.91 \\
4 & 50 & 13.86 \\
\end{longtable}

Selbst bei einem Schätzzeitraum von 10 Jahren beträgt die Breite dieses
Intervalls noch mehr als 30 Prozentpunkte. Dies bedeutet, dass mit einer
Wahrscheinlichkeit von 95\% der Schätzfehler nicht mehr als 15,5
Prozentpunkte beträgt, der Schätzwert also im Intervall
{[}-4,5\%;+26,5\%{]} liegt. Wie wir aber oben gezeigt haben, besitzt
bereits eine deutlich geringere Fehlschätzung einen dramatischen Einfluß
auf die Portfoliozusammensetzung.

Nun wenden wir uns dem Schätzwert für die Varianz zu. Unter Verwendung
der definitorischen Beziehung \(T = n \cdot \Delta t\) lässt sich der
einfache historisch basierte Schätzer schreiben als:

\[ (4)\quad \hat{\sigma}^2_i=\frac{1}{T-\Delta t}\sum_{t=1}^{n}(r_{t,i}-\hat{\mu_i}\Delta t)^2.\]

Die asymptotische Varianz dieses Schätzers beträgt (siehe Memmel, 2004,
S. 33):

\[ (5)\quad var(\hat{\sigma}^2_i)=\frac{2\sigma^4_i\Delta t}{T}.\]

Die Güte des Varianzschätzers lässt sich somit bei gegebenem
Schätzzeitraum \(T\) durch eine Erhöhung der Datenfrequenz
(Verkleinerung von \(\Delta t\)) verbessern. So verdreifacht sich
beispielsweise durch Übergang von Quartalsdaten auf Monatsdaten die
Anzahl der Beobachtungen, und der Schätzfehler reduziert sich ungefähr
auf ein Drittel des ursprünglichen Wertes.

Die asymptotische Varianz des Schätzers für die
Renditestandardabweichung lässt sich bestimmen als :

\[ (6)\quad var(\hat{\sigma}_i)=\frac{1}{2}\sigma^2_i\frac{\Delta t}{T}.\]

In der folgenden Tabelle sind die Breiten des 95\%-Konfidenzintervalles
(in \%) für den Schätzer der Standardabweichung der annualisierten
Renditen für verschiedene Längen der Schätzperioden und verschiedene
Datenfrequenzen aufgeführt.

\begin{Shaded}
\begin{Highlighting}[]
\CommentTok{\# Länge des Schätzzeitraums T in Jahren}
\NormalTok{est\_period}\OperatorTok{=}\NormalTok{[}\DecValTok{1}\NormalTok{, }\DecValTok{5}\NormalTok{, }\DecValTok{10}\NormalTok{, }\DecValTok{20}\NormalTok{, }\DecValTok{50}\NormalTok{]}
\CommentTok{\# Breite des Konfidenzintervalls}
\NormalTok{confidence\_daily}\OperatorTok{=}\NormalTok{np.zeros(}\DecValTok{5}\NormalTok{) }\CommentTok{\# delta\_t=1/250}
\NormalTok{confidence\_weekly}\OperatorTok{=}\NormalTok{np.zeros(}\DecValTok{5}\NormalTok{) }\CommentTok{\# delta\_t=1/52}
\NormalTok{confidence\_monthly}\OperatorTok{=}\NormalTok{np.zeros(}\DecValTok{5}\NormalTok{) }\CommentTok{\# delta\_t=1/12}
\NormalTok{confidence\_quarterly}\OperatorTok{=}\NormalTok{np.zeros(}\DecValTok{5}\NormalTok{) }\CommentTok{\# delta\_t=1/4}

\ControlFlowTok{for}\NormalTok{ i }\KeywordTok{in} \BuiltInTok{range}\NormalTok{(}\DecValTok{5}\NormalTok{):}
\NormalTok{    confidence\_daily[i]}\OperatorTok{=}\BuiltInTok{round}\NormalTok{(}\DecValTok{2}\OperatorTok{*}\FloatTok{1.96}\OperatorTok{*}\NormalTok{np.sqrt(}\FloatTok{0.25}\OperatorTok{**}\DecValTok{2}\OperatorTok{/} \OperatorTok{\textbackslash{}}
\NormalTok{                    (}\DecValTok{2}\OperatorTok{*}\NormalTok{est\_period[i]}\OperatorTok{*}\DecValTok{250}\NormalTok{))}\OperatorTok{*}\DecValTok{100}\NormalTok{, }\DecValTok{2}\NormalTok{)}
\NormalTok{    confidence\_weekly[i]}\OperatorTok{=}\BuiltInTok{round}\NormalTok{(}\DecValTok{2}\OperatorTok{*}\FloatTok{1.96}\OperatorTok{*}\NormalTok{np.sqrt(}\FloatTok{0.25}\OperatorTok{**}\DecValTok{2}\OperatorTok{/} \OperatorTok{\textbackslash{}}
\NormalTok{                    (}\DecValTok{2}\OperatorTok{*}\NormalTok{est\_period[i]}\OperatorTok{*}\DecValTok{52}\NormalTok{))}\OperatorTok{*}\DecValTok{100}\NormalTok{, }\DecValTok{2}\NormalTok{)}
\NormalTok{    confidence\_monthly[i]}\OperatorTok{=}\BuiltInTok{round}\NormalTok{(}\DecValTok{2}\OperatorTok{*}\FloatTok{1.96}\OperatorTok{*}\NormalTok{np.sqrt(}\FloatTok{0.25}\OperatorTok{**}\DecValTok{2}\OperatorTok{/} \OperatorTok{\textbackslash{}}
\NormalTok{                    (}\DecValTok{2}\OperatorTok{*}\NormalTok{est\_period[i]}\OperatorTok{*}\DecValTok{12}\NormalTok{))}\OperatorTok{*}\DecValTok{100}\NormalTok{, }\DecValTok{2}\NormalTok{)}
\NormalTok{    confidence\_quarterly[i]}\OperatorTok{=}\BuiltInTok{round}\NormalTok{(}\DecValTok{2}\OperatorTok{*}\FloatTok{1.96}\OperatorTok{*}\NormalTok{np.sqrt(}\FloatTok{0.25}\OperatorTok{**}\DecValTok{2}\OperatorTok{/} \OperatorTok{\textbackslash{}}
\NormalTok{                    (}\DecValTok{2}\OperatorTok{*}\NormalTok{est\_period[i]}\OperatorTok{*}\DecValTok{4}\NormalTok{))}\OperatorTok{*}\DecValTok{100}\NormalTok{, }\DecValTok{2}\NormalTok{)}
\NormalTok{Zeitraum}\OperatorTok{=}\NormalTok{[}\StringTok{\textquotesingle{}T=1 Jahr\textquotesingle{}}\NormalTok{, }\StringTok{\textquotesingle{}T=5 Jahre\textquotesingle{}}\NormalTok{, }\StringTok{\textquotesingle{}T=10 Jahre\textquotesingle{}}\NormalTok{, }\StringTok{\textquotesingle{}T=20 Jahre\textquotesingle{}}\NormalTok{, }\StringTok{\textquotesingle{}T=50 Jahre\textquotesingle{}}\NormalTok{]    }
\NormalTok{pd.DataFrame(\{}\StringTok{\textquotesingle{}Tagesdaten\textquotesingle{}}\NormalTok{: confidence\_daily, }\StringTok{\textquotesingle{}Wochendaten\textquotesingle{}}\NormalTok{: confidence\_weekly,}\OperatorTok{\textbackslash{}}
              \StringTok{\textquotesingle{}Monatsdaten\textquotesingle{}}\NormalTok{: confidence\_monthly, }\StringTok{\textquotesingle{}Quartalsdaten\textquotesingle{}}\NormalTok{: confidence\_quarterly\},}\OperatorTok{\textbackslash{}}
\NormalTok{             index}\OperatorTok{=}\NormalTok{Zeitraum) }
\end{Highlighting}
\end{Shaded}

\begin{longtable}[]{@{}lllll@{}}
\toprule\noalign{}
& Tagesdaten & Wochendaten & Monatsdaten & Quartalsdaten \\
\midrule\noalign{}
\endhead
\bottomrule\noalign{}
\endlastfoot
T=1 Jahr & 4.38 & 9.61 & 20.00 & 34.65 \\
T=5 Jahre & 1.96 & 4.30 & 8.95 & 15.50 \\
T=10 Jahre & 1.39 & 3.04 & 6.33 & 10.96 \\
T=20 Jahre & 0.98 & 2.15 & 4.47 & 7.75 \\
T=50 Jahre & 0.62 & 1.36 & 2.83 & 4.90 \\
\end{longtable}

Im Vergleich mit der vorherigen Tabelle fällt auf, dass die
Standardabweichung wesentlich genauer geschätzt werden kann als die
erwartete Rendite. Dies gilt c.p. um so stärker, je größer die
Datenfrequenz ist. So kann unter Verwendung von Tagesdaten bereits bei
einer Schätzperiode von einem Jahr eine recht präzise Varianzschätzung
vorgenommen werden, während für diese Frist die erwartete Rendite nur
mit einer hohen Ungenauigkeit geschätzt werden kann. Die bisher
gewonnenen Ergebnisse scheinen damit zu implizieren, dass ein Anleger
bei gegebener Schätzperiode eine möglichst hohe Datenfrequenz wählen
sollte, da er hierdurch gemäß (5) den Schätzfehler in der Varianz
verringert und den Schätzfehler in der erwarteten Rendite unbeeinflußt
läßt. Dieser Aussage ist allerdings nicht uneingeschränkt zuzustimmen.
Eine sehr hohe Datenfrequenz bringt in der Praxis nämlich das Problem
mit sich, dass die empirisch beobachteten Renditen nicht mehr
unabhängig, identisch und normalverteilt sind. So findet man
beispielsweise, dass Aktienrenditen bei einer hohen Datenfrequenz (bspw.
Tagesdaten) Autokorrelation aufweisen und ihre Verteilung zu viel Masse
in den Enden aufweist.

Zusammenfassend kann man damit festhalten: Die Schätzung der erwarteten
Renditen verursacht wesentlich größere Probleme als die Schätzung der
Renditevarianz. Die Schätzung der erwarteten Renditen kann nämlich nicht
durch eine Erhohung der Datenfrequenz verbessert werden, während eine
höhere Datenfrequenz zu besseren Schätzungen der Renditevarianz führt.
Da außerdem die Portfoliozusammensetzung wesentlich sensitiver auf
Fehlschätzungen in den erwarteten Renditen als in den Varianzen
reagiert, muss konstatiert werden, dass Schätzfehler in erwarteten
Renditen das zentrale Problem bei der Implementierung des
Markowitz-Modells darstellen. Deshalb konzentrieren wir uns im folgenden
ausschließlich auf Schätzfehler in den erwarteten Renditen und
unterstellen, dass die zweiten Momente der Verteilung (Varianzen,
Korrelationen) dem Anleger bekannt sind.

\subsection{4. Lösungsansätze für das
Schätzproblem}\label{luxf6sungsansuxe4tze-fuxfcr-das-schuxe4tzproblem}

In den bisherigen Abschnitten haben wir herausgearbeitet, warum die
traditionelle Umsetzung der Portfoliotheorie Probleme aufwirft.
Erwartete Renditen lassen sich mit dem traditionellen Schätzansatz des
arithmetischen Mittels nur sehr ungenau schätzen. Dies führt
typischerweise zu extremen und suboptimalen Portfoliogewichten.

Grundsätzlich bieten sich mehrere Möglichkeiten an, dieses Problems Herr
zu werden. Eine erste Möglichkeit besteht darin, ein ökonomisch
fundiertes Modell zu entwickeln, aus dem sich endogen die optimalen
Portfoliogewichte ergeben und das es erlaubt, auf leichter zu
ermittelnde Größen anstelle der erwarteten Renditen zurückzugreifen. Ein
Beispiel hierfür stellt das CAPM dar, bei dem die optimalen
Portfoliogewichte den anteiligen Marktkapitalisierungen entsprechen.

Eine zweite Möglichkeit, den Einfluss von Schätzfehlern und die daraus
resultierenden extremen Portfoliogewichte zu reduzieren, besteht darin,
\textbf{exogene Schranken (Constraints)} für die erwarteten Renditen
bzw. die Portfoliogewichte vorzugeben. Den ersten Ansatz wählt Merton
(1980), der unterstellt, dass die wahren erwarteten Renditen in einer
Welt mit risikoaversen Anlegern nicht negativ sein können und
gleichzeitig einen bestimmten Höchstwert nicht überschreiten sollten.
Ein einfaches Beispiel für exogene Schranken bezüglich der
Portfoliogewichte - diese sind bereits Bestandteil der Originalarbeit
von Markowitz - besteht im Verbot von Leerverkaufen. Hierdurch werden
die Portfoliogewichte in jeder Aktie auf den Bereich {[}0\%; 100\%{]}
restringiert. Die optimale Strategie besteht hierbei häufig aus
Randlösungen, d.h., in den meisten Fällen konzentriert sich das
Anlagevolumen auf wenige Aktien, die sich in der jüngsten Vergangenheit
besonders gut entwickelt haben (Winner-Strategie). Hierdurch geht
Diverifikationspotential innerhalb des Portfolios verloren. Um diesem
Mißstand abzuhelfen, werden z.B. im Rahmen der relativen (Index- oder
Benchmark-basierten) Optimierung die Abweichungen der Portfolioanteile
von den Gewichten eines Index- oder Benchmarkportfolios begrenzt. Als
Folge der erzwungenen, aber nicht optimalen Diversifikation reduziert
sich zwar das Portfoliorisiko, doch leidet gleichzeitig die
Wertentwicklung dieses Portfolios.

Die dritte Möglichkeit besteht darin, gänzlich auf die Schätzung von
erwarteten Renditen zu verzichten. So könnte ein Anleger beispielsweise
das \textbf{global varianzminimale Aktienportfolio} oder ein
\textbf{gleichgewichtetes Portfolio} wählen. Diese
Portfoliozusammensetzung muss als \emph{heuristisch} bezeichnet werden,
da sie sich nicht aus einem allgemeinen Markowitz-Ansatz ergibt.
Trotzdem haben verschiedene empirische Studien (z.B. DeMiguel et al.,
2009) ergeben, dass mit diesen Strategien bessere Ergebnisse erzielt
werden können als mit der traditionellen Umsetzung der Portfoliotheorie.

Neben diesen klassischen heuristischen Ansätzen haben sich seit der
Finanzkrise 2008-2009 zunehmend risikogesteuerte Ansätze etabliert. Die
Portfoliobildung basiert auch hier allein auf Volatilitäts- und
Korrelationsannahmen und die Diversifikation wird in den Mittelpunkt
gestellt. Prognosen erwarteter Renditen spielen keine Rolle und werden
nicht benötigt. Drei Methoden einer risikogesteuerten
Portfoliokonstruktion werden unterschieden: \textbf{Equal-Risk-Budget
(ERB), Equal-Risk-Contribution (ERC)} und
\textbf{Maximum-Diversification (MD)}, wobei die ersten Beiden als
sogenannte \textbf{Risk Parity} Ansätze bezeichnet werden. Alle drei
Ansätze sind methodisch eng verwandt. Im Grundsatz soll jederzeit mit
vergleichbarem Risiko in alle Assetklassen investiert werden. Das
Gewicht einer Assetklasse wird dabei reduziert, wenn ihre Volatilität
oder ihre Korrelation zu einer anderen Assetklasse steigt.

Die vierte Möglichkeit zur Reduzierung des Einflusses von Schätzrisiken
in den erwarteten Renditen besteht darin, den Schätzer für erwartete
Renditen zu verbessern und diesen verbesserten Schätzwert dann als
Inputvariable für die Portfoliooptimierung zu verwenden. Eine beliebte
Klasse solcher verbesserter Schätzer sowohl für \(\mu\) als auch
\(\Sigma\) stellen die sogenannten \textbf{geschrumpften Schätzer} dar.

Die fünfte Möglichkeit besteht schließlich darin, (wie bei der vierten
Möglichkeit) einen verbesserten Schätzer für die erwarteten Renditen zu
verwenden, aber zusätzlich noch bei der Portfoliooptimierung die
Existenz des verbleibenden Schätzrisikos zu berücksichtigen. In diesem
Ansatz wird also explizit berücksichtigt, dass der Anleger zwei Arten
von Risiken ausgesetzt ist, dem Renditeänderungsrisiko und dem
Schätzrisiko. Zwei bekannte Vertreter dieses Ansatzes sind die
\textbf{Portfolio-Resampling} Methode von Michaud und Michaud (2008),
und das \textbf{Black/Litterman (BL)} Modell (siehe Black und Litterman,
1992).

Die folgende Abbildung untergliedert noch einmal die oben skizzierten
Lösungsansätze nach den zugrundeliegenden Ansatzpunkten.

\pandocbounded{\includegraphics[keepaspectratio]{figure32.png}}

\subsection{Literatur}\label{literatur-2}

Black, F., Litterman R. (1992). Global Portfolio Optimization. Financial
Analysts Journal (September-October), 28-43.

DeMiguel, V., Garlappi, L., Uppa, R. (2009). Optimal Versus Naive
Diversification: How Inefficient is the 1/N Portfolio Strategy? Review
of Financial Studies 22, 1915-1953.

Markowitz, H.M. (1952). Portfolio Selection. Journal of Finance 7,
77-91.

Merton, R.C. (1980). On Estimating the Expected Return on the Market: An
Exploratory Investigation. Journal of Financial Economics 8, 323-361.

Michaud, R.O, Michaud, R.O., (2008). Efficient Asset Management: A
Practical Guide to Stock Portfolio Optimization and Asset Allocation,
2nd Edition, Oxford University Press.

Tobin, J. (1958). Liquidity Preference as Behaviour Towards Risk. Review
of Economic Studies 25, 65-86.




\end{document}
